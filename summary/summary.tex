\chapter*{Summary}
\chaptermark{}
\addcontentsline{toc}{chapter}{Summary}

{\Large\bf
  Measurement of CP Violation in Mixing and Decay of Strange Beauty Mesons
}
\vspace*{0.05\textwidth}

\noindent
Despite its precise and accurate description of elementary-particle interactions, the Standard Model of Particle Physics has several
shortcomings. To find a more complete description of nature, particle interactions are tested for deviations from Standard Model
predictions, which would indicate how to extend the model. The LHCb experiment at \cern's Large Hadron Collider searches for such
deviations in the description of particle decays.

In particular, the LHCb experiment studies decays of the bound states formed by beauty and strange quarks, or ``strange beauty mesons''.
The combination of antibeauty and strange is a $\Bs$ meson, beauty and antistrange form the corresponding antiparticle, denoted by
$\Bsbar$.

An important feature of these particles is that they can turn into each other, which creates a mixed system of a particle and its
antiparticle. Starting with a $\Bs$ meson, the particle evolves and can be either a $\Bs$ or a $\Bsbar$ meson at the time it decays into
other particles. The other way around, a particle created as a $\Bsbar$ meson also has a probability to decay as a $\Bs$ meson.

A particularly interesting mode of decay is that into a $\Jpsi$ meson and a $\phimes$ meson, which is possible for both $\Bs$ and $\Bsbar$.
For this mode there are two possible decay paths for a given initial particle, which is either $\Bs$ or $\Bsbar$. In one path the initial
particle first turns into its antiparticle an then decays into the $\Jpsiphi$ final state and in the other path the initial particle decays
directly into this state.

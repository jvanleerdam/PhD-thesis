\chapter*{Summary}
\chaptermark{}
\addcontentsline{toc}{chapter}{Summary}

{\Large\bf
  Measurement of CP Violation in Mixing and Decay of Strange Beauty Mesons
}
\vspace*{0.05\textwidth}

\noindent
Despite its precise and accurate description of elementary-particle interactions, the Standard Model of Particle Physics has several
shortcomings. To find a more complete description of nature, particle interactions are tested for deviations from Standard Model
predictions, which would indicate how to extend the model. The LHCb experiment at \cern's Large Hadron Collider searches for such
deviations in the description of particle decays.

In particular, the LHCb experiment studies the decay of the bound states formed by beauty and strange quarks, or ``strange beauty mesons''.
The combination of antibeauty and strange is a $\Bs$ meson, beauty and antistrange form the corresponding antiparticle, denoted by
$\Bsbar$.

An important feature of these particles is that they can turn into each other, which creates a mixed system of a particle and its
antiparticle. Starting with a $\Bs$ meson, the particle evolves and can be either a $\Bs$ or a $\Bsbar$ meson at the time it decays into
other particles. Similarly, there is a probability that a particle created as a $\Bsbar$ meson decays as a $\Bs$ meson.

A particularly interesting mode of decay is that into a $\Jpsi$ meson and a $\phimes$ meson, which occurs for both $\Bs$ and $\Bsbar$. For
this mode there are two possible decay paths for each of the two initial particles. In one path the initial particle first turns into its
antiparticle an then decays into the $\Jpsiphi$ final state and in the other path the initial particle decays directly into this state.

This mixing and decay process is predicted to be almost identical for $\Bs$ and $\Bsbar$ within the Standard Model framework. Measurements
indicate that the difference in the rates for the transitions from $\Bs$ to $\Bsbar$ and vice versa are very small. Also the rates of the
decays into the $\Jpsiphi$ state are expected to be nearly equal. This equivalence between matter and antimatter is known as \emph{CP
symmetry}.

Contributions from particle interactions that are not described by the Standard Model may increase the amount of violation of CP symmetry
in the \BstoJpsiphi{} process. In particular, a difference between the complex phases of the probability amplitudes for the \BsBsbar{}
mixing transitions may be introduced. In general this does not lead to CP violation, since the rate of a process only depends on the
magnitude of the corresponding probability amplitude. In this case, however, the amplitude is a sum of the interfering contributions from
the two decay paths. The relative phases of these contributions do affect the magnitude of the sum, leading to an observable difference
between the \BstoJpsiphi{} and \BsbartoJpsiphi{} processes.

This type of CP violation in the interference between decay paths with and without mixing is measured by examining the distribution of the
time between the production and the decay of $\Bs$ and/or $\Bsbar$ mesons. Without CP violation, this distribution is given by the sum of
two exponential contributions with slightly different mean lifetimes. CP violation introduces an oscillation on top of this exponential
shape with an amplitude of opposite sign for initial $\Bs$ and $\Bsbar$ mesons.

In the LHCb experiment, $\Bs$ and $\Bsbar$ mesons are abundantly produced in roughly equal amounts in the proton--proton collisions of the
Large Hadron Collider. Decays into $\Jpsiphi$ followed by decays of the $\Jpsi$ meson into two muons and the $\phimesalt$ meson into two
kaons are selected by requiring the signature of these muons and kaons in the detector is compatible with this decay chain.

The produced $\Bs$ and $\Bsbar$ mesons have a mean lifetime of about 1.5\unitsp{}ps, which means typical distances of several millimetres
are covered before their decay. These distances are measured by determining the positions of the proton--proton collision and the common
point of origin of the muons and kaons from the decay. Also including the measurement of the combined momentum of the decay particles, the
time between production and decay of the original meson is inferred.

The shape of the decay-time distribution is modelled and the resulting model is fitted to the measured distribution to determine the values
of its parameters. Parameters that describe CP violation determine the amplitude of the oscillation in decay time. The frequency of the
oscillation and the lifetimes of the two exponential shapes are controlled by parameters that describe the coupled \BsBsbar{} system.

Different types of CP violation are included in the decay model, which are measured individually for the three different angular-momentum
states of the $\Jpsiphi$ system for the first time in this measurement. Small differences between the contributions of these states are
expected, which become important in a precision measurement of CP violation. The different contributions are separated by including the
measurement of the angles between the momentum directions of the four final-state particles. This results in a four-dimensional
distribution of the decay time and three decay angles.

To describe the measured distribution of these variables, experimental effects such as detection and selection efficiencies and finite
measurement resolutions are included in the model. Also the fact that the measured distribution is a sum of $\Bs$ and $\Bsbar$ decays is
taken into account. Uncertainties in the estimates of these experimental effects lead to systematic uncertainties in the estimated
decay-time and CP-violation parameters, in addition to the statistical uncertainties associated with the size of the sample of decays.

The measured distribution of time and angles is constructed from roughly ninety thousand decays, collected in the years 2011 and 2012.
Estimates of the parameter values with these data are compatible with Standard Model predictions, given the experimental uncertainties.
These results show that potential non-Standard Model contributions to the \BstoJpsiphi{} mixing and decay process must be smaller than the
current experimental precision.

An improvement in precision of an order of magnitude is expected with future data from the LHCb experiment, which provides new
opportunities for measuring deviations from the Standard Model with \BstoJpsiphi{} decays. The measurement with this larger sample of
decays requires some improvements in the experimental procedure, to keep systematic uncertainties smaller than the statistical
uncertainties.

Adopting the new strategy of measuring CP violation individually for the different angular-momentum states of the $\Jpsiphi$ system would
enable interpretation of future precision measurements within a framework of measurements and theoretical calculations of several different
meson decays. Such a combined analysis is likely to be required to overcome limitations in the theoretical predictions of CP-violation
parameters. Combining these experimental and theoretical tools, the measurement of CP violation in mixing and decay of strange beauty
mesons has the potential to continue playing an important role in the search for a more complete description of nature.

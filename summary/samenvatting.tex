\chapter*{Samenvatting}
\chaptermark{}
\addcontentsline{toc}{chapter}{Samenvatting}

{\Large\bf
  Meting van CP schending in het mixen en vervallen van vreemde schoonheidsmesonen
}
\vspace*{0.05\textwidth}

\noindent
Ondanks de nauwkeurige en juiste beschrijving van interacties tussen elementaire deeltjes die het geeft, heeft het Standaardmodel van de
deeltjesfysica meerdere tekortkomingen. Om een completere beschrijving van de natuur te vinden, worden deeltjesinteracties getest op
afwijkingen van voorspellingen door het Standaardmodel, welke een indicatie zouden geven van hoe het model moet worden uitgebreid. Het LHCb
experiment bij de ``Large Hadron Collider'' van \cern{} zoekt naar dit soort afwijkingen in de beschrijving van het verval van deeltjes.

In het bijzonder worden in het LHCb experiment vervallen van gebonden toestanden van schoonheid- en vreemdquarks, of ``vreemde
schoonheidsmesonen'' bestudeerd. De combinatie van antischoonheid en vreemd is een $\Bs$ meson, terwijl schoonheid en antivreemd het
bijbehorende antideeltje vormen, aangeduid met $\Bsbar$.

Een belangrijke eigenschap van deze deeltjes is dat ze in elkaar kunnen overgaan. Dit geeft een gemixt systeem van een deeltje en zijn
antideeltje. Een $\Bs$ meson evolueert in de tijd en kan of een $\Bs$ of een $\Bsbar$ meson zijn op het moment dat het vervalt in andere
deeltjes. Op dezelfde manier kan een deeltje dat is geproduceerd als $\Bsbar$ meson vervallen als een $\Bs$ meson.

Met name de modus van verval in een $\Jpsi$ meson en een $\phimes$ meson is interessant. Deze modus komt zowel voor $\Bs$ als $\Bsbar$
voor. Er zijn daardoor twee mogelijke vervalpaden voor ieder van de twee aanvankelijke deeltjes. In een van de paden gaat het vervallende
deeltje eerst over in zijn antideeltje en vervalt dan naar de $\Jpsiphi$ eindtoestand. In het andere pad vervalt het aanvankelijke deeltje
direct naar deze toestand.

Het Standaardmodel voorspelt dat dit proces van mixen en vervallen bijna gelijk verloopt voor $\Bs$ en $\Bsbar$. Metingen laten zien dat
het verschil in de snelheden van de overgangen van $\Bs$ naar $\Bsbar$ en vice versa erg klein is. Ook de snelheden van de vervallen naar
de $\Jpsiphi$ toestand zijn naar verwachting bijna gelijk. Deze overeenkomst tussen materie en antimaterie staat bekend als
\emph{CP-symmetrie}.

Bijdragen van deeltjesinteracties die niet worden beschreven door het Standaardmodel zouden de mate waarin CP-symmetrie wordt geschonden
in het \BstoJpsiphi{} proces kunnen vergroten. Met name zou er een verschil kunnen ontstaan tussen de complexe fases van de
waarschijnlijkheidsamplitudes van de \BsBsbar{} mixovergangen. In het algemeen leidt dit niet tot CP schending, omdat de waarschijnlijkheid
van een proces alleen afhangt van de absolute waarde van de bijbehorende waarschijnlijkheidsamplitude. In dit geval is de amplitude echter
een som van de interfererende bijdragen van de twee vervalspaden. De relatieve fases van deze bijdragen hebben wel invloed op de absolute
waarde van de som en dit leidt tot een waarneembaar verschil tussen de \BstoJpsiphi{} and \BsbartoJpsiphi{} processen.

Deze vorm van CP schending in de interferentie tussen vervalspaden met en zonder mixen wordt gemeten door de verdeling van de tijd tussen
de productie en het verval van $\Bs$ en/of $\Bsbar$ mesonen te bestuderen. Zonder CP schending is deze verdeling gegeven door de som van
twee exponenti\"ele bijdragen met een klein verschil in gemiddelde levensduur. CP schending introduceert een oscillatie op deze
exponenti\"ele vorm met een amplitude die een tegengesteld teken heeft voor aanvankelijke $\Bs$ en $\Bsbar$ mesonen.

In het LHCb experiment worden $\Bs$ en $\Bsbar$ mesonen in overvloed en in gelijke hoeveelheden geproduceerd in de proton--proton botsingen
van de ``Large Hadron Collider''. Vervallen naar $\Jpsiphi$ gevolgd door vervallen van het $\Jpsi$ meson naar twee muonen en het
$\phimesalt$ meson naar twee kaonen worden geselecteerd door te eisen dat het patroon van deze muonen en kaonen in de detector consistent
is met deze vervalsketen.

De geproduceerde $\Bs$ en $\Bsbar$ mesonen hebben een gemiddelde levensduur van ongeveer 1.5\unitsp{}ps, waardoor ze typische afstanden van
enkele millimeters afgeleggen voor hun verval. Deze afstanden worden gemeten door de posities van de proton--proton botsing en het
gezamenlijke punt van oorsprong van de muonen en kaonen te bepalen. Door ook de meting van de gecombineerde impuls van de vervalsdeeltjes
mee te nemen, kan de tijd tussen productie en verval van het oorspronkelijke meson worden afgeleid.

De vorm van de verdeling van vervalstijden wordt gemodelleerd en het resulterende model wordt gefit aan de gemeten verdeling om
parameterwaarden te bepalen. Parameters die CP schending beschrijven, bepalen de amplitude van de oscillatie in de vervalstijd. De
frequentie van de oscillatie en de levensduren van de twee exponenti\"ele vervallen worden bepaald door parameters die het gekoppelde
\BsBsbar{} systeem beschrijven.

Het vervalsmodel beschrijft verschillende vormen van CP schending, die in deze meting voor het eerst individueel worden gemeten voor de
drie verschillende impulsmomenttoestanden van het $\Jpsiphi$ systeem. Kleine verschillen tussen de bijdragen van deze toestanden worden
verwacht, die belangrijk worden in een precisiemeting van CP schending. De verschillende bijdragen worden gescheiden door de meting van de
hoeken tussen de impulsrichtingen van de vier deeltjes in de eindtoestand mee te nemen in het model. Dit geeft een vierdimensionale
verdeling van de vervalstijd en drie vervalshoeken.

Om de gemeten verdeling van deze variabelen te kunnen beschrijven, worden experimentele effecten als detectie- en selectie-effici\"enties
en eindige resoluties van metingen meegenomen in het model. Ook met het feit dat de waargenomen verdeling een som van $\Bs$ en $\Bsbar$
vervallen wordt rekening gehouden. Onzekerheden in de inschattingen van deze experimentele effecten leiden tot systematische onzekerheden
in de parameterschattingen, naast de statistische onzekerheden die worden geassocieerd met de grootte van de verzameling vervallen.

De gemeten verdeling van tijd en hoeken is opgebouwd uit ongeveer negentigduizend vervallen die verzameld zijn in de jaren 2011 en 2012.
Schattingen van de parameterwaarden met deze data komen overeen met de voorspellingen van het Standaardmodel, gegeven de experimentele
onzekerheden. Deze resultaten laten zien dat potenti\"ele bijdragen van buiten het Standaardmodel aan het mixen en vervallen in het
\BstoJpsiphi{} proces kleiner zijn dan de huidige experimentele nauwkeurigheid.

Een verbetering in nauwkeurigheid van een orde van grootte wordt verwacht met toekomstige data van het LHCb experiment. Dit geeft nieuwe
mogelijkheden voor het meten van afwijkingen van voorspellingen van het Standaardmodel met \BstoJpsiphi{} vervallen. De meting met deze
grotere verzameling vervallen vereist enige verbeteringen in de experimentele procedure om systematische onzekerheden kleiner te houden dan
statistische onzekerheden.

Het overnemen van de nieuwe strategie van het individueel meten van CP schending voor de verschillende impulsmomenttoestanden van het
$\Jpsiphi$ systeem zou het mogelijk maken om toekomstige precisiemetingen te interpreteren in een raamwerk van metingen en theoretische
berekeningen van verschillende meson vervallen. Dit soort gecombineerde analyse is waarschijnlijk nodig om te kunnen omgaan met beperkingen
in theoretische voorspellingen van CP schendingparameters. Door het combineren van deze experimentele en theoretische analyses kan de
meting van CP schending in het mixen en vervallen van vreemde schoonheidsmesonen potentieel een belangrijke rol blijven spelen in de
zoektocht naar een completere beschrijving van de natuur.

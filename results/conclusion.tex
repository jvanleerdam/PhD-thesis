\section{Summary and Outlook}
\label{sec:result_conclusion}

Estimates of the principal parameters are given in Table~\ref{tab:result_paramEst_final}. Even though not all parameter distributions
follow a Gaussian shape, numbers for both statistical and systematic uncertainties are given. As pointed out in
Section~\ref{sec:result_paramEst}, all results are statistically compatible with Standard Model predictions.
\begin{table}[htbp]
  \centering
  \caption{Estimates of the principal parameters.}
  \label{tab:result_paramEst_final}
  %\renewcommand{\arraystretch}{0.8}
  \begin{tabular}{llll}
    \hline
    parameter  &  value  &  \multicolumn{2}{l}{uncertainty}  \\
               &         &  statistical  &  systematic       \\
    \hline
    $\phisav$~[rad]              &  \tm0.05           &  \tpm0.05                        &  \tpm0.01           \\
    $\Delphispara$~[rad]         &  \tm0.02           &  \tpm0.04                        &  \tpm0.02           \\
    $\Delphisperpp$~[rad]        &  \tm0.00           &  \tpm0.03                        &  \tpm0.01           \\
    $\DelphisS$~[rad]            &   +0.01            &  \tpm0.06                        &  \tpm0.02           \\
    \hline
    $\Csav$                      &  \tm0.01           &  \tpm0.04                        &  \tpm0.01           \\
    $\DelCspara$                 &  \tm0.02           &  \tpm0.12                        &  \tpm0.06           \\
    $\DelCsperp$                 &   +0.04            &  \tpm0.16                        &  \tpm0.02           \\
    $\CsavS$                     &   +0.06            &  \tpm0.03                        &  \tpm0.02           \\
    \hline
    $\Gs$~[\invps]               &  \phantom{+}0.659  &  \tpm0.003                       &  \tpm0.001          \\
    $\DGs$~[\invps]              &   +0.078           &  \tpm0.009                       &  \tpm0.003          \\
    $\Dms$~[\invps]              &  \phantom{+}17.70  &  \tpm0.06                        &  \tpm0.02           \\
    \hline
    $\magzeroAvSq$               &  \phantom{+}0.524  &  \tpm0.003                       &  \tpm0.007          \\
    $\magperpAvSq$               &  \phantom{+}0.251  &  \tpm0.005                       &  \tpm0.003          \\
    $\delparzero$~[rad]          &   +3.25            &  $^\text{+0.10}_\text{\tm0.20}$  &  \tpm0.07           \\
    $\delperpzero$~[rad]         &   +3.04            &  $^\text{+0.16}_\text{\tm0.18}$  &  \tpm0.06           \\
    \hline
  \end{tabular}
\end{table}

For some parameters the systematic uncertainty is comparable to or even larger than the statistical uncertainty. With roughly a factor ten
increase in number of decay candidates this would be true for all parameters, which creates a need for improvement of the systematics.

It is possible to reduce some of the largest systematic errors by implementing different
data-analysis methods, in particular for the handling of background and including the decay-time acceptance.

A possible way of reducing the systematic from the dependence of the $\JpsiKK$-mass model on $\cthetal$ would be a background subtraction
in intervals of this variable. Alternatively, introducing the estimate of the uncertainty in the $\JpsiKK$-mass measurement in both the
mass and time/angular models could provide a way to describe the correlations between mass and angles.

The systematic uncertainty resulting from \BctoBsX{} decays could be reduced by including a component for this contribution in the
resolution model for the decay time.

The uncertainties from decay-time reconstruction acceptance dominate the systematics for the parameters $\Gs$, $\DGs$, and $\Dms$. For this
reason a different approach may be required for future measurements with a larger dataset. Further study is required to evaluate the exact
shape of the acceptance function and either implement it in the PDF or use it to correct the data.

Angular acceptance statistical: more simulated decays.\\
Angular acceptance simulation: understand what is going on in the simulation.\\
Angular resolution: Do a better job of estimating resolution (also correlations between angles), finally include resolution in model.

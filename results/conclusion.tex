\section{Summary and Outlook}
\label{sec:result_conclusion}

Estimates of the principal parameters are given in Table~\ref{tab:result_paramEst_final}. As pointed out in
Section~\ref{sec:result_paramEst}, all results are statistically compatible with Standard Model predictions.
\begin{table}[htbp]
  \centering
  \caption{Estimates of the principal parameters.}
  \label{tab:result_paramEst_final}
  %\renewcommand{\arraystretch}{0.8}
  \begin{tabular}{llll}
    \hline
    parameter  &  value  &  \multicolumn{2}{l}{uncertainty}  \\
               &         &  statistical  &  systematic       \\
    \hline
    $\phisav$~[rad]              &  \tm0.05           &  \tpm0.05                        &  \tpm0.01           \\
    $\Delphispara$~[rad]         &  \tm0.02           &  \tpm0.04                        &  \tpm0.02           \\
    $\Delphisperpp$~[rad]        &  \tm0.00           &  \tpm0.03                        &  \tpm0.01           \\
    $\DelphisS$~[rad]            &   +0.01            &  \tpm0.06                        &  \tpm0.02           \\
    \hline
    $\Csav$                      &  \tm0.01           &  \tpm0.04                        &  \tpm0.01           \\
    $\DelCspara$                 &  \tm0.02           &  \tpm0.12                        &  \tpm0.06           \\
    $\DelCsperp$                 &   +0.04            &  \tpm0.16                        &  \tpm0.02           \\
    $\CsavS$                     &   +0.06            &  \tpm0.03                        &  \tpm0.02           \\
    \hline
    $\Gs$~[\invps]               &  \phantom{+}0.659  &  \tpm0.003                       &  \tpm0.001          \\
    $\DGs$~[\invps]              &   +0.078           &  \tpm0.009                       &  \tpm0.003          \\
    $\Dms$~[\invps]              &  \phantom{+}17.70  &  \tpm0.06                        &  \tpm0.02           \\
    \hline
    $\magzeroAvSq$               &  \phantom{+}0.524  &  \tpm0.003                       &  \tpm0.007          \\
    $\magperpAvSq$               &  \phantom{+}0.251  &  \tpm0.005                       &  \tpm0.003          \\
    $\delparzero$~[rad]          &   +3.25            &  $^\text{+0.10}_\text{\tm0.20}$  &  \tpm0.07           \\
    $\delperpzero$~[rad]         &   +3.04            &  $^\text{+0.16}_\text{\tm0.18}$  &  \tpm0.06           \\
    \hline
  \end{tabular}
\end{table}

For some parameters the systematic uncertainty is comparable to or even larger than the statistical uncertainty. With roughly a factor ten
increase in number of decay candidates this would be true for all parameters, which creates a need for reduction of the systematics.

A systematic with a significant contribution for most of the CP-violation parameters (Tables~\ref{tab:systErrsPhases} and
\ref{tab:systErrsMixDecay}) is the uncertainty from decay-angle resolution. If this effect is going to be included as a systematic
uncertainty in further measurements, the study of the resolution and its effect on the parameter estimates will have to be improved.
Instead of describing the resolution separately for the three decay angles, a multidimensional resolution function can be constructed to
include correlations. Further studies can also evaluate the effect on all parameters of the $\phisi$/$\Csi$ model.

When the systematic from angular resolution resolution becomes dominating, the possibility of including this effect in the model of decay
time and angles must be considered. It will be challenging to implement this, since not all functions can be analytically convolved with
the angular functions in the expression for the differential decay rate. Either functions should be found for which this is possible or
numerical convolution methods should be explored.

Also uncertainties in the angular acceptance contribute to the systematics of the CP-violation parameters. The statistical component of
these uncertainties can be reduced by generating more simulated decays. However, this will only emphasize the fact that the simulation does
not perfectly describe real decays. If the simulation is to be used for future measurements, it has to be improved to provide a more
accurate description of the production, decay, and detection of particles in LHCb events.

Another potential method of obtaining the angular acceptance would be to study detection efficiencies in real data particles and decays for
which the underlying physics description is known. This is also an option for the reconstruction component of the decay-time
acceptance, which dominates the systematic uncertainties of the $\Bs$-lifetime parameters (Table~\ref{tab:systErrsLifetime}). For the
latter the shape of the acceptance function is already partially determined with real data.

An improved description of the reconstruction decay-time acceptance is already available and was used here to evaluate a systematic
uncertainty and in reference~\cite{LHCb-PAPER-2014-059} as the nominal description. Although this method provides a better description of
the acceptance shape, it still introduces the dominant systematic uncertainties on the lifetime parameters and further study of this
acceptance effect is required for further measurements.

Removing the uncertainty from decay-time acceptance leaves only the effect of \BctoBsX{} decays for the parameter $\Gs$. This uncertainty
could be reduced by including a component for this contribution in the resolution model for the decay time, since these are real
\BstoJpsiKK{} decays with a modified decay-time measurement. Remaining uncertainties would then originate from the estimates of the number
of contributing \BctoBsX{} decays and the shape of the resolution component.

Dominating contributions to systematic uncertainties for some of the CP-violation parameters and all of the transversity amplitudes
(Table~\ref{tab:systErrsAmps}) are the uncertainties introduced by the model of $\JpsiKK$ mass. The part originating from resonant
backgrounds can be reduced by improving the accuracy of the description of the distributions of these backgrounds in the relevant
variables. This can be achieved by improving the simulation of background decays, exploiting analyses of these background distributions in
real data, or a combination of these methods.

The dominant $\JpsiKK$-mass systematics, however, originate from the background-subtraction procedure. A possible way of reducing the
systematic from the non-factorization of the mass and time/angles models would be background subtraction in intervals of $\cthetal$, as was
done in this analysis to evaluate the systematic uncertainties. Alternatively, introducing the estimate of the uncertainty in the
$\JpsiKK$-mass measurement in both the models could provide a way to describe the correlations between mass and angles. This could be
implemented in combination with a background-subtraction procedure, but an alternative is to fit unweighted \BstoJpsiKK{} decays with a
five-dimensional model of mass, decay time, and decay angles.

In addition to reducing systematic uncertainties, the precision of the CP-violation measurement can potentially be improved by reducing the
number of wrong flavour tags. Given a wrong-tag probability of 40\%, a 10\% improvement of this value gives almost a factor two increase in
the effective number of perfectly tagged decays (see also Section~\ref{subsec:ana_tagging_impl}). Improvements of the flavour-tagging
procedure may include the use of more sophisticated (multivariate) techniques to select and analyse tagging particles, the development of
additional tagging algorithms, and an improved understanding of the properties of particles that are created in association with $\Bs$
mesons.

\section{Systematic Uncertainties}
\label{sec:result_syst}

Systematic uncertainties in the parameter estimates arise from uncertainties in the models used to describe the distributions of invariant
masses, decay time, and decay angles. They are evaluated by repeating the fit of decay time and decay angles with varied values of relevant
nuisance parameters or with manipulated data.

The value of some of the nuisance parameters can be determined with the \BstoJpsiKK{} data that are used in the time and angular fit. In
these cases the value of the parameter is varied in the fit, often constrained by external information. Uncertainties originating from
these parameters are therefore absorbed in the statistical uncertainties quoted in Section~\ref{sec:result_paramEst}.

As described in Section~\ref{subsec:ana_time_acc}, the parameters of the trigger decay-time acceptance function are fully determined from
the \BstoJpsiKK{} data and constrained in the fit by counting decay candidates in the different trigger categories. The externally
determined exponential factors for the reconstruction decay-time acceptance in 2011 and 2012 are combined with the information on their
difference from the decay-time fit by varying these nuisance parameters within Gaussian constraints.

The resulting contributions of the decay-time acceptance uncertainties are estimated by repeating the fit with fixed acceptance parameters
and subtracting the statistical uncertainties with varying and with fixed parameters in quadrature. Only the uncertainty of $\Gs$ is
significantly affected. The contribution from the trigger acceptance is 0.0011~\invps{} and the contribution from the reconstruction
acceptance 0.0015~\invps.

Other nuisance parameters that are varied within Gaussian constraints are the parameters that describe the flavour-tagging calibration, as
discussed in Section~\ref{sec:ana_tagging}. Fixing the tagging parameters does not significantly affect any of the uncertainty estimates
for other parameters.

Systematic effects that are not propagated to the statistical uncertainties are listed in Tables~\ref{tab:systErrsPhases},
\ref{tab:systErrsMixDecay}, \ref{tab:systErrsLifetime}, and \ref{tab:systErrsAmps} for the principal parameters. Also a total uncertainty
is shown for each parameter, determined by adding the statistical and systematic uncertainties in quadrature. The effects for the three
CP-violation parameterizations are found to be similar and for most parameters only the results with the $\phisi$/$\Csi$ model are shown.
Systematic uncertainties for $\phis$ are assumed to be the same as those for $\phisav$ and uncertainties for $\lamsAbs$ are shown in
Table~\ref{tab:systErrsMixDecay}.

\begin{table}[p]
  \centering
  \caption{Systematic uncertainties for the CP-violating phases.}
  \label{tab:systErrsPhases}
  \begin{tabular}{lllll}
    \hline
                             &  $\phisav$  &  $\Delphispara$  &  $\Delphisperpp$  &  $\DelphisS$  \\
                             &  [rad]      &  [rad]           &  [rad]            &  [rad]        \\
    \hline
    \textit{$\JpsiKK$ mass}  &  &  &  &  \\
    resonant                 &  0.003      &  0.001           &  0.001            &  0.009        \\
    factorization            &  0.003      &  0.005           &  0.001            &  0.016        \\[3pt]
    \textit{$\KK$ mass}      &  &  &  &  \\
    integrals                &  \ctm       &  \ctm            &  \ctm             &  0.006        \\[3pt]
    \textit{decay time}      &  &  &  &  \\
    resolution               &  0.003      &  0.002           &  0.001            &  0.001        \\[3pt]
    \textit{decay angles}    &  &  &  &  \\
    resolution               &  0.004      &  0.014           &  0.009            &  0.007        \\
    acc. statistical         &  0.002      &  0.007           &  0.004            &  0.004        \\
    acc. simulation          &  \ctm       &  0.002           &  0.001            &  0.007        \\
    \hline
    total systematic         &  0.007      &  0.017           &  0.010            &  0.022        \\
    \hline
    statistical              &  0.051      &  0.043           &  0.029            &  0.062        \\
    total                    &  0.05       &  0.05            &  0.03             &  0.07         \\
    \hline
  \end{tabular}
\end{table}

\begin{table}[p]
  \centering
  \caption{Systematic uncertainties for the asymmetries from CP-violation in mixing and in decay.
           Uncertainties arising from $\KK$-mass integrals and decay-time resolution are negligible for these parameters.}
  \label{tab:systErrsMixDecay}
  \begin{tabular}{llllll}
    \hline
                             &  $\lamsAbs$  &  $\Csav$  &  $\DelCspara$  &  $\DelCsperp$  &  $\CsavS$  \\
    \hline                                  
    \textit{$\JpsiKK$ mass}  &  &  &  &  \\
    resonant                 &  0.0028      &  0.001    &  0.008         &  0.004         &  0.004     \\
    factorization            &  0.0008      &  0.001    &  0.042         &  0.011         &  0.014     \\[3pt]
    \textit{decay angles}    &  &  &  &  \\
    resolution               &  0.0014      &  0.004    &  0.030         &  0.011         &  0.010     \\
    acc. statistical         &  0.0024      &  0.001    &  0.015         &  0.006         &  0.005     \\
    acc. simulation          &  0.0048      &  0.001    &  0.003         &  0.001         &  0.011     \\
    \hline                                  
    total systematic         &  0.0063      &  0.005    &  0.055         &  0.018         &  0.021     \\
    \hline
    statistical              &  0.0188      &  0.039    &  0.122         &  0.162         &  0.032     \\
    total                    &  0.020       &  0.04     &  0.13          &  0.16          &  0.04      \\
    \hline
  \end{tabular}
\end{table}

\begin{table}[p]
  \centering
  \caption{Systematic uncertainties for the lifetime parameters.}
  \label{tab:systErrsLifetime}
  \begin{tabular}{llll}
    \hline
                             &  $\Gs$     &  $\DGs$    &  $\Dms$    \\
                             &  [\invps]  &  [\invps]  &  [\invps]  \\
    \hline
    \textit{$\JpsiKK$ mass}  &  &  &  \\
    resonant                 &  \ctm      &  0.0002    &  0.001     \\
    factorization            &  \ctm      &  0.0006    &  0.002     \\[3pt]
    \textit{decay time}      &  &  &  \\
    resolution               &  \ctm      &  \ctm      &  0.006     \\
    \BctoBsX                 &  0.0005    &  \ctm      &  \ctm      \\
    reco. acc. model         &  0.0013    &  0.0023    &  0.012     \\[3pt]
    \textit{decay angles}    &  &  &  \\
    resolution               &  \ctm      &  0.0008    &  0.006     \\
    acc. statistical         &  \ctm      &  0.0001    &  0.001     \\
    acc. simulation          &  \ctm      &  0.0001    &  0.001     \\
    \hline
    total systematic         &  0.0014    &  0.0025    &  0.015     \\
    \hline
    statistical              &  0.0033    &  0.0092    &  0.060     \\
    total                    &  0.004     &  0.010     &  0.06      \\
    \hline
  \end{tabular}
\end{table}

\begin{table}[p]
  \centering
  \caption{Systematic uncertainties for the transversity amplitudes.}
  \label{tab:systErrsAmps}
  \begin{tabular}{lllll}
    \hline
                             &  $\magzeroAvSq$  &  $\magperpAvSq$  &  $\delparzero$  &  $\delperpzero$  \\
                             &                  &                  &  [rad]          &  [rad]           \\
    \hline
    \textit{$\JpsiKK$ mass}  &  &  &  &  \\
    resonant                 &  0.0002          &  0.0003          &  0.024          &  0.011           \\
    factorization            &  0.0064          &  0.0030          &  0.050          &  0.048           \\[3pt]
    \textit{$\KK$ mass}      &  &  &  &  \\
    integrals                &  \ctm            &  \ctm            &  0.003          &  0.007           \\[3pt]
    \textit{decay time}      &  &  &  &  \\
    resolution               &  \ctm            &  \ctm            &  0.004          &  0.008           \\[3pt]
    \textit{decay angles}    &  &  &  &  \\
    resolution               &  0.0002          &  0.0009          &  0.036          &  0.022           \\
    acc. statistical         &  0.0005          &  0.0007          &  0.019          &  0.009           \\
    acc. simulation          &  0.0020          &  0.0011          &  0.006          &  0.002           \\
    \hline
    total systematic         &  0.0067          &  0.0034          &  0.069          &  0.056           \\
    \hline
    statistical              &  0.0034          &  0.0049          &  0.132          &  0.165           \\
    total                    &  0.008           &  0.006           &  0.15           &  0.17            \\
    \hline
  \end{tabular}
\end{table}

\begin{description}
\item[$\JpsiKK$-mass model: resonant backgrounds]
Resonant backgrounds from $\Bd$ and $\Lb$ decays are subtracted by injecting simulated events with negative weights into the \BstoJpsiKK{}
data sample, as described in Section~\ref{sec:ana_bkgSub}. Two types of uncertainties arise from this procedure: uncertainties in the
estimates of the numbers of background events that affect the \BstoJpsiKK{} measurement and uncertainties in the distributions of the
relevant variables for the reflection backgrounds.

A systematic uncertainty for the numbers of background events is estimated by changing the subtracted background yields by one standard
deviation. The (absolute) variations resulting from an upward and a downward fluctuation are averaged. The yields of all background
components are varied simultaneously.

The distributions of the decay angles and tagging variables of simulated background events are reweighted to make them match the
distributions in $\Bd$ and $\Lb$ data as closely as possible. Since the results of this procedure are not expected to be perfect, the
difference in results with datasets containing reweighted and not reweighted distributions is taken as a systematic uncertainty. The
resulting numbers are added in quadrature to the average of the yield uncertainty for the total systematics arising from resonant
backgrounds. The total uncertainties are dominated by the results of the reweighting procedure.

\item[$\JpsiKK$-mass model: statistical]
Signal weights for \BstoJpsiKK{} candidates are determined with the $\JpsiKK$-mass model after a fit to the mass distribution in data
(Section~\ref{sec:ana_bkgSub}). Statistical uncertainties in the signal and the combinatorial background mass models are propagated by
repeating the time and angular fit five thousand times with different sets of signal weights. The sets of weights are obtained by
calculating \sweight[s] with different sets of mass-model parameters, which are generated according to a multivariate-Gaussian distribution
using the means and covariances from the mass fit. Effects on the parameter estimates from the time and angular fit are found to be
negligible.

\item[$\JpsiKK$-mass model: factorization]
The $\JpsiKK$-mass model depends on the angle $\thetal$, as shown in Section~\ref{sec:ana_bkgSub}. Since the background-subtraction
procedure relies on the absence of correlations between the mass and other variables, this dependence leads to a systematic uncertainty. A
possible way of removing this systematic would be a background subtraction in bins of the variable $\cthetal$, which could be considered
for future measurements. In this measurement the subtraction in bins is only used to evaluate a systematic uncertainty.

The $\JpsiKK$-mass model is found to be symmetric with respect to the point $\cthetal$\texteq0. Signal weights are calculated separately
for three $\cthetal$ bins defined as $|\cthetal|$\textlt0.25, 0.25\textle$|\cthetal|$\textlt0.7, and 0.7\textle$|\cthetal|$, where the
binning is motivated by the differences in the mass distribution. The differences between the parameter estimates from fits with these
signal weights and the nominal weights are taken as a systematic uncertainty. This uncertainty is the dominating systematic for the $\Csi$
and transversity-amplitude parameters, which emphasizes the need to reconsider the current strategy for future measurements.

\item[$\KK$-mass model: bin integrals]
The expression for the differential decay rate is integrated over the $\KK$ mass in six bins. As discussed in
Section~\ref{sec:pheno_KKMass}, this leads to the $\CSP$ factors in the $\Jpsiphi$--$\KK$ S-wave interference terms, which represent the
ratios of the bin integrals for the two contributions. The $\CSP$ factors depend on the $\KK$-mass models used in their calculation and a
systematic uncertainty arises from uncertainties in the $\Jpsiphi$ and S-wave line shapes.

Two uncertainties are considered. The first originates from a finite $\KK$-mass resolution, which is estimated with simulated events. An
alternative set of $\CSP$ factor is calculated with a resolution that is 20\% larger than the nominal estimate (second column in
Table~\ref{tab:CSPFactors}). To enhance the significance of the effect of resolution, the differences in $\CSP$ factors with the nominal
set are doubled before repeating the time and angular fit and the resulting deviations in the parameter estimates are divided by two for
the corresponding systematic uncertainties.

The second uncertainty comes from the model of the $\KK$ S-wave. Nominally it is assumed that the S-wave consists only of a $\Jpsi\,\fzero$
contribution, which is modelled with a Flatt\'e function. However, the true composition of the S-wave is not exactly known. This
uncertainty is evaluated by repeating the fit with $\CSP$ factors calculated with a uniform line shape in $\KK$ mass (third column in
Table~\ref{tab:CSPFactors}). Deviations in the parameter estimates are added in quadrature to the resolution uncertainty.

\item[Decay-time model: resolution]
In Section~\ref{subsec:ana_time_res} the calibration procedure for the decay-time resolution was described. Resolution parameters are
determined for prompt background candidates with zero decay time and it is assumed that these parameter values are applicable to the
resolution of signal candidates. The systematic uncertainty associated with this assumption is evaluated with simulated events. See
reference~\cite{Aaij:2014} for a detailed description of this procedure.

In simulation the differences between the resolution-model parameters for signal candidates and prompt background are determined. With
these differences a transformation between the two models is constructed, which is applied to the measured real-data parameters to obtain
estimates of the signal-model parameters for real data. The difference in results between fits with the signal and background
resolution models for real data is taken as a systematic uncertainty.

\item[Decay-time model: \BctoBsX{} decays]
A fraction of $\Bs$ mesons in the \BstoJpsiKK{} sample originates from \BctoBsX{} decays, where \textit{X} is a charged particle or
combination of particles, for instance a pion. The $\Bc$ meson has a mean lifetime of approximately 0.5\unitsp{}ps~\cite{PDG}, resulting in
a decay at a significant distance from the primary vertex where the $\Bc$ is produced. Because it is assumed that the $\Bs$ is produced in
the primary vertex in the reconstruction of its decay time, the measured decay time for the \BctoBsX{} contribution is wrong and leads to a
systematic uncertainty. In future measurements this systematic uncertainty could be reduced by including a \BctoBsX{} component in the
resolution model for the decay time.

The contribution of \BctoBsX{} decays was estimated to be 0.8\% of the \BstoJpsiKK{} signal sample~\cite{LHCb-ANA-2014-039}. The
corresponding systematic uncertainty was estimated by generating pseudo experiments, injecting simulated \BctoBsJpsiKKX{} into each
generated data sample. The mean deviations in the resulting parameter estimates are taken as systematic uncertainties. Only the estimate of
$\Gs$ is significantly affected.

\item[Decay-time model: reconstruction acceptance]
The acceptance arising from reconstruction is modelled as a function with an exponential shape in decay time, as described in
Section~\ref{subsec:ana_time_acc}. This model only describes the acceptance function approximately, which gives systematic uncertainties in
the estimates of the lifetime parameters.

The uncertainties are evaluated with the differences between fit results on a data sample that is corrected for the reconstruction
acceptance and the nominal data sample. Data are corrected for acceptance by assigning decay-candidate weights that are inversely
proportional to the estimated acceptance. The weights are calculated with the shape of the acceptance in the reconstruction variables of a
candidate. Although this acceptance shape is still an approximation, it describes the exact function significantly better than an
exponential shape~\cite{LHCb-ANA-2014-039}.

The resulting uncertainties dominate the systematics for the parameters $\Gs$, $\DGs$, and $\Dms$. For this reason a different approach may
be required for future measurements with a larger dataset. Further study is required to evaluate the exact shape of the acceptance function
and either implement it in the PDF or use it to correct the data.

\item[Decay-time model: \BsBsbar{} normalization asymmetries]
In the nominal analysis the absence of all time-independent \BsBsbar{} asymmetries is assumed. That is, the assumption is made that there
is no asymmetry between the $\Bs$ and $\Bsbar$ decay-rate equations from the factor $1-\qf\,\Cm$ (Equation~\ref{eq:timeqfDep}), no
production asymmetry between $\Bs$ and $\Bsbar$, and no asymmetry between the numbers of $\Bs$ and $\Bsbar$ in each tagging category. The
assumption is expected to have no effect on the parameter estimates, as explained in Section~\ref{sec:ana_tagging}. This is verified by
generating ten thousand pseudo experiments with non-zero values for the normalization asymmetries and comparing the resulting parameter
distributions with the distributions in the nominal pseudo experiments. No significant differences are found.

Since no precise values of the normalization asymmetries are available, values for each pseudo experiment are drawn from Gaussian
distributions that represent the current estimates of the asymmetries. The uncertainties in the asymmetry-parameter estimates propagate to
the widths of the parameter and pull distributions in the pseudo experiments.

The asymmetry $\Cm$\textapprox$\tfrac{\text{1}}{\text{2}}\,\afs$ is estimated with the current average of measurements of CP-violation in
mixing, $\afs$\texteq\tm0.0109\textpm0.0040~\cite{Amhis:2012bh}. The uncertainty in the production asymmetry is estimated with a
measurement of the $\Bd$ production asymmetry in LHCb, 0.006\textpm0.009~\cite{LHCb-PAPER-2013-040}, where the $\Bs$ asymmetry is expected
to be smaller than the $\Bd$ asymmetry. As an estimate of the uncertainty for the $\Bs$ asymmetry the sum of the value and uncertainty is
taken and the production asymmetry is generated as 0\textpm0.15. Uncertainties in the tagging-efficiency differences for $\Bs$ and $\Bsbar$
are estimated in tagging calibration~\cite{LHCb-ANA-2014-039} and lead to generated values of 0\textpm0.0015 for OS tagging and
\texteq0\textpm0.0046 for SS tagging.

Tagging-efficiency asymmetries for OS-untagged and SS-untagged candidates are equal but opposite to the asymmetry in the corresponding
tagged category, weighted by the ratio of the tagged and untagged efficiencies. As a consequence, the OS and SS tagging efficiencies are
required to calculate the untagged asymmetries. Assuming small asymmetries, the average of the $\Bs$ and $\Bsbar$ efficiencies is estimated
by the fraction of candidates in a category, yielding $\tagCatCoef[\text{OS}]$\texteq0.3076\textpm0.0017 and
$\tagCatCoef[\text{OS}]$\texteq0.5291\textpm0.0018.

Defining tagging efficiencies for the OS and SS categories implicitly assumes factorization for these categories.  That is, the efficiency
for doubly-tagged candidates is given by $\tagCatCoef[\text{OS}]\,\tagCatCoef[\text{SS}]$, the efficiency for candidates only tagged by the
OS algorithms by $\tagCatCoef[\text{OS}]$\,(1\textminus$\tagCatCoef[\text{SS}]$), and so on. Calculating the OS efficiency separately for
SS-tagged and SS-untagged candidates and vice versa indicates that factorization is not perfect. To account for this half of the difference
between the two efficiency estimates for each category is added in quadrature to the statistical uncertainties, which finally gives
$\tagCatCoef[\text{OS}]$\texteq0.308\textpm0.013 and $\tagCatCoef[\text{OS}]$\texteq0.529\textpm0.016.

\item[Decay-angles model: resolution]
The finite experimental resolution of the decay angles is not accounted for in the decay model. A study of the effects of a finite
resolution is described in reference~\cite{LHCb-ANA-2014-039}. The resolution was estimated with simulated events and found to be of the
order 0.01\unitsp{}rad. Its effect on the parameter estimates was studied by generating pseudo experiments and smearing the values of the
decay angles according to the resolution distributions found in the simulation. Effects were quantified by the distribution of the
difference in pull values for the fit results with and without angle smearing for each pseudo experiment. The mean and the width of the
pull-difference distribution are multiplied by the statistical uncertainty from the real-data fit for each parameter and added in
quadrature to obtain the systematic uncertainty associated to decay-angle resolution.

Unfortunately the study of angular resolution was only performed for the $\phis$/$\lamsAbs$ parameterization. Therefore, the systematic
uncertainties for the CP-violation parameters in the remaining two models have to be estimated in a different way. The systematic
for $\phisav$ is taken to be equal to the estimate for $\phis$. The parameter $\Csav$ appears in similar terms of the PDF as $\phisav$ and,
therefore, the systematic for $\Csav$ is also estimated to be equal to the systematic of $\phis$.

The parameters that describe the differences between the $\phisi$ and $\Csi$ parameters are measured from the interference terms in the
PDF, as are the phase differences $\delparzero$ and $\delperpzero$. To estimate the effect of decay-angle resolution on the CP-violation
parameters, the effects of decay-angle resolution and decay-angle acceptance are compared for $\delparzero$ and $\delperpzero$. The impact
of the propagation of statistical uncertainties in the acceptance, which will be described next, is roughly half of the impact of
resolution for these parameters. To get an estimate for the resolution uncertainty in the CP-violation parameters it is assumed that this
factor of two also applies here.

\item[Decay-angles model: acceptance statistical]
A non-trivial acceptance shape in the decay angles is included in the decay model with the normalization weights that were introduced in
Section~\ref{sec:ana_angles}. The values of the normalization weights are estimated with simulated events. Because of the finite size of
the simulated sample the estimates are affected by statistical uncertainties, which are propagated to the final parameter estimates. The
time and angular fit is repeated for five thousand sets of varied normalization weights, generated according to a multivariate Gaussian
distribution. The square roots of the parameter-estimate variances are taken as systematic uncertainties.

\item[Decay-angles model: acceptance simulation]
Kinematic distributions of simulated events do not match the distributions in real data perfectly. Although corrections are applied to the
simulated distributions before calculating the angular-acceptance weights (see Section~\ref{sec:ana_angles}), the discrepancies in the
simulation are not fully understood and an associated systematic uncertainty remains. The size of this effect is estimated with the
difference between the fit results with weights from a corrected and an uncorrected simulation.
\end{description}

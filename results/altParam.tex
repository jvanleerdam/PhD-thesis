\section{Alternative Parameterizations}
\label{sec:result_altParam}

To justify some of the modelling choices made in Chapter~\ref{chap:ana}, the time and angular fit is repeated with alternative
parameterizations. The effects of an external constraint on the estimate of $\Dms$, a narrower $\KK$-mass window, and flavour-tagging
categories are studied. It is expected that the fits with alternative models yield results that are comparable to or less precise than the
results from the nominal fit.

The parameters of the lower decay-time acceptance function are fixed to their nominal vales in all alternative fits, since this
considerably reduces the time that the fits take. As a result, the estimated statistical uncertainty on $\Gs$ decreases. The fit with the
nominal model yields $\Gs$\texteq0.6591\textpm0.0031\unitsp\invps.


%%%%%%%%%%%%%%%%%%%%%%%%%%%%%%%%%%%%%%%%%%%%%%%%%%
\subsection{Constrained Mass-Difference Parameter}
\label{subsec:result_altParam_Delm}
%%%%%%%%%%%%%%%%%%%%%%%%%%%%%%%%%%%%%%%%%%%%%%%%%%

Currently the most precise estimate of $\Dms$ is given by 17.768\textpm0.024\unitsp\invps{} and comes from an LHCb analysis of
\BstoDsmpip{} decays~\cite{LHCb-PAPER-2013-006}. This estimate can be used to constrain the value of $\Dms$ in the \BstoJpsiKK{} fit, as
was done for previous measurements~\cite{LHCb-PAPER-2011-021,*LHCb-ANA-2011-036,LHCb-PAPER-2013-002,*LHCb-ANA-2012-067}. The constraint is
implemented by adding a parabolic term to the NLL.

The value of the $\Dms$ constraint is about one standard deviation larger than the nominal-fit estimate
($\Dms$\texteq17.697\textpm0.062\unitsp\invps) and its uncertainty is a factor 2.6 smaller. As a result, the estimates of the parameters
that are correlated to $\Dms$ are expected to deviate from the nominal result. Especially the estimate of $\delperpzero$, for which the
correlation with $\Dms$ is 70\%, will be significantly affected.

Table~\ref{tab:result_DelM_polarDep} lists the results of the fit with $\Dms$ constrained. Estimates of the parameter values are given in
the second column from the left. The third column contains the differences of these estimates with respect to the values from the nominal
fit. The uncertainty estimates are given in the fourth column and the relative change in the uncertainties in the fifth column.
\begin{table}[htbp]
  \centering
  \caption{Results of a fit with the estimate of $\Dms$ constrained to an external measurement
           ($\Dms$\texteq17.768\textpm0.024\unitsp\invps~\cite{LHCb-PAPER-2013-006}).}
  \label{tab:result_DelM_polarDep}
  \begin{tabular}{lllll}
    \hline
    parameter        &  \multicolumn{2}{l}{estimate (difference)}              &  \multicolumn{2}{l}{uncertainty (rel. difference)} \\
    \hline
    $\phisav$        &  \tm0.051           &  (\tm0.004)   &  0.050            &  (\tm0.02)                                         \\
    $\Delphispara$   &  \tm0.023           &  (\tm0.004)   &  0.041            &  (\tm0.04)                                         \\
    $\Delphisperpp$  &  \tm0.003           &  (\tm0.001)   &  0.024            &  (\tm0.15)                                         \\
    $\DelphisS$      &    +0.000           &  (\tm0.014)   &  0.064            &    (+0.03)                                         \\
    \hline
    $\Csav$          &  \tm0.004           &    (+0.010)   &  0.038            &  (\tm0.01)                                         \\
    $\DelCspara$     &  \tm0.010           &    (+0.015)   &  0.122            &  (\tm)                                             \\
    $\DelCsperp$     &    +0.035           &  (\tm0.009)   &  0.164            &    (+0.01)                                         \\
    $\CsavS$         &    +0.060           &  (\tm)        &  0.030            &  (\tm0.05)                                         \\
    \hline
    $\Gs$            &  \phantom{+}0.6591  &  (\tm)        &  0.0031           &  (\tm)                                             \\
    $\DGs$           &   +0.0783           &  (\tm0.0001)  &  0.0092           &  (\tm)                                             \\
    $\Dms$           &  \phantom{+}17.758  &    (+0.062)   &  0.022            &  (\tm0.64)                                         \\
    \hline
    $\magzeroAvSq$   &  \phantom{+}0.5236  &  (\tm0.0001)  &  0.0034           &  (\tm)                                             \\
    $\magperpAvSq$   &  \phantom{+}0.2512  &  (\tm0.0001)  &  0.0049           &  (\tm)                                             \\
    $\delparzero$    &   +3.255            &    (+0.009)   &  +0.100 \tm0.177  &  (\tm0.04 \tm0.12)                                 \\
    $\delperpzero$   &   +3.154            &    (+0.118)   &  +0.120 \tm0.125  &  (\tm0.25 \tm0.29)                                 \\
    \hline
  \end{tabular}
\end{table}

The estimates of the value and uncertainty of $\Dms$ agree with what is expected from a combination of the internal and external estimates.
Also the value and uncertainty of $\delperpzero$ change as expected. The decrease of the uncertainty of the $\Delphisperpp$ estimate might
be somewhat surprising, since there is no correlation between $\Delphisperpp$ and $\Dms$ given in
Table~\ref{tab:result_paramEst_correlations}. However, the likelihood shape in $\Delphisperpp$
(Figure~\ref{fig:NLL_CPV_phases_phiCPRel_AperpApar}) is not Gaussian, so the linear covariance matrix is not sufficient to predict the
changes for this parameter. The similarity of these results to the nominal results justifies omission of the $\Dms$ constraint in the fit,
although this choice may have to be re-evaluated for future analyses, depending on the $\Dms$ correlations with a larger dataset.


%%%%%%%%%%%%%%%%%%%%%%%%%%%%%%%%%%%%%%%%%%%%%%%%%%%%%%%%%%
\subsection{Narrow \texorpdfstring{$\KK$}{KK}-Mass Window}
\label{subsec:result_altParam_KKMass}
%%%%%%%%%%%%%%%%%%%%%%%%%%%%%%%%%%%%%%%%%%%%%%%%%%%%%%%%%%

Since the main interest of this measurement is CP violation in the \BstoJpsiphi{} decay, one could argue to limit the range of $\KK$ masses
to the region where the $\phimes$ significantly contributes. This is a region of approximately 24~\MeV{} around the centre of the
$\phimesalt$ peak at 1020~\MeV{} (see also Figure~\ref{fig:KKMassBkgSub}).

Results of a fit in a $\KK$-mass window of [1008, 1032]~\MeV{} are shown in Table~\ref{tab:result_KKMass4_polarDep}. The window was split
up into the four intervals of the nominal analysis. The structure of the table is the same as for Table~\ref{tab:result_DelM_polarDep}.

Uncertainties of the parameter estimates increase with respect to the nominal fit, because the information on the $\KK$ S-wave in the [990,
1008]~\MeV{} and [1032, 1050]~\MeV{} intervals is lost. Parameters which are (partially) determined with the S-wave (interference) terms
are affected. From these results it can be seen that the two outer $\KK$-mass intervals add valuable information for the CP-violation
measurement.

%\begin{table}[htbp]
%  \centering
%  \caption{Results of a fit in a $\KK$-mass window of [1008, 1032]~\MeV{} with one interval.}
%  \label{tab:result_KKMass1_polarDep}
%  \begin{tabular}{lllll}
%    \hline
%    parameter        &  \multicolumn{2}{l}{estimate (difference)}              &  \multicolumn{2}{l}{uncertainty (rel. difference)} \\
%    \hline
%    $\phisav$       &  \tm0.046           &    (+0.001)   &  +0.072 \tm0.058  &  (+0.41 +0.14)                                      \\
%    $\Delphispara$  &  \tm0.004           &    (+0.015)   &  +0.059 \tm0.055  &  (+0.38 +0.28)                                      \\
%    $\Delphisperpp$ &  \tm0.011           &  (\tm0.008)   &  +0.034 \tm0.054  &  (+0.20 +0.87)                                      \\
%    $\DelphisS$     &  \tm0.054           &  (\tm0.069)   &  +1.66  \tm1.57   &  (+1.3  +24)                                        \\
%    \hline
%    $\Csav$         &  \tm0.008           &  (\tm0.001)   &  +0.040 \tm0.048  &  (+0.04 +0.23)                                      \\
%    $\DelCspara$    &  \tm0.013           &    (+0.012)   &  +0.175 \tm0.161  &  (+0.44 +0.32)                                      \\
%    $\DelCsperp$    &   +0.036            &  (\tm0.007)   &  0.169            &  (+0.04)                                            \\
%    $\CsavS$        &   +0.071            &    (+0.012)   &  +0.073 \tm0.174  &  (+1.3  +4.4)                                       \\
%    \hline
%    $\Gs$           &  \phantom{+}0.6577  &  (\tm0.0014)  &  0.0032           &  (+0.04)                                            \\
%    $\DGs$          &   +0.0815           &   (+0.0030)   &  0.0095           &  (+0.04)                                            \\
%    $\Dms$          &  \phantom{+}17.682  &  (\tm0.015)   &  +0.071 \tm0.093  &  (+0.15 +0.51)                                      \\
%    \hline
%    $\magzeroAvSq$  &  \phantom{+}0.5247  &    (+0.0011)  &  0.0036           &  (+0.04)                                            \\
%    $\magperpAvSq$  &  \phantom{+}0.2512  &    (\tm)      &  0.0053           &  (+0.08)                                            \\
%    $\delparzero$   &   +3.248            &    (+0.002)   &  +0.126 \tm0.373  &  (+0.22 +0.85)                                      \\
%    $\delperpzero$  &   +2.997            &  (\tm0.039)   &  +0.200 \tm0.372  &  (+0.25 +1.1)                                       \\
%    \hline
%  \end{tabular}
%\end{table}
\begin{table}[htbp]
  \centering
  \caption{Results of a fit in the $\KK$-mass window of [1008, 1032]~\MeV{} with four intervals.}
  \label{tab:result_KKMass4_polarDep}
  \begin{tabular}{lllll}
    \hline
    parameter        &  \multicolumn{2}{l}{estimate (difference)}  &  \multicolumn{2}{l}{uncertainty (rel. difference)} \\
    \hline
    $\phisav$       &  \tm0.039           &    (+0.007)            &  0.053            &  (+0.05)                       \\
    $\Delphispara$  &  \tm0.044           &  (\tm0.025)            &  +0.053 \tm0.058  &  (+0.24 +0.36)                 \\
    $\Delphisperpp$ &  \tm0.024           &  (\tm0.021)            &  +0.040 \tm0.045  &  (+0.40 +0.56)                 \\
    $\DelphisS$     &    +0.033           &    (+0.018)            &  +0.098 \tm0.117  &  (+0.57 +0.89)                 \\
    \hline
    $\Csav$         &  \tm0.018           &  (\tm0.011)            &  0.039            &  (\tm)                         \\
    $\DelCspara$    &  \tm0.104           &    (+0.129)            &  +0.150 \tm0.140  &  (+0.23 +0.15)                 \\
    $\DelCsperp$    &  \tm0.013           &  (\tm0.056)            &  0.166            &  (+0.02)                       \\
    $\CsavS$        &    +0.094           &    (+0.034)            &  +0.806 \tm0.063  &  (+24 +0.95)                   \\
    \hline
    $\Gs$           &  \phantom{+}0.6572  &  (\tm0.0019)           &  0.0032           &  (+0.04)                       \\
    $\DGs$          &   +0.0821           &    (+0.0036)           &  0.0094           &  (+0.02)                       \\
    $\Dms$          &  \phantom{+}17.665  &  (\tm0.031)            &  0.073            &  (+0.17)                       \\
    \hline
    $\magzeroAvSq$  &  \phantom{+}0.5252  &  (+0.0016)             &  0.0035           &  (+0.02)                       \\
    $\magperpAvSq$  &  \phantom{+}0.2496  &  (\tm0.0016)           &  0.0051           &  (+0.04)                       \\
    $\delparzero$   &   +2.980            &  (\tm0.266)            &  +0.333 \tm0.108  &  (+2.2 \tm0.46)                \\
    $\delperpzero$  &   +2.794            &  (\tm0.242)            &  +0.282 \tm0.195  &  (+0.76 +0.10)                 \\
    \hline
  \end{tabular}
\end{table}


%%%%%%%%%%%%%%%%%%%%%%%%%%%%%%%%%%%%%%%
\subsection{Flavour-Tagging Categories}
\label{subsec:result_altParam_tagCats}
%%%%%%%%%%%%%%%%%%%%%%%%%%%%%%%%%%%%%%%

The flavour-tagging scheme with an estimated wrong-tag probability for each decay candidate is expected to give more precise
results than a scheme with an average wrong-tag probability. To test this assumption the data are fitted using three alternative
flavour-tagging parameterizations.

Table~\ref{tab:result_untagged_polarDep} shows the results of a fit in which no flavour-tagging information is used. Neglecting \BsBsbar{}
nuisance asymmetries, the $\cDms$ and $\sDms$ terms vanish from the PDF used in this untagged fit and only the $\cDGs$ and $\sDGs$ terms
remain (see Section~\ref{sec:ana_tagging}). As a result, there is no sensitivity for the parameter $\Dms$ in this fit, but also the
uncertainties in the estimates of other parameters increase significantly.

Table~\ref{tab:timeFunctionsNoCPV} shows that with small CP violation $\delparzero$ appears in the coefficients of the $\cDGs$ and $\sDGs$
terms, but $\delperpzero$ does not. The latter is fixed to $\pi$ in the untagged fit, because its uncertainty becomes too large. In the
nominal fit the main sensitivity for $\phisav$ and $\Csav$ comes from the $|\Ai|^2$ oscillation terms. In the untagged fit the information
has to come from interference terms and, at second order, from the $|\Ai|^2$ $\cDGs$ and $\sDGs$ terms. As a result, $\DelCsperp$ also
becomes too uncertain and is fixed to zero.

\begin{table}[htbp]
  \centering
  \caption{Results of a fit in which no flavour-tagging information is used.}
  \label{tab:result_untagged_polarDep}
  \begin{tabular}{lllll}
    \hline
    parameter        &  \multicolumn{2}{l}{estimate (difference)}  &  \multicolumn{2}{l}{uncertainty (rel. difference)} \\
    \hline
    $\phisav$        &  \tm0.336           &  (\tm0.290)           &  0.145              &  (+1.9)                      \\
    $\Delphispara$   &  \tm0.099           &  (\tm0.080)           &  +0.103 \tm0.194    &  (+1.4 +3.6)                 \\
    $\Delphisperpp$  &    +0.005           &    (+0.007)           &  +0.108 \tm0.068    &  (+2.8 +1.4)                 \\
    $\DelphisS$      &  \tm0.038           &  (\tm0.053)           &  +0.069 \tm0.573    &  (+0.10 +8.2)                \\
    \hline
    $\Csav$          &  \tm0.282           &  (\tm0.275)           &  0.271              &  (+6.0)                      \\
    $\DelCspara$     &  \tm0.054           &  (\tm0.030)           &  0.128              &  (+0.05)                     \\
    $\DelCsperp$     &  \phantom{+}0       &  (\tm0.044)           &  \tm                &  (\tm)                       \\
    $\CsavS$         &    +0.071           &    (+0.011)           &  +0.190 \tm0.033    &  (+4.9 +0.04)                \\
    \hline
    $\Gs$            &  \phantom{+}0.6606  &    (+0.0015)          &  +0.0038 \tm0.0034  &  (+0.24 +0.08)               \\
    $\DGs$           &   +0.0849           &    (+0.0064)          &  +0.0120 \tm0.0105  &  (+0.31 +0.14)               \\
    $\Dms$           &  \tm                &  (\tm)                &  \tm                &  (\tm)                       \\
    \hline
    $\magzeroAvSq$   &  \phantom{+}0.5242  &    (+0.0006)          &  0.0035             &  (+0.01)                     \\
    $\magperpAvSq$   &  \phantom{+}0.2514  &    (+0.0001)          &  0.0049             &  (\tm)                       \\
    $\delparzero$    &   +2.943            &  (\tm0.303)           &  +0.260 \tm0.171    &  (+1.5 \tm0.15)              \\
    $\delperpzero$   &  \phantom{+}$\pi$   &    (+0.104)           &  \tm                &  (\tm)                       \\
    \hline
  \end{tabular}
\end{table}

\begin{table}[htbp]
  \centering
  \caption{Tagging categories and estimates of the corresponding dilution factors from the fit without tagging-calibration constraints
           (Table~\ref{tab:result_4TagCats_polarDep}).}
  \label{tab:result_4TagCats_polarDep_dilution}
  \begin{tabular}{llll}
    \hline
    OS index      &  $\etaOS$ range                  &  mean $\etaOS$  &  dilution estimate  \\
    \hline
    0 (untagged)  &  $\etaOS$\texteq0.5              &  0.5            &  0                  \\
    1             &  0.34\textle$\etaOS$\textlt0.50  &  0.43           &  0.06\textpm0.04    \\
    2             &  0.22\textle$\etaOS$\textlt0.34  &  0.29           &  0.41\textpm0.08    \\
    3             &  $\etaOS$\textlt0.22             &  0.18           &  0.74\textpm0.14    \\
    \hline
  \end{tabular}

  \vspace*{5pt}
  \begin{tabular}{llll}
    \hline
    SS index      &  $\etaSS$ range                  &  mean $\etaSS$  &  dilution estimate  \\
    \hline
    0 (untagged)  &  $\etaSS$\texteq0.5              &  0.5            &  0                  \\
    1             &  0.40\textle$\etaSS$\textlt0.50  &  0.47           &  0.07\textpm0.03    \\
    2             &  0.30\textle$\etaSS$\textlt0.40  &  0.36           &  0.17\textpm0.07    \\
    3             &  $\etaSS$\textlt0.30             &  0.26           &  0.43\textpm0.12    \\
    \hline
  \end{tabular}
\end{table}

\begin{table}[htbp]
  \centering
  \caption{Results of a fit with four tagging categories for OS tags and four tagging categories for SS tags.
           Tagging-dilution factors are varied independently for the categories, without constraints.}
  \label{tab:result_4TagCats_polarDep}
  \begin{tabular}{lllll}
    \hline
    parameter        &  \multicolumn{2}{l}{estimate (difference)}  &  \multicolumn{2}{l}{uncertainty (rel. difference)} \\
    \hline
    $\phisav$        &  \tm0.061           &  (\tm0.014)           &  0.059            &  (+0.16)                       \\
    $\Delphispara$   &  \tm0.025           &  (\tm0.007)           &  0.043            &  (+0.01)                       \\
    $\Delphisperpp$  &    +0.003           &    (+0.005)           &  +0.036 \tm0.031  &  (+0.27 +0.08)                 \\
    $\DelphisS$      &    +0.013           &  (\tm0.001)           &  +0.060 \tm0.067  &  (\tm0.04 +0.07)               \\
    \hline
    $\Csav$          &    +0.004           &    (+0.011)           &  0.049            &  (+0.26)                       \\
    $\DelCspara$     &  \tm0.011           &    (+0.013)           &  0.131            &  (+0.08)                       \\
    $\DelCsperp$     &    +0.066           &    (+0.021)           &  0.195            &  (+0.20)                       \\
    $\CsavS$         &    +0.059           &  (\tm0.001)           &  +0.036 \tm0.031  &  (+0.11 \tm0.04)               \\
    \hline
    $\Gs$            &  \phantom{+}0.6592  &   (+0.0001)           &  0.0031           &  (\tm)                         \\
    $\DGs$           &   +0.0786           &   (+0.0001)           &  0.0092           &  (\tm)                         \\
    $\Dms$           &  \phantom{+}17.709  &   (+0.013)            &  0.071            &  (+0.15)                       \\
    \hline
    $\magzeroAvSq$   &  \phantom{+}0.5242  &   (+0.0001)           &  0.0035           &  (\tm)                         \\
    $\magperpAvSq$   &  \phantom{+}0.2514  &   (+0.0002)           &  0.0049           &  (\tm)                         \\
    $\delparzero$    &   +2.943            &  (\tm0.040)           &  +0.124 \tm0.236  &  (+0.19 +0.17)                 \\
    $\delperpzero$   &   +3.034            &  (\tm0.003)           &  +0.177 \tm0.211  &  (+0.11 +0.19)                 \\
    \hline
  \end{tabular}
\end{table}

\begin{table}[htbp]
  \centering
  \caption{Results of a fit with four tagging categories for OS tags and four tagging categories for SS tags.
           Tagging-dilution factors follow from the linear model that is also applied in the nominal fit.}
  \label{tab:result_4TagCatsLinear_polarDep}
  \begin{tabular}{lllll}
    \hline
    parameter        &  \multicolumn{2}{l}{estimate (difference)}  &  \multicolumn{2}{l}{uncertainty (rel. difference)} \\
    \hline
    $\phisav$        &  \tm0.040           &    (+0.007)           &  0.054            &  (+0.06)                       \\
    $\Delphispara$   &  \tm0.019           &  (\tm0.001)                &  0.045            &  (+0.05)                       \\
    $\Delphisperpp$  &    +0.003           &    (+0.005)           &  0.032            &  (+0.12)                       \\
    $\DelphisS$      &    +0.017           &    (+0.003)           &  +0.060 \tm0.067  &  (\tm0.03 +0.08)               \\
    \hline
    $\Csav$          &  \tm0.017           &  (\tm0.011)           &  0.041            &  (+0.07)                       \\
    $\DelCspara$     &  \tm0.040           &  (\tm0.016)           &  0.128            &  (+0.05)                       \\
    $\DelCsperp$     &    +0.119           &    (+0.075)           &  0.172            &  (+0.06)                       \\
    $\CsavS$         &    +0.066           &    (+0.006)           &  0.035            &  (+0.08)                       \\
    \hline
    $\Gs$            &  \phantom{+}0.6591  &  (\tm)                &  0.0031           &  (\tm)                         \\
    $\DGs$           &   +0.0783           &  (\tm0.0001)          &  0.0092           &  (\tm)                         \\
    $\Dms$           &  \phantom{+}17.679  &  (\tm0.017)           &  0.065            &  (+0.05)                       \\
    \hline
    $\magzeroAvSq$   &  \phantom{+}0.5236  &  (\tm)                &  0.0034           &  (\tm)                         \\
    $\magperpAvSq$   &  \phantom{+}0.2515  &   (+0.0003)           &  0.0049           &  (\tm)                         \\
    $\delparzero$    &   +3.224            &  (\tm0.023)           &  +0.118 \tm0.253  &  (+0.14 +0.26)                 \\
    $\delperpzero$   &   +2.996            &  (\tm0.042)           &  +0.170 \tm0.212  &  (+0.06 +0.20)                 \\
    \hline
  \end{tabular}
\end{table}

Tables~\ref{tab:result_4TagCats_polarDep} and \ref{tab:result_4TagCatsLinear_polarDep} list the results of fits in which the data sample is
split into tagging categories and separate tagging-calibration parameters are used for each category. The split is based on the estimated
wrong-tag probabilities for OS and SS tags. The categories are given in Table~\ref{tab:result_4TagCats_polarDep_dilution}.

For the results in Table~\ref{tab:result_4TagCats_polarDep} the tagging-dilution factors for the categories are varied independently.
Asymmetries in the tagging calibration are neglected, because the information in the data is insufficient to determine the corresponding
parameters. The resulting dilution factors are also shown in Table~\ref{tab:result_4TagCats_polarDep_dilution}.

The third column in Table~\ref{tab:result_4TagCats_polarDep_dilution} shows the mean of the estimated wrong-tag probability in each
category. These numbers can be used to calculate the tagging-calibration parameters with the linear model that is used in the nominal fit
for each event. The parameter estimates in Table~\ref{tab:result_4TagCatsLinear_polarDep} are the result of a fit with this model.

Uncertainties in the parameter estimates with tagging categories are comparable to the uncertainties in the nominal fit, although they are
still larger. The parameter estimates are more precise with the linear calibration model and also with event-by-event tagging, as expected.

\thispagestyle{empty}
\begin{center}
  \vspace*{\stretch{1}}

  {\Large
  Measurement of CP Violation in Mixing and Decay\\
  of Strange Beauty Mesons}

  %\vspace*{\stretch{2}}

  %\currenttime, \today
  %January 2016

  \vspace*{\stretch{10}}

  {\large
  Jeroen van Leerdam}

  \vspace*{\stretch{1}}
\end{center}

\newpage
\thispagestyle{empty}

\begin{center}
  Cover design by Marcel Buunk \\

  \vspace*{\stretch{1}}
  Set in \href{http://www.linuxlibertine.org/}{Linux Libertine} using \href{https://latex-project.org/}{\LaTeX} \\
  Printed by \href{http://www.gildeprint.nl/}{Gildeprint -- The Netherlands}\\
  %ISBN 123-45-6789-012-3\\

  \vspace*{\stretch{2}}

  \href{http://creativecommons.org/licenses/by/4.0/}{\includegraphics[width=0.2\textwidth]{graphics/title/cc_by_cmyk}}\\
  This work is licensed under a\\
  \href{http://creativecommons.org/licenses/by/4.0/}{Creative Commons Attribution 4.0 International License}.

\end{center}

\vspace*{\stretch{12}}

\noindent%
\hspace*{0.05\textwidth}%
\href{http://www.fom.nl/}{\includegraphics[width=0.20\textwidth]{graphics/title/FOMlogo_fc-crop-cmyk}}%
\hspace*{\stretch{1}}%
\href{http://www.nikhef.nl/}{\includegraphics[width=0.25\textwidth]{graphics/title/NikhefLogoOutline_cmyk}}%
\hspace*{0.05\textwidth}\\
\vspace*{-0.02\textwidth}

\noindent This work is part of the research programme of the Foundation for Fundamental Research on Matter (FOM), which is part of the
Netherlands Organisation for Scientific Research (NWO). It was carried out at the National Institute for Subatomic Physics (Nikhef) in
Amsterdam, the Netherlands.

\newpage
\thispagestyle{empty}

\begin{center}

\vspace*{\stretch{1}}

VRIJE UNIVERSITEIT

\vspace*{\stretch{4}}

{\Large
Measurement of CP Violation in Mixing and Decay\\
of Strange Beauty Mesons}

\vspace*{\stretch{7}}

ACADEMISCH PROEFSCHRIFT

\vspace*{\stretch{1}}

ter verkrijging van de graad Doctor aan\\
de Vrije Universiteit Amsterdam,\\
op gezag van de rector magnificus\\
prof.dr. V. Subramaniam,\\
in het openbaar te verdedigen\\
ten overstaan van de promotiecommissie\\
van de Faculteit der Exacte Wetenschappen\\
op woensdag 18 mei 2016 om 11.45 uur\\
in de aula van de universiteit,\\
De Boelelaan 1105

\vspace*{\stretch{10}}

door

\vspace*{\stretch{1}}

Jeroen van Leerdam

\vspace*{\stretch{1}}

geboren te Doesburg

\vspace*{\stretch{1}}

\end{center}

\newpage
\thispagestyle{empty}

\begin{tabbing}
  promotor:    \hspace{30pt}\=  prof.dr. H.G. Raven  \\
  copromotor:               \>  prof.dr. M.H.M. Merk
\end{tabbing}

\newpage
\thispagestyle{empty}

\vspace*{\stretch{1}}

\begin{center}
  \large
  \emph{to Aukje}
\end{center}

\vspace*{\stretch{3}}

\newpage
\thispagestyle{empty}

\cleardoublepage

\section{The Standard Model of Particle Physics}
\label{sec:intro_SM}

%%%%%%%%%%%%%%%%%%%%%%%%%%%%%%%%%%%%%%%%%%%%%
\subsection{Elementary-Particle Interactions}
\label{subsec:intro_SM_int}
%%%%%%%%%%%%%%%%%%%%%%%%%%%%%%%%%%%%%%%%%%%%%

Interactions between quarks and leptons are described by the \emph{Standard Model of particle physics} (Standard Model, SM). The Standard
Model unifies the electromagnetic and weak forces in a theory of \emph{electroweak
interactions}~\cite{Glashow:1961tr,*Weinberg:1967tq,*Salam:1968rm}, while strong interactions are described separately by \emph{Quantum
Chromodynamics} (QCD)~\cite{Fritzsch:1973pi}. Both theories are based on quantum field theory, in which quarks and leptons are described as
fermionic fields that interact via bosonic fields.

In the electroweak model there are four vector bosons that mediate interactions, three of which acquire a mass through the
\emph{Brout-Englert-Higgs} (BEH) \emph{mechanism}~\cite{Englert:1964et,*Higgs:1964ia,*Higgs:1964pj,*Guralnik:1964eu}. Generating particle
masses with this mechanism implies the existence of the scalar \emph{Higgs boson}, which was recently discovered by the \atlas{} and
CMS experiments at the LHC~\cite{Aad:2012tfa,*Chatrchyan:2012ufa}. The BEH mechanism is believed to be also responsible for generating the
masses of quarks and leptons through so-called Yukawa interactions.

The three massive vector bosons are the $\Wp$, the $\Wm$ and the $\Zz$. These particles are the carriers of the weak interaction. The
remaining massless vector boson is the photon, which mediates the electromagnetic interaction.

Contrary to leptons, quarks carry the ``charge'' of the strong force, or \emph{colour}. This makes them subject to strong interactions,
described by QCD. The mediators of the strong interaction are gluons, which come in eight different types and are themselves colour-charged
particles.

One of the features of the strong force is that it becomes weaker with increasing interaction energy. This phenomenon is called
\emph{asymptotic freedom}~\cite{Gross:1973id,*Politzer:1973fx}. The increasing interaction strength with decreasing energy (or,
equivalently, increasing interaction distance) makes it impossible to completely separate coloured particles. As a result, quarks and
gluons are \emph{confined} within colour-neutral objects, such as hadrons.

%Whenever possible, \emph{perturbation theory} is used to do calculations of particle interactions. Amplitudes for interaction processes are
%expanded as a series in powers of the coupling strength of the interaction. If the coupling strength is small, the exact solution of the
%calculation can be approximated by the leading terms in the series. In QCD this only works for high-energy processes, where the strong
%force is weak. For low-energy QCD processes, like interactions between quarks inside hadrons, the series diverges and non-perturbative
%methods have to be applied.
%
%\begin{table}[hbt]
%  \begin{tabular}{rccc}
%    \hline
%                      &  1st generation         &  2nd generation    &  3rd generation    \\
%    \hline
%    down-type quarks  &  down (d)               &  strange (s)       &  beauty (b)        \\
%    up-type quarks    &  up (u)                 &  charm (c)         &  top (t)           \\
%    charged leptons   &  electron ($\lepe[-]$)  &  muon ($\lmu[-]$)  &  tau ($\ltau[-]$)  \\
%    neutrinos         &  $\nue$                 &  $\numu$           &  $\nutau$          \\
%    \hline
%  \end{tabular}
%  \caption{Quarks and leptons.}
%  \label{tab:quarksLeptons}
%\end{table}
%Both quarks and leptons come in different \emph{flavours}, which are ordered in three generations. This structure is shown in
%Table~\ref{tab:quarksLeptons}. Strong, electromagnetic, and the neutral (Z$^0$) weak interactions conserve flavour and work within a
%generation. In the Standard Model, the only interaction that changes the flavour of quarks and leptons is the charged (W$^\pm$) weak
%interaction \cite{Cabibbo:1963yz,Glashow:1970gm,Kobayashi:1973fv,Pontecorvo:1957cp,*Pontecorvo:1957qd,*Maki:1962mu,*Pontecorvo:1967fh}.
%Mixing of quark flavours is discussed in more detail in Section~\ref{subsec:intro_mixCPV_mix}.

Low-energy QCD interactions between quarks and gluons within hadrons are so strong that they cannot be described with \emph{perturbation
theory}, where amplitudes are expanded as a series in the coupling strength of the interaction. The series would diverge and the exact
solution of a calculation is not approximated by the leading terms in the series. Numerical methods may be applied to treat low-energy
QCD interactions in an alternative way. Strong interactions at high energies and electroweak interactions can be described with
perturbative methods.

Decays of hadrons that contain heavy quarks can be approximately described using a \emph{factorization} approach. In the case of $\Bs$
mesons, for example, the decay of the constituent b quark is treated separately from the strong interactions among the quarks in the $\Bs$
meson and its decay products. The decay takes place at a relatively high energy scale set by the b-quark mass and is calculated
perturbatively. The strong hadronic interactions are less energetic and non-perturbative.

For the decay of a (heavy) quark or lepton a transition to one of the lighter quarks or leptons is required. In the Standard Model, such a
\emph{flavour change} is only possible in a weak interaction that is mediated by the $\Wpm$
boson~\cite{Cabibbo:1963yz,Glashow:1970gm,Kobayashi:1973fv}. This mixing of quark flavours in the context of meson decays is discussed in
more detail in Section~\ref{subsec:intro_mixCPV_mix}.

%For each particle in the Standard Model there is also an antiparticle. Particle and antiparticle states are related by the \emph{charge
%conjugation} operation (C). Since the weak interaction couples to left-handed particle states and right-handed antiparticle states, an
%additional operation is needed to obtain the equivalent antimatter interactions. This operation, which inverts spatial coordinates, is a
%\emph{parity transformation} (P).

Another interesting property of the weak interaction is that it only couples to left-handed particle states and to right-handed
antiparticle states. As a result, two transformations are needed to obtain the equivalent antiparticle interaction for a given particle
interaction: A \emph{charge-conjugation} operation (C) transforms particles into the corresponding antiparticles and a \emph{parity}
operation (P) inverts spatial coordinates, which transforms left-handed states into right-handed states.

Most interactions in the Standard Model are invariant under the combined C and P operation, which makes matter and antimatter almost
symmetric. The only source of CP-symmetry violation is the flavour-changing weak interaction. Measurements of the amount of CP violation
provide tests of the Standard Model description of particle interactions, which is discussed in Section~\ref{subsec:intro_mixCPV_CPV}.

In addition to the charge and parity operations there is the transformation of \emph{time reversal} (T), which inverts the time coordinate.
Quantum field theory is based on the principle of Lorentz invariance, which implies invariance under a combined C, P, and T transformation
for any interaction. CPT is assumed to be an exact symmetry in the studies presented in this thesis.


%%%%%%%%%%%%%%%%%%%%%%%%%%%%%%%%%%%%%%
\subsection{Beyond the Standard Model}
\label{subsec:intro_SM_beyond}
%%%%%%%%%%%%%%%%%%%%%%%%%%%%%%%%%%%%%%

Since the start of its construction in the 1960s, the Standard Model has been tested extensively by experiments. Elementary-particle
interactions have been studied over a wide range of energies, with particles from different sources. The Standard Model has proven to be
a consistent description of particle physics.

Despite its success, there are strong motivations for developing descriptions of particle physics that go beyond the Standard Model. Some
of these motivations are theoretical, while others are based on experimental observations.

The Standard Model contains 18 parameters, for which it predicts no values. There are nine unpredicted quark and lepton masses, the mass
and vacuum expectation value of the Higgs field, four quark-mixing parameters (see Section~\ref{subsec:intro_mixCPV_mix}), and three
interaction coupling constants. These parameters have been measured, but it would be more satisfying to have a more fundamental description
without unpredicted quantities.

Another missing feature in the Standard Model is unification of all forces. While the electric and magnetic forces are unified in
electromagnetism and the electromagnetic and weak forces in electroweak theory, the Standard Model does not attempt to unify the
electroweak and strong forces. The gravitational force is left out completely. Although not strictly necessary to describe current
particle-physics experiments, the knowledge of how to describe all these phenomena consistently would significantly improve our
understanding of nature.

There is a more practical theoretical issue known as the \emph{hierarchy problem}. The mass of the Higgs boson is affected by quantum
corrections from loops of other particles. These corrections are large and with only the particles in the Standard Model one would expect
the Higgs mass to be much larger than the measured value~\cite{Weinberg:1975gm,*Susskind:1978ms,*'tHooft:1979bj}. The only way to get
around this within the Standard Model is a careful fine-tuning of its parameters, which is considered to be unnatural.

Several experimental observations also indicate that the current Standard Model is incomplete. Experiments studying neutrinos from various
sources have observed transitions between neutrinos from different
generations~\cite{Fukuda:1998mi,*Ahmad:2002jz,*Eguchi:2002dm,*An:2012eh}. This implies mixing between leptons of different flavours, but
also non-zero neutrino masses. Neutrinos are massless in the original Standard Model, but in principle their masses can be included by a
minimal extension, assuming neutrino masses have the same origin as quark and charged-lepton masses.

There is also a more exciting possibility to include neutrino masses. Since all quantum numbers of neutrinos and antineutrinos are equal,
neutrinos may be their own antiparticles, which could lead to Majorana mass components~\cite{Majorana:1937vz}. This scenario would open up
new possibilities to explain why neutrino masses are so much smaller than quark and charged-lepton masses. The masses of the left-handed
Standard Model neutrinos could be suppressed by the very large masses of hypothetical right-handed Majorana neutrinos via a so-called
\emph{see-saw mechanism} \cite{Minkowski:1977sc}.

A related observation is that the universe contains matter, but almost no antimatter. One of the required conditions to create such an
imbalance is a sufficient amount of CP violation in particle interactions~\cite{Sakharov:1967dj}. CP violation in the Standard Model is
believed to be too small to generate the large matter--antimatter asymmetry that is
observed~\cite{Gavela:1993ts,*Huet:1994jb,*Gavela:1994dt}.

The right-handed neutrinos introduced by a see-saw mechanism would also provide a natural way to introduce the required additional CP
violation in particle interactions. Couplings of charged leptons to these neutrinos would in general be CP violating, which leads to a
lepton--antilepton asymmetry in neutrino decays. This asymmetry for leptons could subsequently lead to an asymmetry for
baryons~\cite{Kuzmin:1985mm,*Fukugita:1986hr}.

Also other cosmological observations suggest the need for extensions of the Standard Model. If measurements of the spatial fluctuations in
the cosmic microwave background are interpreted with the current cosmological models, the bulk of the matter in the universe has an unknown
origin~\cite{Hinshaw:2012aka}. Less than a fifth is conventional matter, which consists of Standard Model particles. The remainder does not
interact strongly or electromagnetically and is therefore termed \emph{dark matter}. The Standard Model does not provide any particles that
would be viable candidates to form dark matter.

Besides see-saw mechanisms, there are many other ideas how to extend the Standard Model, or even to find a more fundamental theory. A
popular candidate is \emph{Supersymmetry}~\cite{Golfand:1971iw,*Volkov:1973ix,*Wess:1974tw}, which leads to a class of models to be tested
by experiments. Supersymmetry is based on an additional symmetry between bosons and fermions, which would give all particles in the
Standard Model a so called ``superpartner''. Introducing these additional particles could solve the hierarchy problem, may provide a
dark-matter candidate, and could be a first step towards unification of the electroweak and strong forces. See reference
\cite{Feng:2013pwa} for a recent review of the status of Supersymmetry.

To make progress in the search for a more complete theory, observations of particle interactions that cannot be described with the Standard
Model are needed. There are two different approaches to search for such inconsistencies. One can assume a particular beyond-the-Standard
Model theory and test its predictions in a measurement. An example of this \emph{top-down} approach is to search for signs of the
additional particles predicted by Supersymmetry. The other method is a \emph{bottom-up} search, where the Standard Model predictions are
tested instead.

In the bottom-up approach one generally searches for small deviations in predictions for well-known processes. An example is the work
presented here, which is a study of the variables in the decay of a Standard Model particle, the $\Bs$ meson. This particle is produced
abundantly in the proton--proton collisions of the LHC and its decays have been studied previously by other collider experiments. One of
the particularly interesting decays is that into a $\Jpsi$ and a $\phimes$ meson, which could exhibit CP
violation~\cite{Nir:1990hj,*Silverman:1998uj,*Ball:1999yi,*Dunietz:2000cr}.

CP violation in the \BstoJpsiphi{} decay%
\footnote{The symbol ``$\phimesalt$'' will be understood to mean ``$\phimes$'' in this context.}
as predicted by the Standard Model is very small. Studies of various extensions of the Standard Model show that it can be significantly
enhanced by introducing new contributions to this process~\cite{Buras:2009if,Chiang:2009ev,*Datta:2009fk}. The decay is also experimentally
accessible, which makes it an excellent tool to search for deviations from the Standard Model prediction.

The $\Bs$ meson and the \BstoJpsiphi{} decay will be further introduced in Section~\ref{sec:intro_Jpsiphi}. First an overview of CP
violation within the context of the Standard Model is given in the next section.

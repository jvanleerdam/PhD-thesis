\section{The Standard Model of Particle Physics}
% briefly say what we're going to do: study particle interactions at LHC(b)

%%%%%%%%%%%%%%%%%%%%%%%%%%%%%%%%%%
\subsection{Particle Interactions}
%%%%%%%%%%%%%%%%%%%%%%%%%%%%%%%%%%

Interactions between quarks and leptons are described by the \emph{Standard Model of particle physics} (SM). The Standard Model unifies
the electromagnetic and weak forces in a theory of \emph{electroweak interactions} \cite{Glashow:1961tr,*Weinberg:1967tq,*Salam:1968rm},
while strong interactions are described separately by \emph{Quantum Chromodynamics} (QCD) \cite{Fritzsch:1973pi}. Both theories are
based on quantum field theory, in which quarks and leptons are described as fields that interact via bosonic fields.

In the electroweak model there are four vector bosons that mediate interactions, three of which acquire a mass through the
\emph{Brout-Englert-Higgs} (BEH) \emph{mechanism} \cite{Englert:1964et,*Higgs:1964ia,*Higgs:1964pj,*Guralnik:1964eu}. Generating particle
masses with this mechanism implies the existence of the scalar \emph{Higgs boson}, which was recently discovered by the \textsc{Atlas} and
CMS experiments at the LHC \cite{Aad:2012tfa,*Chatrchyan:2012ufa}. The BEH mechanism is believed to be also responsible for generating the
masses of quarks and leptons.

The three massive vector bosons are the W$^+$, the W$^-$ and the Z$^0$. These particles are the carriers of the weak interaction. The
remaining massless vector boson is the photon, which mediates the electromagnetic interaction.

Contrary to leptons, quarks carry the ``charge'' of the strong force, or \emph{colour}. This makes them subject to strong interactions,
described by QCD. The mediators of the strong interaction are gluons, which come in eight different types and are coloured particles
themselves.

One of the features of the strong force is that it becomes weaker with increasing interaction energy. This phenomenon is called
\emph{asymptotic freedom} \cite{Gross:1973id,*Politzer:1973fx}. The increasing interaction strength with decreasing energy (or,
equivalently, increasing interaction distance) makes it impossible to completely separate coloured particles. As a result, quarks and
gluons are \emph{confined} within colour-neutral objects, such as hadrons.

Whenever possible, \emph{perturbation theory} is used to do calculations of particle interactions. Amplitudes for interaction processes are
expanded as a series in powers of the coupling strength of the interaction. If the coupling strength is small, the exact solution of the
calculation can be approximated by the leading terms in the series. In QCD this only works for high-energy processes, where the strong
force is weak. For low-energy QCD processes, like interactions between quarks inside hadrons, the series diverges and non-perturbative
methods have to be applied.

\begin{table}[hbt]
  \begin{tabular}{rccc}
    \hline
                      &  1st generation         &  2nd generation    &  3rd generation    \\
    \hline
    down-type quarks  &  down (d)               &  strange (s)       &  beauty (b)        \\
    up-type quarks    &  up (u)                 &  charm (c)         &  top (t)           \\
    charged leptons   &  electron ($\lepe[-]$)  &  muon ($\lmu[-]$)  &  tau ($\ltau[-]$)  \\
    neutrinos         &  $\nue$                 &  $\numu$           &  $\nutau$          \\
    \hline
  \end{tabular}
  \caption{Quarks and leptons.}
  \label{tab:quarksLeptons}
\end{table}
Both quarks and leptons come in different \emph{flavours}, which are ordered in three generations. This structure is shown in
Table~\ref{tab:quarksLeptons}. Strong, electromagnetic, and the neutral (Z$^0$) weak interactions conserve flavour and work within a
generation. In the Standard Model, the only interaction that changes the flavour of quarks and leptons is the charged (W$^\pm$) weak
interaction \cite{Cabibbo:1963yz,Glashow:1970gm,Kobayashi:1973fv,Pontecorvo:1957cp,*Pontecorvo:1957qd,*Maki:1962mu,*Pontecorvo:1967fh}.
Mixing of quark flavours is discussed in more detail in Section~\ref{sec:intro_mixCPV_mix}.

For each particle in the Standard Model there is also an anti-particle. Particle and anti-particle states are related by the \emph{charge
conjugation} operation (C). Since the weak interaction couples to left-handed particle states and right-handed anti-particle states, an
additional operation is needed to obtain the equivalent anti-matter interactions. This operation, which inverts spatial coordinates, is a
\emph{parity transformation} (P).

Matter and anti-matter interactions in the Standard Model are almost symmetric. That is, interactions are roughly invariant under a
combined C and P operation. The only source of CP-symmetry violation is the flavour-changing weak interaction. This is discussed in
Section~\ref{sec:intro_mixCPV_CPV}. Because the expected amount of CP violation is small, this phenomenon provides a good (null) test of
the Standard-Model description of particle interactions.

In addition to the charge and parity operations there is the transformation of \emph{time reversal} (T), which inverts the time coordinate.
A fundamental assumption of quantum field theory is that all processes are invariant under a combined C, P and T transformation. The
Standard Model as well as the physics processes beyond the Standard Model that are discussed in this work are bound by this restriction.


%%%%%%%%%%%%%%%%%%%%%%%%%%%%%%%%%%%%%%
\subsection{Beyond the Standard Model}
%%%%%%%%%%%%%%%%%%%%%%%%%%%%%%%%%%%%%%


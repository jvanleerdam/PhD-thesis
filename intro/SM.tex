\section{The Standard Model of Particle Physics}
\label{sec:intro_SM}

%%%%%%%%%%%%%%%%%%%%%%%%%%%%%%%%%%%%%%%%%%%%%
\subsection{Elementary-Particle Interactions}
\label{subsec:intro_SM_int}
%%%%%%%%%%%%%%%%%%%%%%%%%%%%%%%%%%%%%%%%%%%%%

Interactions between quarks and leptons are described by the \emph{Standard Model of particle physics} (Standard Model, SM). The Standard
Model unifies the electromagnetic and weak forces in a theory of \emph{electroweak interactions}
\cite{Glashow:1961tr,*Weinberg:1967tq,*Salam:1968rm}, while strong interactions are described separately by \emph{Quantum Chromodynamics}
(QCD) \cite{Fritzsch:1973pi}. Both theories are based on quantum field theory, in which quarks and leptons are described as fields that
interact via bosonic fields.

In the electroweak model there are four vector bosons that mediate interactions, three of which acquire a mass through the
\emph{Brout-Englert-Higgs} (BEH) \emph{mechanism} \cite{Englert:1964et,*Higgs:1964ia,*Higgs:1964pj,*Guralnik:1964eu}. Generating particle
masses with this mechanism implies the existence of the scalar \emph{Higgs boson}, which was recently discovered by the \textsc{Atlas} and
CMS experiments at the LHC \cite{Aad:2012tfa,*Chatrchyan:2012ufa}. The BEH mechanism is believed to be also responsible for generating the
masses of quarks and leptons.

The three massive vector bosons are the W$^+$, the W$^-$ and the Z$^0$. These particles are the carriers of the weak interaction. The
remaining massless vector boson is the photon, which mediates the electromagnetic interaction.

Contrary to leptons, quarks carry the ``charge'' of the strong force, or \emph{colour}. This makes them subject to strong interactions,
described by QCD. The mediators of the strong interaction are gluons, which come in eight different types and are coloured particles
themselves.

One of the features of the strong force is that it becomes weaker with increasing interaction energy. This phenomenon is called
\emph{asymptotic freedom} \cite{Gross:1973id,*Politzer:1973fx}. The increasing interaction strength with decreasing energy (or,
equivalently, increasing interaction distance) makes it impossible to completely separate coloured particles. As a result, quarks and
gluons are \emph{confined} within colour-neutral objects, such as hadrons.

Whenever possible, \emph{perturbation theory} is used to do calculations of particle interactions. Amplitudes for interaction processes are
expanded as a series in powers of the coupling strength of the interaction. If the coupling strength is small, the exact solution of the
calculation can be approximated by the leading terms in the series. In QCD this only works for high-energy processes, where the strong
force is weak. For low-energy QCD processes, like interactions between quarks inside hadrons, the series diverges and non-perturbative
methods have to be applied.

\begin{table}[hbt]
  \begin{tabular}{rccc}
    \hline
                      &  1st generation         &  2nd generation    &  3rd generation    \\
    \hline
    down-type quarks  &  down (d)               &  strange (s)       &  beauty (b)        \\
    up-type quarks    &  up (u)                 &  charm (c)         &  top (t)           \\
    charged leptons   &  electron ($\lepe[-]$)  &  muon ($\lmu[-]$)  &  tau ($\ltau[-]$)  \\
    neutrinos         &  $\nue$                 &  $\numu$           &  $\nutau$          \\
    \hline
  \end{tabular}
  \caption{Quarks and leptons.}
  \label{tab:quarksLeptons}
\end{table}
Both quarks and leptons come in different \emph{flavours}, which are ordered in three generations. This structure is shown in
Table~\ref{tab:quarksLeptons}. Strong, electromagnetic, and the neutral (Z$^0$) weak interactions conserve flavour and work within a
generation. In the Standard Model, the only interaction that changes the flavour of quarks and leptons is the charged (W$^\pm$) weak
interaction \cite{Cabibbo:1963yz,Glashow:1970gm,Kobayashi:1973fv,Pontecorvo:1957cp,*Pontecorvo:1957qd,*Maki:1962mu,*Pontecorvo:1967fh}.
Mixing of quark flavours is discussed in more detail in Section~\ref{subsec:intro_mixCPV_mix}.

For each particle in the Standard Model there is also an antiparticle. Particle and antiparticle states are related by the \emph{charge
conjugation} operation (C). Since the weak interaction couples to left-handed particle states and right-handed antiparticle states, an
additional operation is needed to obtain the equivalent antimatter interactions. This operation, which inverts spatial coordinates, is a
\emph{parity transformation} (P).

Matter and antimatter interactions in the Standard Model are almost symmetric. That is, interactions are roughly invariant under a
combined C and P operation. The only source of CP-symmetry violation is the flavour-changing weak interaction. This is discussed in
Section~\ref{subsec:intro_mixCPV_CPV}. Because the expected amount of CP violation is small, this phenomenon provides a good (null) test of
the Standard-Model description of particle interactions.

In addition to the charge and parity operations there is the transformation of \emph{time reversal} (T), which inverts the time coordinate.
A fundamental assumption of quantum field theory is that all processes are invariant under a combined C, P and T transformation. The
Standard Model as well as the physics processes beyond the Standard Model that are discussed in this work are bound by this restriction.


%%%%%%%%%%%%%%%%%%%%%%%%%%%%%%%%%%%%%%
\subsection{Beyond the Standard Model}
\label{subsec:intro_SM_beyond}
%%%%%%%%%%%%%%%%%%%%%%%%%%%%%%%%%%%%%%

Since the start of its development in the 1960s, the Standard Model has been tested extensively by experiments. Elementary-particle
interactions have been studied at a wide range of energies, with particles from different sources. The Standard Model withstood all tests
thus far and provides a consistent description of particle physics.

Despite its success, there are strong motivations for developing descriptions of particle physics that go beyond the Standard Model. Some
of these motivations are theoretical, while others are based on experimental observations.

The Standard Model contains 19 parameters for which it predicts no values. Among these parameters are particle masses and coupling
strengths of interactions. The parameter values have all been measured, but it would be more satisfying to have a more fundamental
description without unpredicted quantities.

While the electric and magnetic forces are unified in electromagnetism and the electromagnetic and weak forces in electroweak theory, the
Standard Model does not attempt to unify the electroweak and strong forces. The gravitational force is left out completely. Although not
strictly necessary to describe current particle-physics experiments, a theory that includes all these phenomena consistently would
significantly improve our understanding of nature.

There is a more practical theoretical issue known as the \emph{hierarchy problem}. The mass of the Higgs boson is affected by quantum
corrections from loops of other particles. These corrections are large and with only the particles in the Standard Model one would expect
the Higgs mass to be much larger than the measured value \cite{Weinberg:1975gm,*Susskind:1978ms,*'tHooft:1979bj}. The only way to get
around this within the Standard Model is a careful fine-tuning of its parameters, which is considered to be unnatural.

Several experimental observations have shown that the current Standard Model is incomplete. Experiments studying neutrinos from various
sources have observed transitions between neutrinos from different generations
\cite{Fukuda:1998mi,*Ahmad:2002jz,*Eguchi:2002dm,*An:2012eh}. This implies mixing between leptons of different flavours, but also non-zero
neutrino masses. Neutrinos are massless in the original Standard Model, but in principle their masses can be included by a minimal
extension. This would assume neutrino masses have the same origin as quark and charged-lepton masses.

A more exciting possibility is that neutrinos could be their own antiparticles, which would give them Majorana masses
\cite{Majorana:1937vz}. This is possible because all quantum numbers of neutrinos and antineutrinos are equal. This extension of the
Standard Model is more involved and opens up new possibilities for the origin of neutrino masses.

Also observations from fields related to particle physics suggest the Standard Model should be extended. Interpreting measurements of
spatial fluctuations in the cosmic microwave background with the current cosmological models, the bulk of the energy content of the
universe has an unknown origin \cite{Hinshaw:2012aka}. Less than a fifth of the matter in the universe is conventional matter, which
consists of Standard Model particles. The remainder does not emit light and is therefore termed \emph{dark matter}. The Standard Model does
not include particles that would be suitable candidates to form dark matter.

Another observation related to cosmology is that the universe contains matter, but almost no antimatter. One of the conditions to create
such an imbalance is a sufficient amount of CP violation \cite{Sakharov:1967dj}. CP violation in the Standard Model is believed to be too
small to generate the large matter--antimatter asymmetry that is observed.

There are many ideas about how to extend the Standard Model, or even to find a more fundamental theory. Some of these are already excluded
by experimental observations, while others are still good candidates. There is no concrete evidence for any theory yet.

A popular theory is \emph{Supersymmetry} \cite{Golfand:1971iw,*Volkov:1973ix,*Wess:1974tw}, which leads to a class of models to be tested
by experiments. Supersymmetry attempts to solve the hierarchy problem, may provide a dark-matter candidate and could be a first step to
unification of the electroweak and strong forces. It is based on an additional symmetry between bosons and fermions, which would give all
particles in the Standard Model a so called ``superpartner''.

To make progress in the search for a more complete theory, observations of particle interactions that cannot be described with the Standard
Model are needed. There are two different approaches to search for such inconsistencies. One can assume a particular beyond-the-Standard
Model theory and test its predictions in a measurement. An example of this \emph{top-down} approach is to search for signs of the
additional particles predicted by Supersymmetry. The other method is a \emph{bottom-up} search, where the Standard Model predictions are
tested instead.

In the bottom-up approach one generally searches for small deviations in predictions for well-known processes. The work presented here is a
study of the variables in the decay of a Standard Model particle, the $\Bs$ meson. This particle is produced abundantly in the
proton--proton collisions of the LHC and its decays have been studied previously by in other collider experiments. One of the
particularly interesting decays is that into a $\Jpsi$ and a $\phi$ meson, which is affected by CP violation
\cite{Nir:1990hj,*Silverman:1998uj,*Ball:1999yi,*Dunietz:2000cr}.

CP violation in the \BstoJpsiphi{} decay is very small in the Standard Model. Studies of various extensions of the Standard Model show that
it can be significantly enhanced by introducing new processes that affect the decay \cite{Buras:2009if,Chiang:2009ev}. A measurement of
large CP violation would be an unambiguous deviation from the Standard Model, but also smaller corrections would be important to confirm or
rule out new theories.

The $\Bs$ meson and the \BstoJpsiphi{} decay will be further introduced in Section~\ref{sec:intro_Jpsiphi}. First an overview of CP
violation in the Standard Model is given in the next section.

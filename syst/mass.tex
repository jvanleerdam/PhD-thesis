\section{Mass model systematics}
\label{sec:syst_mass}

In the analysis of the decay-time and decay-angle distributions a set of fixed event weights is used to statistically separate signal and
background contributions (see Section~\ref{sec:ana_bkg}). The weights are based on the estimates of the signal and background contributions
to the $\JpsiKK$-mass distribution. Uncertainties in the mass model are propagated to the time and angular analysis by evaluating the
effects of different sets of weights on the final parameter estimates.

\subsection{Reflection backgrounds}
\label{subsec:syst_mass_peaking}

Reflection backgrounds from \BdtoJpsiKst{} and \LbtoJpsipK{} are subtracted by injecting simulated events with negative weights into the
data sample that is used for analysis. Two types of uncertainties arise from this procedure: uncertainties in the estimates of the numbers
of background events that affect the \BstoJpsiKK{} analysis and uncertainties in the relevant distributions of the reflection backgrounds.

A systematic uncertainty for the numbers of background events is estimated by changing the subtracted background yields by one standard
deviation. The (absolute) variations resulting from an upward and a downward fluctuation are averaged. The yields of all background
components are varied simultaneously.

The distributions of the decay angles and tagging variables of simulated background events are reweighted to make them match the
distributions in \BdtoJpsiKst{} and \LbtoJpsipK{} data as closely as possible. Since the results of this procedure are not expected to be
perfect, the difference in results with datasets containing reweighted and not reweighted distributions is taken as a systematic
uncertainty.

As a check, the analysis is also performed without subtracting the reflection backgrounds. The result is not used for evaluation of the
systematic uncertainties.

Results with the different datasets are shown in Tables~\ref{tab:syst_mass_peaking_phi} to \ref{tab:syst_mass_peaking_polarDep} for the
three different time and angular models. Only the differences with the nominal results (see Section~\ref{sec:result_paramEst}) are shown.
The resulting systematic uncertainties are shown in the last column.

\begin{table}[htbp]
  \centering
  \caption{Deviations in parameter estimates from the nominal results with the model with $\lamsAbs\equiv1$ for different schemes of
           subtracting reflection backgrounds. The differences with the nominal result are given in fractions of the statistical errors.
           The resulting systematic uncertainty in the last column is obtained by taking the average of the yield-variation uncertainties
           and adding the reweighting uncertainty in quadrature.}
  \label{tab:syst_mass_peaking_phi}
  \begin{tabular}{lllllll}
    \hline
    parameter            &  no subtr.  &  yield--  &  yield+   &  no rew.  &  \multicolumn{2}{l}{systematic (abs.)}  \\
    \hline
    $\phis$              &  --0.041    &  --0.025  &   +0.025  &  --0.021  &  0.033  &  (0.0015)                     \\
    \hline
    $\Gs$                &   +0.000    &  --0.013  &   +0.016  &  --0.012  &  0.019  &  (0.000058~\invps)            \\
    $\DGs$               &  --0.069    &  --0.007  &   +0.013  &  --0.046  &  0.047  &  (0.00043~\invps)             \\
    $\Dms$               &   +0.086    &   +0.022  &  --0.020  &  --0.004  &  0.021  &  (0.0013~\invps)              \\
    \hline
    $\magzeroSq$         &  --0.251    &  --0.007  &   +0.008  &  --0.072  &  0.072  &  (0.00025)                    \\
    $|A_\perp|^2$        &   +0.180    &   +0.008  &  --0.003  &   +0.080  &  0.080  &  (0.00039)                    \\
    $\FS[1]$             &  --0.271    &  --0.043  &   +0.039  &   +0.070  &  0.081  &  (0.0044)                     \\
    $\FS[2]$             &   +0.070    &   +0.019  &  --0.011  &   +0.023  &  0.027  &  (0.00049)                    \\
    $\FS[3]$             &  --0.129    &  --0.019  &   +0.021  &  --0.010  &  0.022  &  (0.00015)                    \\
    $\FS[4]$             &  --0.084    &  --0.002  &   +0.005  &   +0.024  &  0.024  &  (0.00014)                    \\
    $\FS[5]$             &   +0.232    &   +0.025  &  --0.023  &   +0.023  &  0.033  &  (0.00054)                    \\
    $\FS[6]$             &  --0.140    &  --0.036  &   +0.039  &   +0.185  &  0.19   &  (0.0048)                     \\
    \hline
    $\delpar-\delzero$   &   +0.016    &   +0.020  &  --0.014  &  --0.195  &  0.20   &  (0.024)                      \\
    $\delperp-\delzero$  &   +0.065    &   +0.029  &  --0.034  &  --0.088  &  0.093  &  (0.015)                      \\
    $\delS[1]-\delperp$  &  --0.055    &  --0.025  &   +0.023  &   +0.117  &  0.12   &  (0.024)                      \\
    $\delS[2]-\delperp$  &   +0.302    &   +0.036  &  --0.044  &   +0.105  &  0.11   &  (0.032)                      \\
    $\delS[3]-\delperp$  &   +0.083    &   +0.011  &  --0.008  &  --0.040  &  0.041  &  (0.0094)                     \\
    $\delS[4]-\delperp$  &  --0.058    &  --0.002  &  --0.000  &   +0.019  &  0.019  &  (0.0041)                     \\
    $\delS[5]-\delperp$  &   +0.106    &   +0.007  &  --0.008  &  --0.100  &  0.100  &  (0.018)                      \\
    $\delS[6]-\delperp$  &  --0.051    &   +0.003  &  --0.002  &   +0.047  &  0.047  &  (0.0065)                     \\
    \hline
  \end{tabular}
\end{table}

\begin{table}[htbp]
  \centering
  \caption{Deviations in parameter estimates from the nominal results with the model with a free $\lamsAbs$ for different schemes of
           subtracting reflection backgrounds. The resulting systematic uncertainty in the last column is obtained by taking the average of
           the yield-variation uncertainties and adding the reweighting uncertainty in quadrature.}
  \label{tab:syst_mass_peaking_lamb_phi}
\end{table}

\begin{table}[htbp]
  \centering
  \caption{Deviations in parameter estimates from the nominal results with the model with polarization-dependent CP violation for
           different schemes of subtracting reflection backgrounds. The resulting systematic uncertainty in the last column is obtained by
           taking the average of the yield-variation uncertainties and adding the reweighting uncertainty in quadrature.}
  \label{tab:syst_mass_peaking_polarDep}
\end{table}

\subsection{Signal and combinatorial background: statistical}
\label{subsec:syst_mass_stat}

\subsection{Signal and combinatorial background: factorization}
\label{subsec:syst_mass_factor}

\chapter{Angular Differential Decay Rate}
\label{chap:angularDecay}

%%%%%%%%%%%%%%%%%%%%%%%%%%%%
\section{Angular Amplitude}
\label{sec:angularDecay_amp}
%%%%%%%%%%%%%%%%%%%%%%%%%%%%

The dependence of the \BstoJpsiKK{} decay on the three decay angles, as described in Section~\ref{sec:pheno_angles}, can be derived within
the helicity formalism~\cite{Jacob:1959at,Chung:1971ri,*Richman:1984gh,*Kutschke:1996}. In this formalism the decay is described in terms
of three two-body decays; first the decay of the $\Bs$ into a $\Jpsi$ and a second intermediate particle, followed by the decay of the
$\Jpsi$ into a $\mumu$ pair and the decay of the second particle into a $\KK$ pair. The angular dependence arises from the rotations of the
spin vectors of the decaying particles into the momentum directions of the decaying particles.

\begin{figure}[p]
  \centering
  \resizebox{\textwidth}{!}{\input{graphics/angularDecay/helFormFrames.pdftex_t}}
  \caption{Helicity frames for the decay \BtoaPPbPP{} in (a) a general
    configuration where the coordinate systems in the $a$ and $b$ rest frames are aligned and (b)
    with the Jacob--Wick convention, in which the coordinate system in the $b$
    rest frame is rotated. A definition of the helicity angle $\phihel$ in the Jacob--Wick
    convention is shown in (c).}
  \label{fig:helFormFrames}
\end{figure}

A generalized form of the decay is shown in Figure~\ref{fig:helFormFrames}. The three coordinate systems in part (a) of the figure are
defined in the centre-of-mass frames of particles $B$, $a$, and $b$ in the decay \BtoaPPbPP. The spherical coordinates $\theta$ and
$\varphi$ specify the momentum direction of one of the particles in each two-body decay. The other particle, which is not shown in the
figure, has an opposite momentum.

The directions of the decay-product momenta define a \emph{helicity axis} for each of the three decays. The z axis of a particle coordinate
system lies along the helicity axis that is associated to the production of the particle. The momentum direction of one of the two
particles in a decay is chosen as the positive z direction, as depicted in Figure~\ref{fig:helFormFrames}a for particles $a$ and $b$.

The sum of the spin projections of the two particles in each decay along the helicity axis is given by the difference of the particle
helicities. Assuming the spin projection of the decaying particle in the z direction is known, the amplitude for the measuring the required
spin projection along the helicity axis is given by the complex conjugate of a \emph{Wigner D-matrix}, which is a function of three Euler
angles that specify the rotation of the spin projection from the z axis to the helicity axis.

The D-matrix for each of the two-body decays is expressed in terms of a real-valued d-matrix and two exponential functions as
\begin{equation}
  \label{eq:Ddef}
  \WD{j}{m}{n}{}(\varphi,\theta,\varphi') = e^{-im\varphi}\, \Wd{j}{m}{n}(\theta)\, e^{-in\varphi'}\ ,
\end{equation}
where $j$ is the spin of the decaying particle, $m$ the projection of the spin in the direction of the z axis, $n$ the projection of the
spin in the positive direction of the helicity axis. The Euler angles that specify the rotation of the spin projection are $\varphi$,
$\theta$, and $\varphi'$. The angles $\theta$ and $\varphi$ are equal to the spherical coordinates in the mother-particle rest frame, while
the angle $\varphi'$ rotates the coordinate system of a decay product around the helicity axis. In the \emph{Jacob-Wick} convention
$\varphi'$ is chosen to be equal to \tm$\varphi$, resulting in
\begin{equation}
  \label{eq:DdefJW}
  \WD{j}{m}{n}{}(\varphi,\theta,-\varphi) = \Wd{j}{m}{n}(\theta)\, e^{-i(m-n)\varphi}\ .
\end{equation}

A boost along the helicity axis does not affect the spin projections along this axis. Therefore, the projection of the spin of a decay
product along the helicity axis in the centre-of-mass frame of its production is equal to the spin projection in the z direction of the
coordinate system in its rest frame. As a result, this projection relates the spins the decays that are described in the two frames.

To define angles that are consistent with the helicity angles in Figure~\ref{fig:helAngles}, the coordinate system of particle $b$ is
rotated by 180\textdegree{} around the $\mathrm{y}_b$ axis, as shown in Figure~\ref{fig:helFormFrames}b. This operation introduces an
additional D-matrix, which represents the amplitude for a rotation with $\theta$\texteq$\pi$ and the other two Euler angles arbitrary, but
equal. The spin projection in the $\mathrm{z}_b$ direction goes from \tm$\lambda_b$ to \tp$\lambda_b$, where $\lambda_b$ is the helicity of
particle $b$ in the $B$ rest frame. Using the d-matrix property
$\Wd{j}{m}{n}(\pi)$\texteq$(\text{\tm1})^{j\text{\tm}n}\,\delta_{m,\text{\tm}n}$, this D-matrix can be written as
\begin{equation}
  \WD{j_b}{-\lambda_b}{\lambda_b}{*}(\gamma,\pi,\gamma)
      = e^{-i\lambda_b\gamma}\, \Wd{j_b}{-\lambda_b}{\lambda_b}(\theta)\, e^{+i\lambda_b\gamma}
      = (-1)^{j_b-\lambda_b}\ .
\end{equation}

Identifying the particles $P_1$ and $P_3$ with the particles $\Kp$ and $\lmu[+]$, respectively, the angle
$\theta_a$ corresponds to $\thetaK$ and the angle $\theta_b$ to $\thetal$. The angle $\phihel$ is given by the sum of $\varphi_a$ and
$\varphi_b$, as shown in Figure~\ref{fig:helFormFrames}c.

Combining the D-matrices for the three decays, the factor $\angAmp[]$ in Equation~\ref{eq:decayAmp} is given by
\begin{equation}
  \label{eq:angAmp}
  \begin{aligned}
    \angAmp[h] &=
         \sqrt{\tfrac{2j_B+1}{4\pi}\; \tfrac{2j_a^h+1}{4\pi}\; \tfrac{2j_b^h+1}{4\pi}}\\
         &\quad\times
         (-1)^{j_b^h-\lambda_b^h}\
         \WD{j_B}{\lambda_B}{\lambda_a^h-\lambda_b^h}{*}(\Omega_B)\
         \WD{j_a^h}{\lambda_a^h}{\lambda_1-\lambda_2}{*}(\Omega_a)\
         \WD{j_b^h}{\lambda_b^h}{\lambda_3-\lambda_4}{*}(\Omega_b)\ ,
  \end{aligned}
\end{equation}
where $\Omega_p$ is a shorthand notation for the set of Euler angles in the decay of particle $p$ and the factors
$\sqrt{\frac{2j+1}{4\pi}}$ are normalization factors for the two-particle states in the $B$, $a$, and $b$ decays. The index $h$ runs over
intermediate helicity states, for which the particles have definite helicities. The final expressions for helicity states will be combined
into expressions for the transversity states of Equation~\ref{eq:decayAmp}. Notice that the spins and helicities of particles $B$ and 1--4
do not depend on the intermediate state, since these particles are external and their spin state is, in principle, observable.

Concentrating only on the case where particle $B$ is a spinless particle, $j_B$, $\lambda_B$, and $\lambda_a$\textminus$\lambda_b$ are
equal to zero. As a result, the D-matrix corresponding to the $B$ decay reduces to $\WD{0}{0}{0}{*}$\texteq1, which makes the decay
independent of the orientation of the helicity axis with respect to the direction of the $B$ momentum. Introducing the notation
$\lambda_a$\texteq$\lambda_b$\textequiv$\lambda$, $\lambda_1$\textminus$\lambda_2$\textequiv$\alpha$, and
$\lambda_3$\textminus$\lambda_4$\textequiv$\beta$, Equation~\ref{eq:angAmp} reduces to
\begin{equation}
  \label{eq:angAmpRed}
  \begin{aligned}
    \angAmp[h] &=
         \frac{(-1)^{j_b^h-\lambda^h}}{(4\pi)^{3/2}} \sqrt{(2j_a^h+1)\, (2j_b^h+1)}\
         \WD{j_a^h}{\lambda^h}{\alpha}{*}(\Omega_a)\
         \WD{j_b^h}{\lambda^h}{\beta}{*}(\Omega_b)\ .
  \end{aligned}
\end{equation}

%%%%%%%%%%%%%%%%%%%%%%%%%%%%%%%%%%%
\section{Squared Angular Amplitude}
\label{sec:angularDecay_squareAmp}
%%%%%%%%%%%%%%%%%%%%%%%%%%%%%%%%%%%

With Equation~\ref{eq:angAmpRed} the products $\angAmp[h]^*\,\angAmp[h']^{\phantom{*}}$ that appear in Equation~\ref{eq:sqAmp} can be
expressed as
\begin{equation}
  \label{eq:angAmpSq}
  \begin{aligned}
    &\angAmp[h]^*\,\angAmp[h']^{\phantom{*}} \\
    &\qquad=\frac{(-1)^{j_b^h+j_b^{h'}+M^{hh'}-2\lambda^h}}{(4\pi)^{3}}
      \sqrt{(2j_a^h+1)\, (2j_a^{h'}+1)\, (2j_b^h+1)\, (2j_b^{h'}+1)}\\
      &\qquad\qquad\times
        \WD{j_a^h}{\lambda^h}{\alpha}{}(\Omega_a)\ \WD{j_a^{h'}}{\lambda^{h'}}{\alpha}{*}(\Omega_a)\
        \WD{j_b^h}{\lambda^h}{\beta}{}(\Omega_b)\ \WD{j_b^{h'}}{\lambda^{h'}}{\beta}{*}(\Omega_b)\ .
  \end{aligned}
\end{equation}
To evaluate the dependence between the D-matrices in this expression the following relations are applied:
\begin{subequations}
  \label{eq:Dprops}
  \begin{align}
    \WD{j}{m}{n}{*} &= (-1)^{m-n}\ \WD{j}{-m}{-n}{} \\
    \WD{j}{m}{n}\,\WD{j'}{m'}{n'}{}
      &= \sum_{J=|j-j'|}^{j+j'}
      \langle j\,m,\,j'\,m'|J\ m+m' \rangle \nonumber\\
      &\qquad\qquad\times
      \langle j\,n,\,j'\,n'|J\ n+n' \rangle\,
      \WD{J}{m+m'}{n+n'}{} \\
    \langle j\,m,\,j'\,m'|J\ M \rangle
      &= (-1)^{j-j'+M}\,\sqrt{2\,J+1}\
      \begin{pmatrix}
        \mss{j} & \mss{j'} & \mss{J} \\
        \mss{m} & \mss{m'} & \mss{-M}
      \end{pmatrix}
  \end{align}
\end{subequations}
where $\langle j\,m,\,j'\,m'|J\ M \rangle$ is a Clebsch-Gordan coefficient, which is related to the Wigner 3j symbol
$\begin{pmatrix} \mss{j} & \mss{j'} & \mss{J} \\ \mss{m} & \mss{m'} & \mss{-M} \end{pmatrix}$. The product of a D-matrix and a complex
conjugate D-matrix with equal indices $n$ can now be written as
\begin{equation}
  \label{eq:DDstar}
  \begin{aligned}
    \WD{j}{m}{n}{}\,\WD{j'}{m'}{n}{*}
    &= (-1)^{m'-n}\ \WD{j}{m}{n}{}\,\WD{j'}{-m'}{-n}{} \\
    &= (-1)^{m'-n} \sum_{J=|j-j'|}^{j+j'}
      \langle j\, m,\, j'\, -m' | J\ m-m' \rangle\\
      &\qquad\qquad\qquad\qquad\quad\times\langle j\,n,\,j'\,-n|J\,0 \rangle\
      \WD{J}{m-m'}{0}{} \\
    &= (-1)^{m-n} \sum_{J=|j-j'|}^{j+j'} \left(2J+1\right)\
      \begin{pmatrix}
        \mss{j} & \mss{j'} & \mss{J} \\
        \mss{m} & \mss{-m'} & \mss{-m+m'}
      \end{pmatrix}\\
      &\qquad\qquad\qquad\qquad\qquad\ \ \times
      \begin{pmatrix}
        \mss{j} & \mss{j'} & \mss{J} \\
        \mss{n} & \mss{-n} & \mss{0}
      \end{pmatrix}
      \WD{J}{m-m'}{0}{}\ .
  \end{aligned}
\end{equation}
With this expression the products of two D-matrices in Equation~\ref{eq:angAmpSq} that are functions of the same set of angles can be
substituted by a sum of single D-matrices. After this substitution and with the definition
$M^{hh'}$\textequiv$\lambda^h$\textminus$\lambda^{h'}$, the product $\angAmp[h]^*\,\angAmp[h']^{\phantom{*}}$ is given by
\begin{align}
  \label{eq:angAmpSqEval}
  &\angAmp[h]^*\,\angAmp[h']^{\phantom{*}} \nonumber\\*
  &\qquad=\frac{(-1)^{j_b^h+j_b^{h'}+M^{hh'}-\alpha-\beta}}{(4\pi)^{3}}
    \sqrt{(2j_a^h+1)\, (2j_a^{h'}+1)\, (2j_b^h+1)\, (2j_b^{h'}+1)} \nonumber\\
    &\qquad\quad \times
    \sum_{J_a^{hh'}=|j_a^h-j_a^{h'}|}^{j_a^h+j_a^{h'}} \left(2J_a^{hh'}+1\right)\
    \begin{pmatrix}
      \mss{j_a^h} & \mss{j_a^{h'}} & \mss{J_a^{hh'}} \\
      \mss{\lambda^h} & \mss{-\lambda^{h'}} & \mss{-M^{hh'}}
    \end{pmatrix}\
    \begin{pmatrix}
      \mss{j_a^h} & \mss{j_a^{h'}} & \mss{J_a^{hh'}} \\
      \mss{\alpha} & \mss{-\alpha} & \mss{0}
    \end{pmatrix} \nonumber\\
    &\qquad\quad \times
    \sum_{J_b^{hh'}=|j_b^h-j_b^{h'}|}^{j_b^h+j_b^{h'}} \left(2J_b^{hh'}+1\right)\
    \begin{pmatrix}
      \mss{j_b^h} & \mss{j_b^{h'}} & \mss{J_b^{hh'}} \\
      \mss{\lambda^h} & \mss{-\lambda^{h'}} & \mss{-M^{hh'}}
    \end{pmatrix}\
    \begin{pmatrix}
      \mss{j_b^h} & \mss{j_b^{h'}} & \mss{J_b^{hh'}} \\
      \mss{\beta} & \mss{-\beta} & \mss{0}
    \end{pmatrix} \nonumber\\*
    &\qquad\qquad\qquad\qquad\qquad\qquad\qquad\qquad\times
      \WD{J_a^{hh'}}{M^{hh'}}{0}{}(\Omega_a)\ \WD{J_b^{hh'}}{M^{hh'}}{0}{}(\Omega_b)\ .
\end{align}

Notice that the angular dependence is still written in terms of the sets of Euler angles corresponding to the decays of particles $a$ and
$b$, $\Omega_a$ and $\Omega_b$, respectively. The dependence on the helicity angles, $\thetaK$\textequiv$\theta_a$,
$\thetal$\textequiv$\theta_b$, and $\phihel$\textequiv$\varphi_a$\textplus$\varphi_b$, becomes apparent by expressing the product of
D-matrices in Equation~\ref{eq:angAmpSqEval} in terms of d-matrices and exponential functions (see Equation~\ref{eq:DdefJW}):
\begin{equation}
  \begin{aligned}
    \label{eq:Eultohel}
    &\WD{J_a^{hh'}}{M^{hh'}}{0}{}(\Omega_a)\ \WD{J_b^{hh'}}{M^{hh'}}{0}{}(\Omega_b) \\
    &\qquad= \Wd{J_a^{hh'}}{M^{hh'}}{0}(\theta_a)\ e^{-iM^{hh'}\varphi_a}\
             \Wd{J_b^{hh'}}{M^{hh'}}{0}(\theta_b)\ e^{-iM^{hh'}\varphi_b} \\
    &\qquad= \Wd{J_a^{hh'}}{M^{hh'}}{0}(\thetaK)\
             \Wd{J_b^{hh'}}{M^{hh'}}{0}(\thetal)\ e^{-iM^{hh'}\phihel}\\
    &\qquad= \sqrt{\frac{\left(J_a^{hh'}-M^{hh'}\right)!}{\left(J_a^{hh'}+M^{hh'}\right)!}\, \frac{4\pi}{2J_b^{hh'}+1}}\
             \aLp{J_a^{hh'}}{M^{hh'}}(\cthetaK)\,\sph{J_b^{hh'}}{M^{hh'}}{*}(\thetal,\phihel)\ .
  \end{aligned}
\end{equation}
The functions $\aLp{j}{m}(\cos\theta)$ and $\sph{j}{m}{}(\theta,\varphi)$ are associated Legendre polynomials and spherical harmonics,
respectively, which are defined by
\begin{subequations}
  \label{eq:PYdef}
  \begin{align}
    \Lp{j}(x) &\equiv \frac{1}{2^j j!} \frac{\ud^j}{\ud x^j} \left(x^2-1\right)^j \qquad \text{with } j\ge0 \\
    \aLp{j}{m}(x) &\equiv
      \left\{\begin{array}{cl}
        (-1)^m\,\left(1-x^2\right)^{\tfrac{1}{2}m}\,\frac{\ud^m}{\ud x^m}\Lp{j}(x) &\quad(m\ge0) \\
        \tfrac{(j-|m|)!}{(j+|m|)!}\,\left(1-x^2\right)^{\tfrac{1}{2}|m|}\,\frac{\ud^{|m|}}{\ud x^{|m|}}\Lp{j}(x) &\quad(m<0)
      \end{array}\right.\\
    \sph{j}{m}{}(\theta,\phi) &\equiv \sqrt{\tfrac{2j+1}{4\pi}\tfrac{(j-m)!}{(j+m)!}}\ \aLp{j}{m}(\cos\theta)\, e^{im\phi} \\
    \Wd{j}{m}{0}(\theta) &= \sqrt{\tfrac{(j-m)!}{(j+m)!}}\ \aLp{j}{m}(\cos\theta)\ .
  \end{align}
\end{subequations}

The angular functions that eventually appear in the expression of the differential decay rate are given by the real and imaginary parts of
the products $\angAmp[h]^*\,\angAmp[h']^{\phantom{*}}$. The only complex-valued factor on the right-hand side of
Equation~\ref{eq:angAmpSqEval} after substituting the expression for $\WD{J_a}{M}{0}{}(\Omega_a)\ \WD{J_b}{M}{0}{}(\Omega_b)$ from
Equation~\ref{eq:Eultohel} is the spherical harmonic $\sph{J_b}{M}{*}(\thetal,\phihel)$. Notice that this function is real valued if its
upper index is equal to zero:
\begin{equation}
    \sph{j}{0}{}(\theta,\phi) = \sqrt{\tfrac{2j+1}{4\pi}}\ \Lp{j}(\cos\theta)\ .
\end{equation}
Real-valued spherical harmonics are used in the angular functions, which are defined by
\begin{equation}
  \label{eq:realYdef}
  \rsph{j}{m}(\theta,\phi) = \left\{\begin{array}{cl}
                               Y_j^0(\theta,\phi) &\quad(m=0) \\[0.2em]
                               \sqrt{2}\ \Re[Y_j^m(\theta,\phi)] &\quad(m>0) \\[0.2em]
                               \sqrt{2}\ \Im[Y_j^{|m|}(\theta,\phi)] &\quad(m<0)
                             \end{array}\right.
\end{equation}

The squared amplitude in the helicity basis can be expressed in a form that is similar to Equation~\ref{eq:sqAmpExpand}:
\begin{equation}
  \label{eq:sqAmpHel}
  \begin{aligned}
    |\mathcal{A}|^2
      &\propto \sum_h |H_h\,\angAmp[h]|^2 + \sum_{h\neq h'} \Re( H_h^*\,H_{h'}\, \angAmp[h]^*\,\angAmp[h'] ) \\
      &\propto \sum_h |H_h|^2\, |\angAmp[h]|^2 \\
        &\qquad + \sum_{h\neq h'} \Re(H_h^*\,H_{h'})\, \Re(\angAmp[h]^*\,\angAmp[h'])
                               - \Im(H_h^*\,H_{h'})\, \Im(\angAmp[h]^*\,\angAmp[h']) \ .
  \end{aligned}
\end{equation}
Terms in the second sum of Equation~\ref{eq:sqAmpHel} with $h$ and $h'$ swapped are identical, since $H_h^*\,H_{h'}\,
\angAmp[h]^*\,\angAmp[h']$ and $H_{h'}^*\,H_h\, \angAmp[h']^*\,\angAmp[h]$ are complex conjugates, which have identical real parts. The
sign of $M^{hh'}$\textequiv$\lambda^h$\textminus$\lambda^{h'}$ is opposite for these terms. Adding the two contributions for each of the
$h\leftrightarrow h'$ pairs, the angular functions in the helicity basis are given by
\begin{subequations}
  \label{eq:angFuncs}
  \begin{align}
    |\angAmp[h]|^2
      &\propto \aLp{J_a^{hh'}}{0}(\cthetaK)\,\sph{J_b^{hh'}}{0}{*}(\thetal,\phihel) \nonumber\\
      &\qquad= \aLp{J_a^{hh'}}{0}(\cthetaK)\, \rsph{J_b^{hh'}}{0}(\thetal,\phihel) \\
    2\,\Re(\angAmp[h]^*\,\angAmp[h'])
      &\propto 2\,\Re\!\left[ \aLp{J_a^{hh'}}{|M^{hh'}|}(\cthetaK)\,\sph{J_b^{hh'}}{|M^{hh'}|}{*}(\thetal,\phihel) \right] \nonumber\\
      &\qquad= \sqrt{2}\, \aLp{J_a^{hh'}}{|M^{hh'}|}(\cthetaK)\, \rsph{J_b^{hh'}}{|M^{hh'}|}(\thetal,\phihel) \\
    -2\,\Im(\angAmp[h]^*\,\angAmp[h'])
      &\propto -2\,\Im\!\left[ \aLp{J_a^{hh'}}{|M^{hh'}|}(\cthetaK)\,\sph{J_b^{hh'}}{|M^{hh'}|}{*}(\thetal,\phihel) \right] \nonumber\\
      &\qquad= \sqrt{2}\, \aLp{J_a^{hh'}}{|M^{hh'}|}(\cthetaK)\, \rsph{J_b^{hh'}}{-|M^{hh'}|}(\thetal,\phihel)
  \end{align}
\end{subequations}

%%%%%%%%%%%%%%%%%%%%%%%%%%%%%%%%%%%%%%%%%%%
\section{Angular Functions for \texorpdfstring{\BstoJpsiKK}{Bs0->J/psi K+K-}}
\label{sec:angularDecay_functions}
%%%%%%%%%%%%%%%%%%%%%%%%%%%%%%%%%%%%%%%%%%%

Identifying particles $B$, $a$, and $b$ with the $\Bs$, the $\KK$ pair, and the $\Jpsi$, respectively, the angular dependence of the
\BstoJpsiKK{} decay can be derived from Equations~\ref{eq:angAmpSqEval}, \ref{eq:Eultohel}, \ref{eq:sqAmpHel}, and \ref{eq:angFuncs}. As
discussed in Section~\ref{sec:pheno_decay}, the indices $h$ and $h'$ in Equation~\ref{eq:sqAmpHel} run over three \BstoJpsiphi{}
polarization states and one \BstoJpsiKK{} state where the $\KK$ pair is in an S-wave configuration.

Both the spins of the $\phimesalt$ and the $\Jpsi$ are equal to one ($j_a$\texteq$j_b$\texteq1), which results in sums over
$J_a$,\,$J_b$\texteq0,\,1,\,2 for each combination of the \BstoJpsiphi{} helicity states. The three states are given by the possible
$\phimesalt$ and $\Jpsi$ helicities: $\lambda$\texteq0 (``0''), $\lambda$\texteq\tp1 (``\tp''), and $\lambda$\texteq\tm1 (``\tm'').
Since the $\KK$ system has no orbital angular momentum for the $\KK$ S-wave (``S'') and the kaons are spinless, both $j_a$ and $\lambda$
are equal to zero for this state. The value of $J_a$ can only be zero for $h$\texteq$h'$\texteq{}S and one for $h$\textneq{}S and
$h'$\texteq{}S. Only one value for $|M^{hh'}|$\textequiv$|\lambda^h$\textminus$\lambda^{h'}|$ is possible for each combination of states,
varying from zero ($\lambda^h$\texteq$\lambda^{h'}$\texteq0) to two ($\lambda^h$\texteq\tm$\lambda^{h'}$\texteq1).

Kaons are spinless, which results in $\alpha$\texteq0 for the \BstoJpsiKK{} decay. The helicities of the muons from the $\Jpsi$ decay can
both be either \tp$\frac{\text{1}}{\text{2}}$ or \tm$\frac{\text{1}}{\text{2}}$, which gives the combinations $\beta$\textin\{0, \tpm1\}.
Since the $\mup$ and the $\mum$ are produced with opposite chirality, the case $\beta$\texteq0 requires the helicity of one of the muons to
be opposite to its chirality. These contributions are suppressed by a factor
$m_{\lmu[]}^2$/$m_{\mumu}^2$\textapprox\tenpow{-3}~\cite{Altmannshofer:2008dz}, where $m_{\lmu[]}$ is the muon mass and $m_{\mumu}$
the dimuon invariant mass, the latter of which is equal to the mass of the $\Jpsi$. For this reason only contributions with opposite
helicities are considered, for which $\beta$\texteq\tpm1.

\begin{table}[p]
  \centering
  \caption{Angular functions for the \BstoJpsiphi{} decay expressed in terms of associated Legendre polynomials and spherical harmonics
           in helicity angles. Functions are shown for $\beta$\texteq\tpm1.
           Top: functions in the helicity basis. Bottom: functions in the transversity basis.}
  \label{tab:angDistJpsiphiPY}
  \renewcommand{\arraystretch}{1.2}
  \begin{tabular}{cc}
    \hline
    amplitudes                             &
    %$hh'$                                  &
      $f(\Omega) \times 16\sqrt{\pi}$      \\
      %$f(\Omega) \times \tfrac{32\pi}{9}$  \\

    \hline

    $\AmpSq[H]{0}$  &
    %00  &
      $4\, (P_0^0 + 2\, P_2^0)\, (Y_{0,\,0} - \tfrac{1}{\sqrt{5}}\, Y_{2,\,0})$  \\
      %$2\, \cos^2\thetaK\, \sin^2\thetal$  \\

    $\AmpSq[H]{+}$  &
    %\tp\tp  &
      $2\, (P_0^0 - P_2^0)\, (2\, Y_{0,\,0} + \tfrac{1}{\sqrt{5}}\, Y_{2,\,0} \pm \sqrt{3}\, Y_{1,\,0})$  \\
      %$\tfrac{1}{2}\, \sin^2\thetaK\, (1 \pm \cos\thetal)^2$  \\
      &
      $\qquad\quad = P_2^2\, (2\, Y_{0,\,0} + \tfrac{1}{\sqrt{5}}\, Y_{2,\,0} \pm \sqrt{3}\, Y_{1,\,0})$  \\

    $\AmpSq[H]{-}$  &
    %\tm{}\tm  &
      $2\, (P_0^0 - P_2^0)\, (2\, Y_{0,\,0} + \tfrac{1}{\sqrt{5}}\, Y_{2,\,0} \mp \sqrt{3}\, Y_{1,\,0})$  \\
      %$\tfrac{1}{2}\, \sin^2\thetaK\, (1 \mp \cos\thetal)^2$  \\
      &
      $\qquad\quad = P_2^2\, (2\, Y_{0,\,0} + \tfrac{1}{\sqrt{5}}\, Y_{2,\,0} \mp \sqrt{3}\, Y_{1,\,0})$  \\

    $\ReAmp[H][H]{0}{+}$  &
    %0\tp{} ($\Re$)  &
      $+2\sqrt{\tfrac{3}{5}}\, P_2^1 (Y_{2,\,+1} \pm \sqrt{5}\, Y_{1,\,+1})$  \\
      %$\pm \sin2\thetaK\, \sin\thetal\, (1 \pm \cos\thetal) \cos\phihel$  \\

    $\ImAmp[H][H]{0}{+}$  &
    %0\tp{} ($\Im$)  &
      $-2\sqrt{\tfrac{3}{5}}\, P_2^1 (Y_{2,\,-1} \pm \sqrt{5}\, Y_{1,\,-1})$  \\
      %$\mp \sin2\thetaK\, \sin\thetal\, (1 \pm \cos\thetal) \sin\phihel$  \\

    $\ReAmp[H][H]{0}{-}$  &
    %0\tm{} ($\Re$)  &
      $+2\sqrt{\tfrac{3}{5}}\, P_2^1 (Y_{2,\,+1} \mp \sqrt{5}\, Y_{1,\,+1})$  \\
      %$\mp \sin2\thetaK\, \sin\thetal\, (1 \mp \cos\thetal) \cos\phihel$  \\

    $\ImAmp[H][H]{0}{-}$  &
    %0\tm{} ($\Im$)  &
      $+2\sqrt{\tfrac{3}{5}}\, P_2^1 (Y_{2,\,-1} \mp \sqrt{5}\, Y_{1,\,-1})$  \\
      %$\mp \sin2\thetaK\, \sin\thetal\, (1 \mp \cos\thetal) \sin\phihel$  \\

    $\ReAmp[H][H]{+}{-}$  &
    %\tp\tm{} ($\Re$)  &
      $-2\sqrt{\tfrac{3}{5}}\, P_2^2\, Y_{2,\,+2}$  \\
      %$-\sin^2\thetaK\, \sin^2\thetal\, \cos2\phihel$  \\

    $\ImAmp[H][H]{+}{-}$  &
    %\tp\tm{} ($\Im$)  &
      $-2\sqrt{\tfrac{3}{5}}\, P_2^2\, Y_{2,\,-2}$  \\
      %$-\sin^2\thetaK\, \sin^2\thetal\, \sin2\phihel$  \\
    \hline

    $\AmpSq{0}$  &
    %00  &
      $4\, (P_0^0 + 2\, P_2^0)\, (Y_{0,\,0} - \tfrac{1}{\sqrt{5}}\, Y_{2,\,0})$  \\
      %$2\, \cos^2\thetaK\, \sin^2\thetal$  \\

    $\AmpSq{\parallel}$  &
    %$\parallel\parallel$  &
      $P_2^2\, (2\, Y_{0,\,0} + \tfrac{1}{\sqrt{5}}\, Y_{2,\,0} - \sqrt{\tfrac{3}{5}}\, Y_{2,\,+2})$  \\
      %$\sin^2\thetaK\, (1 - \sin^2\thetal\, \cos^2\phihel)$  \\

    $\AmpSq{\perp}$  &
    %$\perp\perp$  &
      $P_2^2\, (2\, Y_{0,\,0} + \tfrac{1}{\sqrt{5}}\, Y_{2,\,0} + \sqrt{\tfrac{3}{5}}\, Y_{2,\,+2})$  \\
      %$\sin^2\thetaK\, (1 - \sin^2\thetal\, \sin^2\phihel)$  \\

    $\ReAmp{0}{\parallel}$  &
    %0$\parallel$ ($\Re$)  &
      $+2\sqrt{2}\sqrt{\tfrac{3}{5}}\, P_2^1\, Y_{2,\,+1}$  \\
      %$+\frac{1}{\sqrt{2}}\, \sin2\thetaK\, \sin2\thetal\, \cos\phihel$  \\

    $\ImAmp{0}{\parallel}$  &
    %0$\parallel$ ($\Im$)  &
      $\mp 2\sqrt{2}\;\sqrt{3}\, P_2^1\, Y_{1,\,-1}$  \\
      %$\mp \sqrt{2}\, \sin2\thetaK\, \sin\thetal\, \sin\phihel$  \\

    $\ReAmp{0}{\perp}$  &
    %0$\perp$ ($\Re$)  &
      $\pm 2\sqrt{2}\;\sqrt{3}\, P_2^1\, Y_{1,\,+1}$  \\
      %$\pm \sqrt{2}\, \sin2\thetaK\, \sin\thetal\, \cos\phihel$  \\

    $\ImAmp{0}{\perp}$  &
    %0$\perp$ ($\Im$)  &
      $-2\sqrt{2}\sqrt{\tfrac{3}{5}}\, P_2^1\, Y_{2,\,-1}$  \\
      %$-\frac{1}{\sqrt{2}}\, \sin2\thetaK\, \sin2\thetal\, \sin\phihel$  \\

    $\ReAmp{\parallel}{\perp}$  &
    %$\parallel\perp$ ($\Re$)  &
      $\pm 2\sqrt{3}\, P_2^2\, Y_{1,\,0}$  \\
      %$\pm 2\, \sin^2\thetaK\, \cos\thetal$  \\

    $\ImAmp{\parallel}{\perp}$  &
    %$\parallel\perp$ ($\Im$)  &
      $+2\sqrt{\tfrac{3}{5}}\, P_2^2\, Y_{2,\,-2}$  \\
      %$+\sin^2\thetaK\, \sin^2\thetal\, \sin2\phihel$  \\
    \hline
  \end{tabular}
\end{table}

\begin{table}[p]
  \centering
  \caption{Angular functions for the \BstoJpsiKK{} decay with a $\KK$ S-wave and the \BstoJpsiphi{} and $\KK$ S-wave interference
           expressed in terms of associated Legendre polynomials and spherical harmonics in helicity angles.
           Functions are shown for $\beta$\texteq\tpm1.
           Top: functions in the helicity basis. Bottom: functions in the transversity basis.}
  \renewcommand{\arraystretch}{1.2}
  \label{tab:angDistSWavePY}
  \begin{tabular}{cc}
    \hline
    amplitudes                             &
      $f(\Omega) \times 16\sqrt{\pi}$      \\
      %$f(\Omega) \times \tfrac{32\pi}{9}$  \\

    \hline

    $\AmpSq[H]{{\text{S}}}$  &
      $4\, P_0^0\, (Y_{0,\,0} - \tfrac{1}{\sqrt{5}}\, Y_{2,\,0})$  \\
      %$\tfrac{2}{3}\, \sin^2\thetal$  \\

    $\ReAmp[H][H]{0}{{\text{S}}}$  &
      $8\sqrt{3}\, P_1^0\, (Y_{0,\,0} - \tfrac{1}{\sqrt{5}}\, Y_{2,\,0})$  \\
      %$\tfrac{4}{3}\sqrt{3}\, \cos\thetaK\, \sin^2\thetal$  \\

    $\ImAmp[H][H]{0}{{\text{S}}}$  &
      0  \\
      %0  \\

    $\ReAmp[H][H]{+}{{\text{S}}}$  &
      $+6\, P_1^1\, (\tfrac{1}{\sqrt{5}}\, Y_{2,\,+1} \pm Y_{1,\,+1})$  \\
      %$\pm \tfrac{2}{3}\sqrt{3}\, \sin\thetaK\, \sin\thetal\, (1 \pm \cos\thetal)\, \cos\phihel$ \\

    $\ImAmp[H][H]{+}{{\text{S}}}$  &
      $+6\, P_1^1\, (\tfrac{1}{\sqrt{5}}\, Y_{2,\,-1} \pm Y_{1,\,-1})$  \\
      %$\pm \tfrac{2}{3}\sqrt{3}\, \sin\thetaK\, \sin\thetal\, (1 \pm \cos\thetal)\, \sin\phihel$ \\

    $\ReAmp[H][H]{-}{{\text{S}}}$  &
      $+6\, P_1^1\, (\tfrac{1}{\sqrt{5}}\, Y_{2,\,+1} \mp Y_{1,\,+1})$  \\
      %$\mp \tfrac{2}{3}\sqrt{3}\, \sin\thetaK\, \sin\thetal\, (1 \mp \cos\thetal)\, \cos\phihel$ \\

    $\ImAmp[H][H]{-}{{\text{S}}}$  &
      $-6\, P_1^1\, (\tfrac{1}{\sqrt{5}}\, Y_{2,\,-1} \mp Y_{1,\,-1})$  \\
      %$\pm \tfrac{2}{3}\sqrt{3}\, \sin\thetaK\, \sin\thetal\, (1 \mp \cos\thetal)\, \sin\phihel$ \\
    \hline

    $\AmpSq{{\text{S}}}$  &
      $4\, P_0^0\, (Y_{0,\,0} - \tfrac{1}{\sqrt{5}}\, Y_{2,\,0})$  \\
      %$\tfrac{2}{3}\, \sin^2\thetal$  \\

    $\ReAmp{0}{{\text{S}}}$  &
      $8\sqrt{3}\, P_1^0\, (Y_{0,\,0} - \tfrac{1}{\sqrt{5}}\, Y_{2,\,0})$  \\
      %$\tfrac{4}{3}\sqrt{3}\, \cos\thetaK\, \sin^2\thetal$  \\

    $\ImAmp{0}{{\text{S}}}$  &
      0  \\
      %0  \\

    $\ReAmp{\parallel}{{\text{S}}}$  &
      $+6\sqrt{2}\tfrac{1}{\sqrt{5}}\, P_1^1\, Y_{2,\,+1}$  \\
      %$+\tfrac{1}{3}\sqrt{6}\, \sin\thetaK\, \sin2\thetal\, \cos\phihel$  \\

    $\ImAmp{\parallel}{{\text{S}}}$  &
      $\pm 6\sqrt{2}\, P_1^1\, Y_{1,\,-1}$  \\
      %$\pm \tfrac{2}{3}\sqrt{6}\, \sin\thetaK\, \sin\thetal\, \sin\phihel$  \\

    $\ReAmp{\perp}{{\text{S}}}$  &
      $\pm 6\sqrt{2}\, P_1^1\, Y_{1,\,+1}$  \\
      %$\pm \tfrac{2}{3}\sqrt{6}\, \sin\thetaK\, \sin\thetal\, \cos\phihel$  \\

    $\ImAmp{\perp}{{\text{S}}}$  &
      $+6\sqrt{2}\tfrac{1}{\sqrt{5}}\, P_1^1\, Y_{2,\,-1}$  \\
      %$+\tfrac{1}{3}\sqrt{6}\, \sin\thetaK\, \sin2\thetal\, \sin\phihel$  \\
    \hline
  \end{tabular}
\end{table}

Evaluation of the angular functions for the combinations of the \BstoJpsiphi{} states leads to Table~\ref{tab:angDistJpsiphiPY}. The first
column of the table shows a combination of coefficients (helicity and transversity amplitudes) and the second column the corresponding
angular functions. The angular functions for the $\KK$ S-wave and the interference between \BstoJpsiphi{} and the $\KK$ S-wave are listed
in Table~\ref{tab:angDistSWavePY}. All functions are multiplied by a factor 8$\pi^\text{2}$, which makes their integrals over all three
angles equal to either one or zero.

The top halves of the tables show the angular functions in the helicity basis, while the bottom halves show the functions in the
transversity basis. For both bases helicity angles are used. The transversity functions are obtained by substituting the helicity
amplitudes ($H_0$, $H_+$, and $H_-$) by combinations of transversity amplitudes ($\Azero$, $\Apar$, and $\Aperp$), after which the helicity
functions can be combined into functions corresponding to the transversity amplitudes. The amplitude for the $\KK$ S-wave is the same in
both bases ($H_\text{S}$\texteq$\AS$). The substitution is given by~\cite{Dighe:1995pd}
\begin{equation}
  \label{eq:helToTransAmps}
  \begin{aligned}
    H_0   &= \Azero \\
    H_\pm &= \tfrac{1}{\sqrt{2}}(\Apar \pm \Aperp) \ .
  \end{aligned}
\end{equation}

The differences between contributions from $\beta$\texteq\tpm1 are distinguished with \tpm{} and \tmp{} signs in the tables. Because the
decay of the $\Jpsi$ is not a weak process it conserves parity, which results in equal helicity/transversity amplitudes for
$\beta$\texteq\tp1 and $\beta$\texteq\tm1. Since the muon helicities are not measured in the experiment, the two contributions are added,
resulting in cancellation of the terms with opposite sign.

Both Legendre polynomials and real-valued spherical harmonics can be expressed in terms of sines and cosines of the angles.
Tables~\ref{tab:angDistJpsiphiSinCos} and \ref{tab:angDistSWaveSinCos} show the angular functions in terms of sines and cosines that
correspond to the functions in Tables~\ref{tab:angDistJpsiphiPY} and \ref{tab:angDistSWavePY}, respectively.

\begin{table}[p]
  \centering
  \caption{Angular functions for the \BstoJpsiphi{} decay expressed in terms of sines and cosines in helicity angles.
           Functions are shown for $\beta$\texteq\tpm1.
           Top: functions in the helicity basis. Bottom: functions in the transversity basis.}
  \label{tab:angDistJpsiphiSinCos}
  \renewcommand{\arraystretch}{1.2}
  \begin{tabular}{cc}
    \hline
    amplitudes                             &
    %$hh'$                                  &
      %$f(\Omega) \times 16\sqrt{\pi}$      &
      $f(\Omega) \times \tfrac{32\pi}{9}$  \\

    \hline

    $\AmpSq[H]{0}$  &
    %00  &
      %$4\, (P_0^0 + 2\, P_2^0)\, (Y_{0,\,0} - \tfrac{1}{\sqrt{5}}\, Y_{2,\,0})$  &
      $2\, \cos^2\thetaK\, \sin^2\thetal$  \\

    $\AmpSq[H]{+}$  &
    %\tp\tp  &
      %$2\, (P_0^0 - P_2^0)\, (2\, Y_{0,\,0} + \tfrac{1}{\sqrt{5}}\, Y_{2,\,0} \pm \sqrt{3}\, Y_{1,\,0})$  &
      $\tfrac{1}{2}\, \sin^2\thetaK\, (1 \pm \cos\thetal)^2$  \\

    $\AmpSq[H]{-}$  &
    %\tm{}\tm  &
      %$2\, (P_0^0 - P_2^0)\, (2\, Y_{0,\,0} + \tfrac{1}{\sqrt{5}}\, Y_{2,\,0} \mp \sqrt{3}\, Y_{1,\,0})$  &
      $\tfrac{1}{2}\, \sin^2\thetaK\, (1 \mp \cos\thetal)^2$  \\

    $\ReAmp[H][H]{0}{+}$  &
    %0\tp{} ($\Re$)  &
      %$+2\sqrt{\tfrac{3}{5}}\, P_2^1 (Y_{2,\,+1} \pm \sqrt{5}\, Y_{1,\,+1})$  &
      $\pm \sin2\thetaK\, \sin\thetal\, (1 \pm \cos\thetal) \cos\phihel$  \\

    $\ImAmp[H][H]{0}{+}$  &
    %0\tp{} ($\Im$)  &
      %$-2\sqrt{\tfrac{3}{5}}\, P_2^1 (Y_{2,\,-1} \pm \sqrt{5}\, Y_{1,\,-1})$  &
      $\mp \sin2\thetaK\, \sin\thetal\, (1 \pm \cos\thetal) \sin\phihel$  \\

    $\ReAmp[H][H]{0}{-}$  &
    %0\tm{} ($\Re$)  &
      %$+2\sqrt{\tfrac{3}{5}}\, P_2^1 (Y_{2,\,+1} \mp \sqrt{5}\, Y_{1,\,+1})$  &
      $\mp \sin2\thetaK\, \sin\thetal\, (1 \mp \cos\thetal) \cos\phihel$  \\

    $\ImAmp[H][H]{0}{-}$  &
    %0\tm{} ($\Im$)  &
      %$+2\sqrt{\tfrac{3}{5}}\, P_2^1 (Y_{2,\,-1} \mp \sqrt{5}\, Y_{1,\,-1})$  &
      $\mp \sin2\thetaK\, \sin\thetal\, (1 \mp \cos\thetal) \sin\phihel$  \\

    $\ReAmp[H][H]{+}{-}$  &
    %\tp\tm{} ($\Re$)  &
      %$-2\sqrt{\tfrac{3}{5}}\, P_2^2\, Y_{2,\,+2}$  &
      $-\sin^2\thetaK\, \sin^2\thetal\, \cos2\phihel$  \\

    $\ImAmp[H][H]{+}{-}$  &
    %\tp\tm{} ($\Im$)  &
      %$-2\sqrt{\tfrac{3}{5}}\, P_2^2\, Y_{2,\,-2}$  &
      $-\sin^2\thetaK\, \sin^2\thetal\, \sin2\phihel$  \\
    \hline

    $\AmpSq{0}$  &
    %00  &
      %$4\, (P_0^0 + 2\, P_2^0)\, (Y_{0,\,0} - \tfrac{1}{\sqrt{5}}\, Y_{2,\,0})$  &
      $2\, \cos^2\thetaK\, \sin^2\thetal$  \\

    $\AmpSq{\parallel}$  &
    %$\parallel\parallel$  &
      %$P_2^2\, (2\, Y_{0,\,0} + \tfrac{1}{\sqrt{5}}\, Y_{2,\,0} - \sqrt{\tfrac{3}{5}}\, Y_{2,\,+2})$  &
      $\sin^2\thetaK\, (1 - \sin^2\thetal\, \cos^2\phihel)$  \\

    $\AmpSq{\perp}$  &
    %$\perp\perp$  &
      %$P_2^2\, (2\, Y_{0,\,0} + \tfrac{1}{\sqrt{5}}\, Y_{2,\,0} + \sqrt{\tfrac{3}{5}}\, Y_{2,\,+2})$  &
      $\sin^2\thetaK\, (1 - \sin^2\thetal\, \sin^2\phihel)$  \\

    $\ReAmp{0}{\parallel}$  &
    %0$\parallel$ ($\Re$)  &
      %$+2\sqrt{2}\sqrt{\tfrac{3}{5}}\, P_2^1\, Y_{2,\,+1}$  &
      $+\frac{1}{\sqrt{2}}\, \sin2\thetaK\, \sin2\thetal\, \cos\phihel$  \\

    $\ImAmp{0}{\parallel}$  &
    %0$\parallel$ ($\Im$)  &
      %$\mp 2\sqrt{2}\;\sqrt{3}\, P_2^1\, Y_{1,\,-1}$  &
      $\mp \sqrt{2}\, \sin2\thetaK\, \sin\thetal\, \sin\phihel$  \\

    $\ReAmp{0}{\perp}$  &
    %0$\perp$ ($\Re$)  &
      %$\pm 2\sqrt{2}\;\sqrt{3}\, P_2^1\, Y_{1,\,+1}$  &
      $\pm \sqrt{2}\, \sin2\thetaK\, \sin\thetal\, \cos\phihel$  \\

    $\ImAmp{0}{\perp}$  &
    %0$\perp$ ($\Im$)  &
      %$-2\sqrt{2}\sqrt{\tfrac{3}{5}}\, P_2^1\, Y_{2,\,-1}$  &
      $-\frac{1}{\sqrt{2}}\, \sin2\thetaK\, \sin2\thetal\, \sin\phihel$  \\

    $\ReAmp{\parallel}{\perp}$  &
    %$\parallel\perp$ ($\Re$)  &
      %$\pm 2\sqrt{3}\, P_2^2\, Y_{1,\,0}$  &
      $\pm 2\, \sin^2\thetaK\, \cos\thetal$  \\

    $\ImAmp{\parallel}{\perp}$  &
    %$\parallel\perp$ ($\Im$)  &
      %$+2\sqrt{\tfrac{3}{5}}\, P_2^2\, Y_{2,\,-2}$  &
      $+\sin^2\thetaK\, \sin^2\thetal\, \sin2\phihel$  \\
    \hline
  \end{tabular}
\end{table}

\begin{table}[p]
  \centering
  \caption{Angular functions for the \BstoJpsiKK{} decay with a $\KK$ S-wave and the \BstoJpsiphi{} and $\KK$ S-wave interference
           expressed in terms of sines and cosines in helicity angles. Functions are shown for $\beta$\texteq\tpm1.
           Top: functions in the helicity basis. Bottom: functions in the transversity basis.}
  \renewcommand{\arraystretch}{1.2}
  \label{tab:angDistSWaveSinCos}
  \begin{tabular}{cc}
    \hline
    amplitudes                              &
      %$f(\Omega) \times 16\sqrt{\pi}$      &
      $f(\Omega) \times \tfrac{32\pi}{9}$  \\

    \hline

    $\AmpSq[H]{{\text{S}}}$  &
      %$4\, P_0^0\, (Y_{0,\,0} - \tfrac{1}{\sqrt{5}}\, Y_{2,\,0})$  &
      $\tfrac{2}{3}\, \sin^2\thetal$  \\

    $\ReAmp[H][H]{0}{{\text{S}}}$  &
      %$8\sqrt{3}\, P_1^0\, (Y_{0,\,0} - \tfrac{1}{\sqrt{5}}\, Y_{2,\,0})$  &
      $\tfrac{4}{3}\sqrt{3}\, \cos\thetaK\, \sin^2\thetal$  \\

    $\ImAmp[H][H]{0}{{\text{S}}}$  &
      %0  &
      0  \\

    $\ReAmp[H][H]{+}{{\text{S}}}$  &
      %$+6\, P_1^1\, (\tfrac{1}{\sqrt{5}}\, Y_{2,\,+1} \pm Y_{1,\,+1})$  &
      $\pm \tfrac{2}{3}\sqrt{3}\, \sin\thetaK\, \sin\thetal\, (1 \pm \cos\thetal)\, \cos\phihel$ \\

    $\ImAmp[H][H]{+}{{\text{S}}}$  &
      %$+6\, P_1^1\, (\tfrac{1}{\sqrt{5}}\, Y_{2,\,-1} \pm Y_{1,\,-1})$  &
      $\pm \tfrac{2}{3}\sqrt{3}\, \sin\thetaK\, \sin\thetal\, (1 \pm \cos\thetal)\, \sin\phihel$ \\

    $\ReAmp[H][H]{-}{{\text{S}}}$  &
      %$+6\, P_1^1\, (\tfrac{1}{\sqrt{5}}\, Y_{2,\,+1} \mp Y_{1,\,+1})$  &
      $\mp \tfrac{2}{3}\sqrt{3}\, \sin\thetaK\, \sin\thetal\, (1 \mp \cos\thetal)\, \cos\phihel$ \\

    $\ImAmp[H][H]{-}{{\text{S}}}$  &
      %$-6\, P_1^1\, (\tfrac{1}{\sqrt{5}}\, Y_{2,\,-1} \mp Y_{1,\,-1})$  &
      $\pm \tfrac{2}{3}\sqrt{3}\, \sin\thetaK\, \sin\thetal\, (1 \mp \cos\thetal)\, \sin\phihel$ \\
    \hline

    $\AmpSq{{\text{S}}}$  &
      %$4\, P_0^0\, (Y_{0,\,0} - \tfrac{1}{\sqrt{5}}\, Y_{2,\,0})$  &
      $\tfrac{2}{3}\, \sin^2\thetal$  \\

    $\ReAmp{0}{{\text{S}}}$  &
      %$8\sqrt{3}\, P_1^0\, (Y_{0,\,0} - \tfrac{1}{\sqrt{5}}\, Y_{2,\,0})$  &
      $\tfrac{4}{3}\sqrt{3}\, \cos\thetaK\, \sin^2\thetal$  \\

    $\ImAmp{0}{{\text{S}}}$  &
      %0  &
      0  \\

    $\ReAmp{\parallel}{{\text{S}}}$  &
      %$+6\sqrt{2}\tfrac{1}{\sqrt{5}}\, P_1^1\, Y_{2,\,+1}$  &
      $+\tfrac{1}{3}\sqrt{6}\, \sin\thetaK\, \sin2\thetal\, \cos\phihel$  \\

    $\ImAmp{\parallel}{{\text{S}}}$  &
      %$\pm 6\sqrt{2}\, P_1^1\, Y_{1,\,-1}$  &
      $\pm \tfrac{2}{3}\sqrt{6}\, \sin\thetaK\, \sin\thetal\, \sin\phihel$  \\

    $\ReAmp{\perp}{{\text{S}}}$  &
      %$\pm 6\sqrt{2}\, P_1^1\, Y_{1,\,+1}$  &
      $\pm \tfrac{2}{3}\sqrt{6}\, \sin\thetaK\, \sin\thetal\, \cos\phihel$  \\

    $\ImAmp{\perp}{{\text{S}}}$  &
      %$+6\sqrt{2}\tfrac{1}{\sqrt{5}}\, P_1^1\, Y_{2,\,-1}$  &
      $+\tfrac{1}{3}\sqrt{6}\, \sin\thetaK\, \sin2\thetal\, \sin\phihel$  \\
    \hline
  \end{tabular}
\end{table}

\section{Invariant \texorpdfstring{$\KK$}{KK}-Mass Distribution}
\label{sec:pheno_KKMass}

Besides the dependence on decay-time via $\timeAmp$, the angular coefficients in Equation~\ref{eq:sqAmpExpand} also depend on $\KK$ mass
via $\mKKAmp$. This dependence is described by a phenomenological model. To simplify the analysis of the time and angular distributions,
the model for the \BstoJpsiKK{} decay is integrated over the $\KK$ mass. As will be discussed in
Sections~\ref{subsec:pheno_equations_symmetry} and \ref{sec:ana_KKIntegrals}, the model is split into mass bins, for which separate
integrals are calculated.

As discussed in Section~\ref{subsec:intro_Jpsiphi_final}, the $\KK$ pair is required to have an invariant mass between 990 and
1050\unitsp\MeV. The plot of the $\KK$-mass spectrum in Figure~\ref{fig:KKComponents} shows that the contribution of the $\phimes$
dominates in this region, but that there are also $\fzero$ and non-resonant contributions.

The $\KK$-mass integral gives a factor for each term in Equation~\ref{eq:sqAmp}, which is, in principle, different for each combination of
intermediate states. However, because the dependence on $\KK$ mass is approximately equal for the three \BstoJpsiphi{} states, this
contribution can be factored out as an overall normalization and only relative factors between the $\KK$ S-wave and the \BstoJpsiphi{}
contributions remain (see also \cite{Azfar:2010nz,LHCb-PAPER-2013-002,*LHCb-ANA-2012-067}).

Before integrating the expressions for $\mKKAmp^*\,\mKKAmp[j]$ over $\mKK$ they are multiplied by the factor $|\pJpsi|\,|\pK|$ from
Equation~\ref{eq:diffRateAngles}. The dependence of this factor on $\mKK$ can be derived from the kinematic relations in the ``two-body
decays'' of the $\Bs$ and the $\KK$ pair. The expressions for the magnitudes of the $\Jpsi$ momentum in the $\Bs$ rest frame and the
$\Kp$ momentum in the $\KK$ rest frame depend only on the $\Bs$, $\Jpsi$, $\KK$, and $\Kp$ masses. Working out the kinematic relations, the
factor $|\pJpsi|\,|\pK|$ is given by
\begin{equation}
  |\pJpsi|\,|\pK| = \frac{\sqrt{\lambda(\mBs,\mJpsi,\mKK)\, \lambda(\mKK,\mK,\mK)}}{4\,\mBs\,\mKK}\ ,
\end{equation}
with
\begin{equation}
  \lambda(M,m_1,m_2) \equiv M^4 + m_1^4 + m_2^4 - 2\,M^2 m_1^2 - 2\,M^2 m_2^2 - 2\,m_1^2 m_2^2\ .
\end{equation}

The functions $\mKKAmp(\mKK)$ can be normalized by absorbing a constant factor into the corresponding transversity amplitude $\Ai$, such
that the integrals for diagonal terms are given by $\int\ud\mKK\,|\pJpsi|\,|\pK|\,|\mKKAmp|^2$\textequiv1. This procedure moves the issue
of overall normalization to the transversity amplitudes and leaves only non-trivial $\mKK$ integrals in the interference terms. These
integrals are given by (the real and imaginary parts of)
\begin{equation}
  \label{eq:couplingFac}
  \int_{\mKK^-}^{\mKK^+}\ud\mKK\,|\pJpsi|\,|\pK|\,\mKKAmp^*\,\mKKAmp[j] \equiv \Kij\, e^{-i\kij} \ ,
\end{equation}
where $\mKK^-$\texteq990\unitsp\MeV, $\mKK^+$\texteq1050\unitsp\MeV. The real-valued parameters $\Kij$ and $\kij$ form the $\KK$-mass
factor for interference term $ij$. By construction, the conditions $\Kij[i][i]$\texteq1 and $\kij[i][i]$\texteq0 apply. With the
Cauchy--Schwarz inequality, $|\int\ud x\, A^*B\,|^2$\textle$\int\ud x\,|A|^2\cdot\int\ud x\,|B|^2$, it follows that 0\textle$\Kij$\textle1.

Because only the magnitude of the \BstoJpsiKK{} decay amplitude is observable, the overall phase of the transversity amplitudes is
arbitrary and only phase differences between the individual amplitudes can be observed. Integrating over $\mKK$, the phases $\kij$ can be
absorbed into these phase differences:
\begin{equation}
  \label{eq:couplingFacAmps}
  \begin{aligned}
    \int_{\mKK^-}^{\mKK^+}\ud\mKK\,|\pJpsi|\,|\pK|\, \Ai^*\Ai[j]\, \mKKAmp^*\,\mKKAmp[j]
      &= \Ai^*\Ai[j]\, \Kij\, e^{-i\kij} \\
      &= \Kij\, |\Ai|\,|\Ai[j]|\, e^{i[\arg(\Ai[j]) - \arg(\Ai) - \kij]} \ .
  \end{aligned}
\end{equation}

The interference terms become proportional to the \emph{coupling factors} between states $i$ and $j$, $\Kij$, which are treated separately
from the transversity amplitudes.  Although these factors can in principle be determined from the angular distributions in the
\BstoJpsiKK{} data, their statistical uncertainties would be large. For this reason the $\Kij$ values are estimated by assuming
phenomenological models for the $\KK$-mass shapes of the different components of the decay and performing the integrals of
Equation~\ref{eq:couplingFac}.

It is assumed that the $\KK$-mass models only depend on the resonant state of the $\KK$ system and not on the angular momentum of the
$\JpsiKK$ system. That is, the same mass shape is used for all three angular-momentum states of the \BstoJpsiphi{} decay. See
references~\cite{Zhang:2012zk} and \cite{LHCb-PAPER-2012-040} for a discussion of the small dependence on the orbital angular momentum of
the $\JpsiKK$ system.

Because the $\KK$-mass shapes for the three \BstoJpsiphi{} intermediate states are identical, only one non-trivial $\Kij$ factor remains.
This factor is indicated by ``$\CSP$'', where ``P'' stands for ``P-wave'' and ``S'' for ``S-wave'', indicating the $\KK$ angular-momentum
states for the \BstoJpsiphi{} and $\KK$ S-wave contributions, respectively.

Similarly, the only remaining contribution to the phases of the transversity amplitudes is $\thetaSP$. The following phase differences are
defined:
\begin{equation}
  \label{eq:phaseDiffDef}
  \begin{alignedat}{2}
    \delparzero  &\equiv \arg(\Apar)  &&- \arg(\Azero) \\
    \delperpzero &\equiv \arg(\Aperp) &&- \arg(\Azero) \\
    \delSperp    &\equiv \arg(\AS)    &&- \arg(\Aperp) - \thetaSP \ .
  \end{alignedat}
\end{equation}
The remaining phase differences can be expressed in terms of the above three.

A relativistic Breit-Wigner function (see e.g.\ reference \cite{PDG}) is used to model the $\mKK$ dependence of the $\phimes$ component of
the $\KK$ system. The $\fzero$ contribution is modelled by a Flatt\'e function~\cite{Flatte:1976xu}. The non-resonant contribution, which
is small compared to the $\phimes$ and $\fzero$ contributions in the 990--1050~\MeV{} mass window, is neglected in this measurement.  See
Section~\ref{sec:ana_KKIntegrals} for the result of the $\CSP$ calculation, where also the experimental $\mKK$ resolution is taken into
account.

%\begin{description}
%\item[Centrifugal-barrier factors]
%The maximum orbital angular momentum in a two-body decay is limited by the linear momentum of the decay products in the rest frame of the
%decaying particle. \cite{Blatt:1952,*VonHippel:1972fg}
%
%\item[$\KK$-resonance mass shape]
%A different $\KK$-mass shape is included for each resonance of the $\KK$ system. For the $\phimes$ a relativistic Breit-Wigner shape for a
%spin-one resonance is used, which is given by~\cite{PDG}
%\begin{equation}
%  \label{eq:breitWigner}
%  \frac{1}{\mphi^2 - \mKK^2 - i\,\mphi\,\Gamma_\text{KK}(\mKK)}\ ,
%\end{equation}
%where $\mphi$ is the mass of the $\phimes$ meson and the mass-dependent width, $\Gamma_\text{KK}$, is given by
%\begin{equation}
%  \Gamma_\text{KK}(\mKK) = \Gamma_{\phimesalt}\, \left(\frac{|\pK|}{|\pK^0|}\right)^3\, \frac{\mphi}{\mKK}\, B_\phimesalt(|\pK|)\ ,
%\end{equation}
%where $\Gamma_{\phimesalt}$ is the width of the $\phimes$ meson, $\pK$ is the kaon three-momentum in the $\KK$ rest frame, and $\pK^0$ is
%the kaon three-momentum in the $\KK$ rest frame at the resonance mass ($\mKK$\texteq$\mphi$).
%
%Flatt\'e~\cite{Flatte:1976xu}
%
%\end{description}

\section{Decay-Time Distribution}
\label{sec:pheno_time}

The distribution in decay time follows from the products $\timeAmp^*\timeAmp[j]$ (see Equation~\ref{eq:amplitudeAmpLami}). With $|g_\pm|^2$
and $g_+^*\,g_-$ from Equation~\ref{eq:mixDecayQpSq}, these products are given by
\begin{align}
  \label{eq:timeDepProd}
  \timeAmp^*\timeAmp[j]
  &=\left[ |g_+|^2 + \lamsi^*\,\lamsi[j]\, |g_-|^2 + \lamsi^*\, (g_+^*\,g_-)^* + \lamsi[j]\, g_+^*\,g_- \right] \nonumber \\
  &=\tfrac{1}{2}\, \eGst\, \Big[    (1+\lamsi^*\lamsi[j]) \, \cDGs
                                +   (1-\lamsi^*\lamsi[j]) \, \cDms \nonumber \\
  &\quad\quad            -   (\lamsi^* + \lamsi[j])\,   \sDGs
                                - i (\lamsi^* - \lamsi[j])\,   \sDms \Big] \ .
\end{align}
Defining
\begin{equation}
  \label{eq:CDSPolarDep}
  \begin{aligned}
    \Cf[ij]^\pm    &\equiv \frac{1\pm\lamsi^*\lamsi[j]}{\sqrt{(1+\lamsiSq)(1+\lamsiSq[j])}} \\
    \qquad \Df[ij] &\equiv \frac{-    (\lamsi^* + \lamsi[j])}{\sqrt{(1+\lamsiSq)(1+\lamsiSq[j])}} \\
    \qquad \Sf[ij] &\equiv \frac{+i\, (\lamsi^* - \lamsi[j])}{\sqrt{(1+\lamsiSq)(1+\lamsiSq[j])}}
  \end{aligned}
\end{equation}
in accordance with Equation~\ref{eq:mixDecayCDS} and a ``CP-average'' decay amplitude
\begin{equation}
  \label{eq:CPAvAmp}
  \begin{gathered}
    \AAv \equiv \tfrac{1}{\sqrt{2}}\, \Ai\, \sqrt{1+\lamsiSq} \\
    \left|\AAv\right|^2
      = \tfrac{1}{2}\, |\Ai|^2\, \left(1 + \lamsiSq\right)
      = \tfrac{1}{2} \left(|\Ai|^2 + \qpAbsAlt^2\,|\Abari|^2\right) \ ,
  \end{gathered}
\end{equation}
the combination $\Ai^*\Ai[j]\,\timeAmp^*\timeAmp[j]$ can be expressed as
\begin{equation}
  \label{eq:timeDep}
  \begin{aligned}
    \Ai^*\Ai[j]\,\timeAmp^*\timeAmp[j]
      = \AAvConj&\AAv[{\Ai[j]}]\, \eGst \\
        \times\Big[ &\Cf[ij]^+\,\cDGs + \Cf[ij]^-\,\cDms \\
                    &+ \Df[ij]\,\sDGs - \Sf[ij]\,\sDms \Big] \ .
  \end{aligned}
\end{equation}

In general, the coefficients $\Cf[ij]^\pm$, $\Df[ij]$, and $\Sf[ij]$ are complex numbers and hence the terms
$\Ai^*\Ai[j]\,\timeAmp^*\timeAmp[j]$ have both real and imaginary parts. Notice that if $i=j$, the coefficients are real:
\begin{equation}
  \begin{alignedat}{2}
    \Cf[ii]^+ &= 1                                                    &  \Cf[ii]^- &= \frac{1-\lamsiSq}{1+\lamsiSq} \\
    \Df[ii]   &= -\frac{2\,\Re(\lamsi)}{1+\lamsiSq} \qquad\qquad  &  \Sf[ii]   &= \frac{2\,\Im(\lamsi)}{1+\lamsiSq} \ ,
  \end{alignedat}
\end{equation}
as required by the relation $\Ai^*\Ai^{\phantom{*}}\,\timeAmp^*\timeAmp^{\phantom{*}} = |\Ai\,\timeAmp|^2$.

The equivalent expression for the $\Bsbartof$ decay is obtained by changing the signs of the $\cDms$ and $\sDms$ terms and multiplying by a
factor $\pqAbsAlt^\text{2}$ (see Equations~\ref{eq:mixDecayAmps} and \ref{eq:mixDecayQpSq}). Using the definition of $\Cm$ from
Equation~\ref{eq:CmDef}, the time dependence of the $\Bstof$ and $\Bsbartof$ decays is given by
\begin{equation}
  \label{eq:timeqfDep}
  \begin{aligned}
    \Ai^*\Ai[j]\,\timeAmp[i,\qf]^*\timeAmp[j,\qf]^{\phantom{*}} =
      \AAvConj&\AAv[{\Ai[j]}]\, \frac{1-\qf\,\Cm}{1-\Cm}\, \eGst \\
          \times\Big[ &\Cf[ij]^+\,\cDGs + \qf\,\Cf[ij]^-\,\cDms \\
                      &+ \Df[ij]\,\sDGs - \qf\,\Sf[ij]\,\sDms \Big] \ ,
  \end{aligned}
\end{equation}
where the expressions have become dependent on the flavour of the decaying beauty meson, $\qf$ (see Equation~\ref{eq:mixDecayDiffRateAll}).
In Sections~\ref{sec:pheno_angles}, \ref{sec:pheno_KKMass}, and \ref{sec:pheno_equations} only the expression for the differential decay
rate of the $\Bs$ decay will be discussed, but the dependence on $\qf$ can be reintroduced at any point by multiplying the $\cDms$ and
$\sDms$ terms by $\qf$ and the total rate by $\frac{1-\qf\,\Cm}{1-\Cm}$.


%%%%%%%%%%%%%%%%%%%%%%%%%%%%%%%%%%%
\subsection{Common CP Violation}
\label{subsec:pheno_time_commonCPV}
%%%%%%%%%%%%%%%%%%%%%%%%%%%%%%%%%%%

In case CP symmetry is violated equally for all intermediate states, the common parameter $\lamsi[]$\textequiv$\frac{1}{\eta_i}\lamsi$ can
be defined, with $\phis$\texteq\tm$\arg(\lamsi[])$. The corresponding real-valued parameters in the decay-time distribution are then given
by
\begin{equation}
  \label{eq:CDSCommon}
         \Cs \equiv  \frac{1-\lamsiSq[]}{1+\lamsiSq[]}
  \qquad \Ds \equiv -\frac{2\,\Re(\lamsi[])}{1+\lamsiSq[]}
  \qquad \Ss \equiv  \frac{2\,\Im(\lamsi[])}{1+\lamsiSq[]}
  \ .
\end{equation}

For states $i$ and $j$ that are both CP even or both CP odd ($\eta_i$\texteq$\eta_j$), the coefficients from
Equations~\ref{eq:CDSPolarDep}, \ref{eq:timeDep}, and \ref{eq:timeqfDep} reduce to
\begin{subequations}
\label{eq:CDSPolarToCommon}
\begin{equation}
  \label{eq:CDSPolarToCommonEqual}
  \Cf[ij]^+ \to         1   \qquad
  \Cf[ij]^- \to         \Cs \qquad
  \Df[ij]   \to \eta_i\,\Ds \qquad
  \Sf[ij]   \to \eta_i\,\Ss \ ,
\end{equation}
Similarly, for $\eta_i$\texteq\tm$\eta_j$, the coefficients reduce to
\begin{equation}
  \label{eq:CDSPolarToCommonNotEqual}
  \Cf[ij]^+ \to             \Cs \qquad
  \Cf[ij]^- \to             1   \qquad
  \Df[ij]   \to  i\,\eta_i\,\Ss \qquad
  \Sf[ij]   \to -i\,\eta_i\,\Ds \ .
\end{equation}
\end{subequations}
Notice that in this special case the coefficients for $\eta_i$\texteq$\eta_j$ are all real, the coefficients $\Cf[ij]^\pm$ for
$\eta_i$\texteq\tm$\eta_j$ are real, and the coefficients $\Df[ij]$ and $\Sf[ij]$ for $\eta_i$\texteq\tm$\eta_j$ are imaginary. As a
result, simultaneously flipping the signs of $\Ds$ and $\sDGs$, which is accomplished by the operations
$\phis$\textto$\pi$\textminus$\phis$ and $\DGs$\textto\tm$\DGs$, gives the complex conjugate of the product $\timeAmp^*\timeAmp[j]$. This
turns out to give a discrete ambiguity in the parameter values, as will be explained in Section~\ref{subsec:pheno_equations_symmetry}.


%%%%%%%%%%%%%%%%%%%%%%%%%%%%%%%%%%%%%%%%%
\subsection{Alternative Parameterization}
\label{subsec:pheno_time_altParam}
%%%%%%%%%%%%%%%%%%%%%%%%%%%%%%%%%%%%%%%%%

The coefficients of the time-dependent functions in Equation~\ref{eq:timeqfDep} are functions of $\lamsi$ and hence contain both the
effects of CP violation in mixing and CP violation in decay. These two effects can be separated with an alternative parameterization, which
has been implemented in the software that was used for data analysis in the measurement presented in this thesis.

Starting again from Equation~\ref{eq:timeDepProd}, an alternative expression for the product
$\timeAmp[i,\qf]^*\timeAmp[j,\qf]^{\phantom{*}}$ is derived by splitting the $\lamsi$ parameter into a $\qpAlt$ factor and an amplitude
factor. Defining
\begin{equation}
  \label{eq:decayRho}
  \begin{gathered}
    \Ppm \equiv \frac{1}{2} \left[ 1 \pm \left(\frac{\Abari}{\Ai}\right)^{\!*}\, \frac{\Abari[j]}{\Ai[j]} \right] \\[0.6em]
    \Pre \equiv \frac{1}{2} \left[ \left(\frac{\Abari}{\Ai}\right)^{\!*} + \frac{\Abari[j]}{\Ai[j]} \right] \quad\quad
    \Pim \equiv \frac{i}{2} \left[ \left(\frac{\Abari}{\Ai}\right)^{\!*} - \frac{\Abari[j]}{\Ai[j]} \right]
  \end{gathered}
\end{equation}
and
\begin{equation}
  \label{eq:mixCDS}
  \Cm \equiv \frac{1-\qpAbsAlt^2}{1+\qpAbsAlt^2}
  \qquad \Dm \equiv -\frac{2\,\Re\big(\qpAlt\big)}{1+\qpAbsAlt^2}
  \qquad \Sm \equiv  \frac{2\,\Im\big(\qpAlt\big)}{1+\qpAbsAlt^2}
  \ ,
\end{equation}
the product $\Ai^*\Ai[j]\,\timeAmp[i,\qf]^*\timeAmp[j,\qf]^{\phantom{*}}$ can be expressed as
\begin{equation}
  \label{eq:timeqfDepSep}
  \begin{aligned}
    \Ai^*\Ai[j]\,\timeAmp[i,\qf]^*\timeAmp[j,\qf]^{\phantom{*}} =
      \Ai^*&\Ai[j]\, \frac{1-\qf\,\Cm}{1-\Cm^2}\, \eGst \\
        \times\Big[ &(\Ppl+\Pmn\,\Cm)\,\cDGs \\
                    &\quad + \qf\,(\Pmn+\Ppl\,\Cm)\,\cDms \\
                    &\quad\quad + (\Pre\,\Dm+\Pim\,\Sm)\,\sDGs \\
                    &\quad\quad\quad + \qf\,(\Pim\,\Dm-\Pre\,\Sm)\,\sDms \Big] \ .
  \end{aligned}
\end{equation}
Notice that the amplitudes $\Ai$ and $\Ai[j]$ occur in this expression and not the CP-average amplitudes $\AAv$ and $\AAv[{\Ai[j]}]$.

The fact that the phases of $\qp$ and $\frac{\Abari}{\Ai}$ cannot be observed independently is reflected by the observation that only the
combinations $\Pre\,\Dm$\textplus$\Pim\,\Sm$ and $\Pim\,\Dm$\textminus$\Pre\,\Sm$ are observable and not the $\Pre$, $\Pmn$, $\Dm$, and
$\Sm$ parameters individually. A phase convention can be chosen by, for example, setting $\arg(\qpAlt)$\textequiv0, which makes the phase
of $\frac{\Abari}{\Ai}$ equal to the phase of $\lamsi$. Another choice is $\arg(\frac{\Abari[\text{0}]}{\Ai[\text{0}]})$\textequiv0, which
results in the equations $\arg(\qpAlt)$\texteq$\arg(\lamsi[\text{0}])$ and
$\arg(\frac{\Abari}{\Ai})$\texteq$\arg(\lamsi)$\textminus$\arg(\lamsi[\text{0}])$ for the remaining parallel, perpendicular, and S-wave
amplitudes. This convention directly gives a measurement of phase differences.

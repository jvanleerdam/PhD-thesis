\section{Decay-Angle Distributions}
\label{sec:pheno_angles}

Expanding Equation~\ref{eq:sqAmp} into terms with the real and imaginary parts of the angular dependence gives
\begin{equation}
  \label{eq:sqAmpExpand}
  \begin{aligned}
    |\mathcal{A}(\Bstof)|^2
      &\propto \sum_i |c_i|^2\, |\angAmp|^2 \\
        &\qquad + \sum_{i\neq j} \Re(c_i^*\,c_j)\, \Re(\angAmp^*\,\angAmp[j])
                               - \Im(c_i^*\,c_j)\, \Im(\angAmp^*\,\angAmp[j]) \ ,
  \end{aligned}
\end{equation}
where the amplitude coefficients $c_i$ are defined as $c_i$\textequiv$\Ai\,\timeAmp\,\mKKAmp$. The angular dependence of this squared
amplitude is described by the functions $|\angAmp|^\text{2}$, 2\,$\Re(\angAmp^*\,\angAmp[j])$, and \tm2\,$\Im(\angAmp^*\,\angAmp[j])$. The
factors two in the functions of the interference terms originate from adding the terms with indices $ij$ and $ji$, which give identical
contributions.

Expressions for the products $\angAmp[i]^*\,\angAmp[j]$ are derived in Appendix~\ref{chap:angularDecay} using the \emph{helicity
formalism}~\cite{Jacob:1959at,Chung:1971ri,*Richman:1984gh,*Kutschke:1996}. In this description the angular dependence for an intermediate
state in the \BstoJpsiKK{} decay with definite particle helicities is given by the product of two Wigner D-matrices (see
Equation~\ref{eq:angAmpRed}).

Squaring the magnitude of the sum over D-matrices for the different helicity states yields angular functions for each of the states and for
their interferences (Equations~\ref{eq:angAmpSqEval}, \ref{eq:Eultohel}, \ref{eq:sqAmpHel}, and \ref{eq:angFuncs}). Because helicity states
are not CP eigenstates, the functions are combined into functions in the transversity basis by substituting their coefficients
(\emph{helicity amplitudes}) by combinations of the transversity amplitudes, $\Ai$ (Equation~\ref{eq:helToTransAmps}). Finally, this gives
the expressions for the functions $|\angAmp|^\text{2}$, 2\,$\Re(\angAmp^*\,\angAmp[j])$, and \tm2\,$\Im(\angAmp^*\,\angAmp[j])$.

The resulting angular functions are shown for each combination of intermediate states $i$ and $j$ in
Tables~\ref{tab:angDistJpsiphiPY}--\ref{tab:angDistSWaveSinCos}. A summary is given in Table~\ref{tab:angDist}. While in the appendix the
coefficients of the amplitudes are given by only the helicity/transversity amplitudes, the coefficients $c_i$ are used in
Table~\ref{tab:angDist} to also introduce the dependence on decay time ($\timeAmp$) and $\KK$ mass ($\mKKAmp$).

%The angular functions are combined with the decay time and $\KK$ mass functions by expanding the combinations of coefficients in
%Table~\ref{tab:angDist}. This gives expressions of the form
%\begin{equation}
%  \begin{aligned}
%    \AmpSq[c]{i}       &= |\Ai\,\timeAmp|^2\,|\mKKAmp|^2 \\
%    \ReAmp[c][c]{i}{j} &= \Re(\Ai^*\Ai[j]\,\timeAmp^*\timeAmp[j])\,\Re(\mKKAmp^*\mKKAmp[j])
%                        - \Im(\Ai^*\Ai[j]\,\timeAmp^*\timeAmp[j])\,\Im(\mKKAmp^*\mKKAmp[j]) \\
%    \ImAmp[c][c]{i}{j} &= \Re(\Ai^*\Ai[j]\,\timeAmp^*\timeAmp[j])\,\Im(\mKKAmp^*\mKKAmp[j])
%                        + \Im(\Ai^*\Ai[j]\,\timeAmp^*\timeAmp[j])\,\Re(\mKKAmp^*\mKKAmp[j]) \ .
%  \end{aligned}
%\end{equation}
%The expression for the product $\Ai^*\Ai[j]\,\timeAmp^*\timeAmp[j]$ is given in Equation~\ref{eq:timeqfDep}. Its real and imaginary parts
%can be expanded further in terms of the real and imaginary parts of $\AAvConj\AAv[{\Ai[j]}]$ and the coefficients $\Cf[ij]^\pm$,
%$\Df[ij]$, and $\Sf[ij]$. The dependence on the $\KK$ mass, $\mKKAmp^*\mKKAmp[j]$, is discussed in the next section.

\begin{table}[htb]
  \centering
  \caption{Angular functions for the \BstoJpsiKK{} decay in helicity angles.
           The coefficients, which depend on the decay time and the $\KK$ mass,
           are given by $c_i$\textequiv$\Ai\,\timeAmp\,\mKKAmp$.}
  \renewcommand{\arraystretch}{1.2}
  \label{tab:angDist}
  \begin{tabular}{cc}
    \hline
    coefficient                            &  $f(\Omega) \times \tfrac{32\pi}{9}$  \\
    \hline
    $\AmpSq[c]{0}$                         &  $2\, \cos^2\thetaK\, \sin^2\thetal$  \\
    $\AmpSq[c]{\parallel}$                 &  $\sin^2\thetaK\, (1 - \sin^2\thetal\, \cos^2\phihel)$  \\
    $\AmpSq[c]{\perp}$                     &  $\sin^2\thetaK\, (1 - \sin^2\thetal\, \sin^2\phihel)$  \\
    $\ReAmp[c][c]{0}{\parallel}$           &  $+\frac{1}{\sqrt{2}}\, \sin2\thetaK\, \sin2\thetal\, \cos\phihel$  \\
    $\ImAmp[c][c]{0}{\perp}$               &  $-\frac{1}{\sqrt{2}}\, \sin2\thetaK\, \sin2\thetal\, \sin\phihel$  \\
    $\ImAmp[c][c]{\parallel}{\perp}$       &  $+\sin^2\thetaK\, \sin^2\thetal\, \sin2\phihel$  \\
    $\AmpSq[c]{{\text{S}}}$                &  $\tfrac{2}{3}\, \sin^2\thetal$  \\
    $\ReAmp[c][c]{0}{{\text{S}}}$          &  $\tfrac{4}{3}\sqrt{3}\, \cos\thetaK\, \sin^2\thetal$  \\
    $\ReAmp[c][c]{\parallel}{{\text{S}}}$  &  $\tfrac{1}{3}\sqrt{6}\, \sin\thetaK\, \sin2\thetal\, \cos\phihel$  \\
    $\ImAmp[c][c]{\perp}{{\text{S}}}$      &  $\tfrac{1}{3}\sqrt{6}\, \sin\thetaK\, \sin2\thetal\, \sin\phihel$  \\
    \hline
  \end{tabular}
\end{table}

After adding the results for the two possible helicity configurations of the muons in the $\Jpsi$ decay, only ten terms in
Equation~\ref{eq:sqAmpExpand} give non-zero contributions. As a result, each combination of states $i$ and $j$ only appears once in
Table~\ref{tab:angDist}. For the interference terms either the real or the imaginary part of the product $c_i^*c_j$ appears.

As indicated in Equation~\ref{eq:diffRateAngles}, the functions $\cthetaK$ and $\cthetal$ are used as variables in the decay model rather
than the corresponding angles $\thetaK$ and $\thetal$. These angles are polar angles in a spherical coordinate system, which are defined
between 0 and $\pi$. The corresponding cosines, with ranges [\tm1,\,\tp1], completely specify the values of these two angles and are the
natural variables to use. The angle $\phihel$ is an azimuthal angle, defined between \tm$\pi$ and \tp$\pi$.

The angular functions in Table~\ref{tab:angDist} are normalized such that the integrals over all three angular variables are equal to one
for the diagonal terms ($\AmpSq[c]{0}$, $\AmpSq[c]{\parallel}$, $\AmpSq[c]{\perp}$, and $\AmpSq[c]{\text{S}}$). All interference terms
vanish when integrated over all three variables. Also the one-angle distributions, given in Table~\ref{tab:angDistOneAng}, are dominated
by the contributions from the diagonal terms. Interference terms that survive the integration over two angular variables are the
$\ReAmp[c][c]{0}{{\text{S}}}$ term in $\cthetaK$ and the $\ImAmp[c][c]{\parallel}{\perp}$ term in $\phihel$, but these contributions almost
vanish when integrated over decay time (see Table~\ref{tab:timeFunctionsNoCPV} in Section~\ref{subsec:pheno_equations_approx}).
\begin{table}[htb]
  \centering
  \caption{Angular functions for the \BstoJpsiKK{} decay integrated over two of the three angular variables.
           The coefficients, which depend on the decay time and the $\KK$ mass,
           are given by $c_i$\textequiv$\Ai\,\timeAmp\,\mKKAmp$.}
  \renewcommand{\arraystretch}{1.2}
  \label{tab:angDistOneAng}
  \begin{tabular}{cccc}
    \hline
    coefficient                       &  $f(\cthetaK)$
                                      &  $f(\cthetal)$                       &  $f(\phihel)$  \\
    \hline
    $\AmpSq[c]{0}$                    &  $\tfrac{3}{2}\, \cos^2\thetaK$
                                      &  $\tfrac{3}{4}\, (1-\cos^2\thetal)$  &  $\tfrac{1}{2\pi}$  \\
    $\AmpSq[c]{\parallel}$            &  $\tfrac{3}{4}\, (1-\cos^2\thetaK)$
                                      &  $\tfrac{3}{8}\, (1+\cos^2\thetal)$  &  $\tfrac{1}{4\pi}\, (2-\cos2\phihel)$  \\
    $\AmpSq[c]{\perp}$                &  $\tfrac{3}{4}\, (1-\cos^2\thetaK)$
                                      &  $\tfrac{3}{8}\, (1+\cos^2\thetal)$  &  $\tfrac{1}{4\pi}\, (2+\cos2\phihel)$  \\
    $\AmpSq[c]{{\text{S}}}$           &  $\tfrac{1}{2}$
                                      &  $\tfrac{3}{4}\, (1-\cos^2\thetal)$  &  $\tfrac{1}{2\pi}$  \\
    $\ImAmp[c][c]{\parallel}{\perp}$  &  \tm
                                      &  \tm                                 &  $\tfrac{1}{2\pi}\, \sin2\phihel$  \\
    $\ReAmp[c][c]{0}{{\text{S}}}$     &  $\sqrt{3}\, \cos\thetaK$
                                      &  \tm                                 &  \tm  \\
    \hline
  \end{tabular}
\end{table}

\section{Mixing and Decay of the \texorpdfstring{\BsBsbar{}}{Bs0Bs0bar} System}
\label{sec:pheno_mix}

%%%%%%%%%%%%%%%%%%%%%%%%%
\subsection{Mixing}
\label{sec:pheno_mix_mix}
%%%%%%%%%%%%%%%%%%%%%%%%%

A meson produced in a pure $\Bs$ or $\Bsbar$ state will evolve in time and become a mixture of these two flavour states. If the time
coordinate in the \BsBsbar{} centre-of-mass system is given by $t$, the state of the system ($\Psi$) can be written as
\begin{equation}
  \label{eq:timeEvolBBbarState}
  |\Psi(t)\rangle = a(t)\,\Bsst + b(t)\,\Bsbarst
  \ .
\end{equation}
The states $\Bsst\equiv|\text{s}\bar{\text{b}}\rangle$ and $\Bsbarst\equiv|\bar{\text{s}}\text{b}\rangle$ are the \emph{flavour
eigenstates} of the system and $a$ and $b$ are coefficients that describe its time dependence.

Assuming that the time scale of interest is much larger than the time scale of strong interactions, the Weisskopf-Wigner approximation can
be used and the decay of the system has an exponential time dependence~\cite{Weisskopf:1930au,*Weisskopf:1930ps,*Lee:1957qq}. The time
evolution then follows from a Schr\"odinger equation with a constant Hamiltonian, which is given by\footnote{All equations in this chapter
are expressed in \emph{natural units}, in which the reduced Planck constant and the speed of light are equal to one ($\hbar \equiv c \equiv
1$).}
\begin{equation}
  \label{eq:timeEvolSchr}
  i\; \frac{\partial}{\partial t} \begin{pmatrix} a(t) \\ b(t) \end{pmatrix}
    = \vec{H} \begin{pmatrix} a(t) \\ b(t) \end{pmatrix}
    \ .
\end{equation}

Without loss of generality, the Hamiltonian matrix $\vec{H}$ can be written as the sum of a Hermitian matrix and an anti-Hermitian matrix:
$\vec{H} \equiv \vec{M} - \tfrac{i}{2}\,\vec{\Gamma}$, where the \emph{mass matrix} $\vec{M}$ and the \emph{decay matrix} $\vec{\Gamma}$
are both Hermitian. Assuming CPT invariance, the mass and lifetime of a particle are equal to the respective mass and lifetime of the
corresponding anti-particle. This results in equal diagonal elements of both the mass and decay matrices:
\begin{equation}
  \label{eq:timeEvolHamil}
  \vec{H}
    \equiv \begin{pmatrix} H_0 & H_{12} \\ H_{21} & H_0 \end{pmatrix}
    = \vec{M} - \tfrac{i}{2}\,\vec{\Gamma}
    \equiv \begin{pmatrix} \ms & \mmix \\ \mmix^* & \ms \end{pmatrix}
      - \tfrac{i}{2} \begin{pmatrix} \Gs & \gmix \\ \gmix^* & \Gs \end{pmatrix}
    \ ,
\end{equation}
where $\ms$ is the $\Bs$ mass and $\Gs=\frac{1}{\taus}$ the $\Bs$ decay width, with $\taus$ the $\Bs$ mean lifetime.

Mixing of the $\Bs$ and $\Bsbar$ states is governed by the off-diagonal elements of the Hamiltonian. The parameter $\mmix$ is the
\emph{dispersive} part of $H_{12}$. This part originates from contributions of virtual intermediate states to the mixing process.
$\gmix$ is the \emph{absorptive} part, which originates from contributions of real states into which both $\Bs$ and $\Bsbar$ can decay.
% say something about the relative magnitudes M12, Gamma12 (and also relative to Ms and Gammas?)

To solve Equation~\ref{eq:timeEvolSchr} and obtain expressions for the time evolution of the $\Bs$ and $\Bsbar$ states, the system is
decoupled with a transformation of the flavour eigenstates that diagonalizes the Hamiltonian matrix. The decoupled states are \emph{mass
eigenstates}, which have definite mass and lifetime. With transformation matrix $\vec{P}$, diagonalized Hamiltonian $\vec{H'}$ and
mass-eigenstate coefficients $a'$ and $b'$, the transformation is specified by
\begin{equation}
  \label{eq:timeEvolTrans}
  \vec{H'} = \vec{P}^{-1}\,\vec{H}\,\vec{P}
  \qquad \text{and} \qquad
  \begin{pmatrix} a(t) \\ b(t) \end{pmatrix}
    = \vec{P}\, \begin{pmatrix} a'(t) \\ b'(t) \end{pmatrix}
  \ .
\end{equation}
The two eigenvalues of $\vec{H}$ become the diagonal elements of the decoupled Hamiltonian $\vec{H'}$ and the transformation matrix is
constructed from the corresponding eigenvectors:
\begin{subequations}
  \label{eq:timeEvolMassStates}
  \begin{align}
    \vec{H'} &\equiv \begin{pmatrix} \mL & 0 \\ 0 & \mH \end{pmatrix}
             - \tfrac{i}{2}\, \begin{pmatrix} \GL & 0 \\ 0 & \GH \end{pmatrix}
    \label{eq:timeEvolMassStatesHamil1} \\
    &= \begin{pmatrix} H_0 - \sqrt{H_{12} H_{21}} & 0 \\ 0 & H_0 + \sqrt{H_{12} H_{21}} \end{pmatrix}
    \label{eq:timeEvolMassStatesHamil2} \\
    \vec{P}  &= \begin{pmatrix*}[r]
                   \alpha\sqrt{H_{12}} &  \beta\sqrt{H_{12}} \\
                  -\alpha\sqrt{H_{21}} & +\beta\sqrt{H_{21}}
                \end{pmatrix*}
    \label{eq:timeEvolMassTrans}
    \ ,
  \end{align}
\end{subequations}
where the subscript L (light) is used for the state with the smaller mass and the subscript H (heavy) for the state with the larger mass.
The parameters $\alpha$ and $\beta$ are arbitrary constants as far as diagonalizing the Hamiltonian is concerned.

Mass and decay parameters of $\Bs$ and $\Bsbar$ are related to the masses and decay widths of $\BL$ and $\BH$ by
Equations~\ref{eq:timeEvolMassStatesHamil1} and \ref{eq:timeEvolMassStatesHamil2}: $H_0 = \tfrac{1}{2}\,(\mH+\mL) -
\tfrac{i}{4}\,(\GL+\GH)$ and $\sqrt{H_{12} H_{21}} = \tfrac{1}{2}\,(\mH-\mL) + \tfrac{i}{4}\,(\GL-\GH)$. Using also
Equation~\ref{eq:timeEvolHamil}:
\begin{subequations}
  \begin{alignat}{2}
    \ms   &\,=\,&     \Re(H_0) &\,=\, \tfrac{1}{2}\,(\mH+\mL)     \\
    \Gs &\,=\,& -2\,\Im(H_0) &\,=\, \tfrac{1}{2}\,(\GL+\GH)
  \end{alignat}%
  \begin{align}
    \Dms   &\equiv \mH-\mL = 2\,\Re\!\left(\!\sqrt{H_{12} H_{21}}\right) \nonumber\\
             &= 2\,\Re\!\left(\sqrt{(\mmix-\tfrac{i}{2}\,\gmix)\,(\mmix^*-\tfrac{i}{2}\,\gmix^*)}\right) \\
    \DGs &\equiv \GL-\GH = 4\,\Im\!\left(\!\sqrt{H_{12} H_{21}}\right) \nonumber\\
             &= 4\,\Im\!\left(\sqrt{(\mmix-\tfrac{i}{2}\,\gmix)\,(\mmix^*-\tfrac{i}{2}\,\gmix^*)}\right)
  \end{align}
\end{subequations}

To find the transformation between flavour eigenstates and mass eigenstates, the state of Equation~\ref{eq:timeEvolBBbarState} is expressed
in matrix form and the transformation of Equation~\ref{eq:timeEvolTrans} is applied:
\begin{equation}
  \label{eq:timeEvolBLBHState}
  \begin{split}
    |\Psi(t)\rangle &= \begin{pmatrix} a'(t) & b'(t) \end{pmatrix} \begin{pmatrix} \BLst \\ \BHst \end{pmatrix} \\
                    &= \begin{pmatrix} a(t) & b(t) \end{pmatrix} \begin{pmatrix} \Bsst \\ \Bsbarst \end{pmatrix}
                     = \begin{pmatrix} a'(t) & b'(t) \end{pmatrix} \vec{P}^\text{T} \begin{pmatrix} \Bsst \\ \Bsbarst \end{pmatrix}
    \ .
  \end{split}
\end{equation}
By comparing the first and second line of Equation~\ref{eq:timeEvolBLBHState} it can be seen that the transformation matrix between flavour
eigenstates and mass eigenstates is the transpose of the matrix that diagonalizes the Hamiltonian (Equation~\ref{eq:timeEvolMassTrans}).

Introducing the complex parameters $p$ and $q$ and normalizing the mass eigenstates, the transformation to flavour eigenstates can be
expressed as
% fix a phase convention?
\begin{equation}
  \label{eq:timeEvolTransInv}
  \begin{pmatrix} \BLst \\ \BHst \end{pmatrix}
    = \vec{P}^\text{T} \begin{pmatrix} \Bsst \\ \Bsbarst \end{pmatrix}
    = \begin{pmatrix} p & +q \\ p & -q \end{pmatrix}
      \begin{pmatrix} \Bsst \\ \Bsbarst \end{pmatrix}
  \qquad
  \text{with } |p|^2+|q|^2\equiv1
  \ .
\end{equation}
From Equations~\ref{eq:timeEvolMassTrans} and \ref{eq:timeEvolTransInv} it now follows that $\alpha=\beta$ and, also with
Equation~\ref{eq:timeEvolHamil},
\begin{equation}
  \qp = -\sqrt{\frac{H_{21}}{H_{12}}} = -\sqrt{\frac{\mmix^*-\tfrac{i}{2}\,\gmix^*}{\mmix-\tfrac{i}{2}\,\gmix}}
  \quad .
\end{equation}

The time evolution of the mass eigenstates is obtained by solving the Schr\"o\-ding\-er equation with the diagonal Hamiltonian of
Equation~\ref{eq:timeEvolMassStates}, which gives decoupled exponential decays for $\BL$ and $\BH$. Transforming back to the flavour
basis then gives the coupled time evolution of $\Bs$ and $\Bsbar$:
\begin{subequations}
  \label{eq:timeEvolTimeCoefs}
  \begin{equation}
    \begin{split}
      \begin{pmatrix} a(t) \\ b(t) \end{pmatrix}
        &= e^{-i\,\vec{H}\,t} \begin{pmatrix} a(0) \\ b(0) \end{pmatrix} \\
        &= \vec{P} \begin{pmatrix} a'(t) \\ b'(t) \end{pmatrix}
         = \vec{P}\, e^{-i\,\vec{H'}\,t} \begin{pmatrix} a'(0) \\ b'(0) \end{pmatrix}
         = \vec{P}\, e^{-i\,\vec{H'}\,t}\, \vec{P}^{-1} \begin{pmatrix} a(0) \\ b(0) \end{pmatrix}
    \end{split}
  \end{equation}%
  \begin{align}
    e^{-i\,\vec{H}\,t} &= \vec{P}\, e^{-i\,\vec{H'}\,t}\, \vec{P}^{-1} \nonumber\\
                       &= \begin{pmatrix*}[r] p & p \\ q & -q \end{pmatrix*}\!
                          \begin{pmatrix} e^{-i\,(\mL-i\,\GL/2)\,t} & 0 \\ 0 & e^{-i\,(\mH-i\,\GH/2)\,t} \end{pmatrix}\!
                          \begin{pmatrix*}[r] q & p \\ q & -p \end{pmatrix*} \frac{1}{2\,p\,q} \nonumber\\
                       &= \begin{pmatrix*}[r] g_+(t) & \pq\,g_-(t) \\ \qp\,g_-(t) & g_+(t) \end{pmatrix*}
    \ ,
  \end{align}
\end{subequations}
where the functions $g_\pm$ are given by
\begin{align}
  \label{eq:timeEvolQp}
  g_\pm(t) &\equiv \tfrac{1}{2}\, \left( e^{-i\,(\mL-i\,\GL/2)\,t} \pm e^{-i\,(\mH-i\,\GH/2)\,t} \right) \\
           &= \tfrac{1}{2}\, e^{-i\,\ms\,t}\, e^{-\Gs/2\,t}
                \left( e^{+i\,\Dms/2\,t}\, e^{-\DGs/4\,t} \pm e^{-i\,\Dms/2\,t}\, e^{+\DGs/4\,t} \right) \nonumber
           \ .
\end{align}

The time evolution for mesons produced as $\Bs$ ($a(0)=1$ and $b(0)=0$) and mesons produced as $\Bsbar$ ($a(0)=0$ and $b(0)=1$) can now be
inferred from Equations~\ref{eq:timeEvolBBbarState} and \ref{eq:timeEvolTimeCoefs}:
\begin{subequations}
  \label{eq:timeEvolBBbarStateInit}
  \begin{align}
    |\Psi_{\Bs}(t)\rangle    = g_+(t)\,\Bsst    + \qp\,g_-(t)\,\Bsbarst \\
    |\Psi_{\Bsbar}(t)\rangle = g_+(t)\,\Bsbarst + \pq\,g_-(t)\,\Bsst
  \end{align}
\end{subequations}

%%%%%%%%%%%%%%%%%%%%%%%%%%%%%
\subsection{Mixing and Decay}
\label{sec:pheno_mix_decay}
%%%%%%%%%%%%%%%%%%%%%%%%%%%%%

Time-dependent amplitudes for mixing and decay of the \BsBsbar{} system are obtained by combining the state of
Equation~\ref{eq:timeEvolBBbarState} with the amplitudes for the decays of $\Bsst$ and $\Bsbarst$. Assuming the system is produced as
either $\Bsst$ or $\Bsbarst$, the required time-dependent states are given by Equation~\ref{eq:timeEvolBBbarStateInit}. The decay amplitude
of a decay of $\Bsst$ into a final state $\fst$ is labelled by $\Af$. The amplitude for a system produced as a $\Bs$, being in a $\Bs$
state at the time of decay and decaying into $\fst$ is given by $\langle\Bs|\Psi_{\Bs}(t)\rangle\,\Af \propto g_+(t)\,\Af$.

If also $\Bsbarst$ can decay into the final state there is a second contribution to the $\Bstof$ process, which is proportional to the
decay amplitude labelled by $\Abarf$. Also considering decays into the CP conjugate of the final state, $\fbarst$, the amplitudes of the
four possible combinations of initial and final states are given by
\begin{equation}
  \label{eq:mixDecayAmps}
  \begin{alignedat}{4}
    \mathcal{A}(\Bstof) &\propto& & g_+\, \Af + \qp\, g_-\, \Abarf &
    \quad
    \mathcal{A}(\Bstofbar) &\propto& \;\qp \Big( & g_-\, \Abarfbar + \pq\, g_+\, \Afbar \Big)
    \\
    \mathcal{A}(\Bsbartof) &\propto& \;\pq \Big( & g_-\, \Af + \qp\, g_+\, \Abarf \Big) &
    \mathcal{A}(\Bsbartofbar) &\propto& & g_+\, \Abarfbar + \pq\, g_-\, \Afbar
    \ .
  \end{alignedat}
\end{equation}

Note that the amplitudes for the processes with final state $\fbarst$ have the same structure as the amplitudes for the processes with
final state $\fst$. $\mathcal{A}(\Bsbartofbar)$ and $\mathcal{A}(\Bstofbar)$ can be obtained from $\mathcal{A}(\Bstof)$ and
$\mathcal{A}(\Bsbartof)$, respectively, by interchanging $p$ and $q$, replacing $\Af$ by $\Abarfbar$ and replacing $\Abarf$ by $\Afbar$.
The amplitude for $\Bsbartof$\ \ ($\Bstofbar$) is obtained from the amplitude for $\Bstof$\ \ ($\Bsbartofbar$) by interchanging $g_+$ and $g_-$
and multiplying by a factor $\pq$ $\big(\qp\big)$.

The magnitudes of the amplitudes in Equation~\ref{eq:mixDecayAmps} are squared to get an expression for the differential decay rates
in time. For the $\Bstof$ amplitude this yields
\begin{align}
    &\frac{\ud\Gamma(\Bstof)}{\ud t} \propto \left| \mathcal{A}(\Bstof) \right|^2 \nonumber\\
    &\qquad\propto |g_+|^2\, |\Af|^2 + \Big|\qp\Big|^2\,|g_-|^2\, |\Abarf|^2
      + \qp\, g_+^*\,g_-\, \Af^*\,\Abarf + \left( \qp\, g_+^*\,g_-\, \Af^*\,\Abarf \right)^* \nonumber\\
    &\qquad\propto |g_+|^2\, |\Af|^2 + \Big|\qp\Big|^2\,|g_-|^2\, |\Abarf|^2 \nonumber\\
      &\qquad\qquad + 2\,\Re(g_+^*\,g_-)\, \Re\!\left(\qp\,\Af^*\,\Abarf\right)
                    - 2\,\Im(g_+^*\,g_-)\, \Im\!\left(\qp\,\Af^*\,\Abarf\right)
    \ .
\end{align}
Using the definition of $g_\pm$ from Equation~\ref{eq:timeEvolQp}, the required products of $g_+$ and $g_-$ are given by
\begin{subequations}
  \label{eq:mixDecayQpSq}
  \begin{alignat}{4}
    |g_\pm|^2  &\,=\,& \tfrac{1}{2}\, e^{-\Gs\,t} \Big[ & & \cDGs\, & & \pm\,  & \cDms \Big] \\
    g_+^*\,g_- &\,=\,& \tfrac{1}{2}\, e^{-\Gs\,t} \Big[ &-& \sDGs\, & & +\,i\, & \sDms \Big]
    \ ,
  \end{alignat}
\end{subequations}
which yields
\begin{equation}
  \label{eq:mixDecayDiffRate}
  \begin{aligned}
    \frac{\ud\Gamma(\Bstof)}{\ud t} \propto \tfrac{1}{2}\, e^{-\Gs\,t}
      \bigg[ &   \left(|\Af|^2 + \Big|\qp\Big|^2\, |\Abarf|^2\right)\,\cDGs \\
             & + \left(|\Af|^2 - \Big|\qp\Big|^2\, |\Abarf|^2\right)\,\cDms \\
             & - 2\,\Re\!\left(\qp\,\Af^*\,\Abarf\right)\,\sDGs \\
             & - 2\,\Im\!\left(\qp\,\Af^*\,\Abarf\right)\,\sDms
    \bigg]
    \ .
  \end{aligned}
\end{equation}
With the definitions
\begin{subequations}
\begin{equation}
  \label{eq:mixDecayLamfDef}
  \lamf\equiv\qp\frac{\Abarf}{\Af}
\end{equation}
\begin{equation}
  \label{eq:mixDecayCDS}
  \Cf \equiv \frac{1-\lamfSq}{1+\lamfSq}
  \qquad \Df \equiv -\frac{2\,\Re(\lamf)}{1+\lamfSq}
  \qquad \Sf \equiv  \frac{2\,\Im(\lamf)}{1+\lamfSq}
  \ ,
\end{equation}
\end{subequations}
the differential decay rate can be expressed as
\begin{equation}
  \label{eq:mixDecayDiffRateLambda}
  \begin{aligned}
    \frac{\ud\Gamma(\Bstof)}{\ud t} \propto \tfrac{1}{2}\, |\Af|^2\, &(1+\lamfSq)\, e^{-\Gs\,t} \\
      \times \Big[ &\cDGs + \Cf\,\cDms \\
      &+ \Df\,\sDGs - \Sf\,\sDms \Big]
    \ ,
  \end{aligned}
\end{equation}

Expressions for the differential rates of the remaining three processes can be obtained by applying the aforementioned relations between
the four amplitudes. From Equation~\ref{eq:mixDecayQpSq} it can be seen that interchanging $g_+$ and $g_-$ results in a sign change of the
$\cDms$ and $\sDms$ terms. For the $\fbarst$ final state the parameter $\lamf$ goes to $\lamfbar$:
\begin{subequations}
\begin{equation}
  \label{eq:mixDecayLamfbarDef}
  \lamfbar\equiv\pq\frac{\Afbar}{\Abarfbar}
\end{equation}
\begin{equation}
  \label{eq:mixDecayCDSbar}
  \Cfbar \equiv \frac{1-\lamfbarSq}{1+\lamfbarSq}
  \qquad \Dfbar \equiv -\frac{2\,\Re(\lamfbar)}{1+\lamfbarSq}
  \qquad \Sfbar \equiv  \frac{2\,\Im(\lamfbar)}{1+\lamfbarSq}
  \ .
\end{equation}
\end{subequations}

With Equations~\ref{eq:mixDecayAmps} and \ref{eq:mixDecayDiffRateLambda}, the differential rates are given by
\begin{subequations}
  \label{eq:mixDecayDiffRateAll}
  \begin{align}
    \label{eq:mixDecayDiffRateB}
    \frac{\ud\Gamma(\text{f})}{\ud t}
      &\propto \tfrac{1}{2}\, |\Af|^2\, (1+\lamfSq)\, \frac{1+\Cm}{1-\Cm^2}\, (1-\qf\,\Cm)\, e^{-\Gs\,t} \nonumber\\
      &\qquad\quad\times \Big[ \cDGs + \qf\,\Cf\,\cDms \nonumber\\
      &\qquad\qquad\quad + \Df\,\sDGs - \qf\,\Sf\,\sDms \Big]
  \end{align}
  \begin{align}
    \label{eq:mixDecayDiffRateBbar}
    \frac{\ud\Gamma\!\left(\overline{\text{f}}\right)}{\ud t}
      &\propto \tfrac{1}{2}\, |\Abarfbar|^2\, (1+\lamfbarSq)\, \frac{1-\Cm}{1-\Cm^2}\, (1-\qf\,\Cm)\, e^{-\Gs\,t} \nonumber\\
      &\qquad\quad\times \Big[ \cDGs - \qf\,\Cfbar\,\cDms \nonumber\\
      &\qquad\qquad\quad + \Dfbar\,\sDGs + \qf\,\Sfbar\,\sDms \Big]
    \ ,
  \end{align}
\end{subequations}
where the variable $\qf$ takes the value $+1$ for a $\Bs$ initial state and $-1$ for a $\Bsbar$ initial state and the parameter for CP
violation in mixing $\Cm$ is given by
\begin{equation}
  \Cm \equiv \frac{1-\qpAbsAlt^2}{1+\qpAbsAlt^2}
  \ .
\end{equation}

For flavour-specific final states, where $\Bsst$ can only decay into $\fst$ and $\Bsbarst$ only into $\fbarst$, the parameters $\lamf$ and
$\lamfbar$ vanish. In that case $\Df$, $\Sf$, $\Dfbar$ and $\Sfbar$ are equal to zero and $\Cf$ and $\Cfbar$ are equal to one. Only the
$\cDGs$ and $\cDms$ terms then remain in the expressions for the differential decay rates. An example of this case are the CP-conjugate
processes $\Bs \to \text{D}_\text{s}^-\, \text{π}^+$ and $\Bsbar \to \text{D}_\text{s}^+\, \text{π}^-$.

Another special case is a CP self-conjugate final state, for which $\fst$ and $\fbarst$ are the same. Equations~\ref{eq:mixDecayDiffRateB}
and \ref{eq:mixDecayDiffRateBbar} then contain the same information and either one of them can be used to describe the decay. The final
state $\Jpsi\,\KK$ is an example of this case. The structure of the amplitude and the differential decay rate for the \BstoJpsiKK{} decay
is discussed in Section~\ref{sec:pheno_decay}.

%%%%%%%%%%%%%%%%%%%%%%%%%%%%%%%%%%%%%
\subsection{CP-Violation Observables}
\label{sec:pheno_obs}
%%%%%%%%%%%%%%%%%%%%%%%%%%%%%%%%%%%%%

% lambda, expression for phi_s, q/p --> arg(M_12), arg(-M_12/G_12), Deltam, DeltaGamma, a_fs, magnitude of G_12 / M_12 (< 0.01), ...

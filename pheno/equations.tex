\section{Decay-Rate Equations}
\label{sec:pheno_equations}

Combining the expressions for the decay-time and $\KK$-mass dependence from Equations~\ref{eq:timeDep} and \ref{eq:couplingFacAmps} yields
the coefficients in the sum of the angular dependence in Equation~\ref{eq:sqAmpExpand}:
\begin{equation}
  \begin{aligned}
    \label{eq:angCoefs}
    &\int_{\mKK^-}^{\mKK^+}\ud\mKK\,|\pJpsi|\,|\pK|\, c_i^*c_j
      = \int_{\mKK^-}^{\mKK^+}\ud\mKK\,|\pJpsi|\,|\pK|\, \Ai^*\Ai[j]\, \timeAmp^*\,\timeAmp[j]\, \mKKAmp^*\,\mKKAmp[j] \\
      &\quad\ \ = \Kij\, e^{-i\kij}\, \AAvConj\AAv[{\Ai[j]}]\, \eGst
              \Big[ \Cf[ij]^+\,\cDGs + \Cf[ij]^-\,\cDms \\
              &\qquad\qquad\qquad\qquad\qquad\qquad\quad\ \
               + \Df[ij]\,\sDGs - \Sf[ij]\,\sDms \Big] \ .
  \end{aligned}
\end{equation}
The phase of the integral over the $\KK$-mass functions can be absorbed in the phase difference between the amplitudes, as in
Equation~\ref{eq:couplingFacAmps}:
\begin{equation}
  \begin{aligned}
    e^{-i\kij}\, \AAvConj\AAv[{\Ai[j]}]
      &= |\AAv|\,|\AAv[{\Ai[j]}]|\, e^{i(\delta_j-\delta_i)} \\
      &= |\AAv|\,|\AAv[{\Ai[j]}]|\, [ \cos(\delta_j-\delta_i) + i\,\sin(\delta_j-\delta_i) ]\ ,
  \end{aligned}
\end{equation}
where
\begin{equation}
  \delta_j - \delta_i \equiv \arg(\AAv[{\Ai[j]}]) - \arg(\AAv) - \kij\ .
\end{equation}

Depending on whether the real or imaginary part of the coefficient is required in Equation~\ref{eq:sqAmpExpand}, the real and imaginary
parts of the time-dependence of Equation~\ref{eq:angCoefs} are multiplied by either $\cos(\delta_j-\delta_i)$ or $\sin(\delta_j-\delta_i)$.
Taking the $\cDGs$ term as an example, the coefficients of the angular distribution are proportional to
\begin{subequations}
  \label{eq:angCoefsReIm}
  \begin{align}
    \label{eq:angCoefsRe}
    \Re\!\big(e^{i(\delta_j-\delta_i)}\, \Cf[ij]^+ \big)
      &= \cos(\delta_j-\delta_i)\, \Re(\Cf[ij]^+) - \sin(\delta_j-\delta_i)\, \Im(\Cf[ij]^+) \\
    \label{eq:angCoefsIm}
    \Im\!\big(e^{i(\delta_j-\delta_i)}\, \Cf[ij]^+ \big)
      &= \sin(\delta_j-\delta_i)\, \Re(\Cf[ij]^+) + \cos(\delta_j-\delta_i)\, \Im(\Cf[ij]^+)\ ,
  \end{align}
\end{subequations}
where the first equation applies to the diagonal terms and the ``0$\parallel$'', ``0S'', and ``$\parallel$S'' terms and the second equation
to the ``0$\perp$'', ``$\parallel\perp$'', and ``$\perp$S'' terms.

The full expression for the differential decay rate in decay time and decay angles is obtained by summing the products of the angular
functions (Table~\ref{tab:angDist}) and their time-dependent coefficients (real or imaginary part of Equation~\ref{eq:angCoefs}). This
expression is used to build the probability density function that models the \BstoJpsiKK{} decay, as will be discussed in
Chapter~\ref{chap:ana}.


%%%%%%%%%%%%%%%%%%%%%%%%%%%%%%%%%%%%%%%%%%%%%%%%%
\subsection{Approximate Equations}
\label{subsec:pheno_equations_approx}
%%%%%%%%%%%%%%%%%%%%%%%%%%%%%%%%%%%%%%%%%%%%%%%%%

To indicate where the sensitivity to the different parameters in the decay model comes from, two approximations of the terms in the
differential rate are given in Tables~\ref{tab:timeFunctionsNoCPV}, \ref{tab:timeFunctionsApprox}, and \ref{tab:timeFunctionsApproxEqual}.
The first table shows the time-dependent functions in the differential decay rate in the case of no CP violation. Since CP violation in the
\BstoJpsiKK{} decay is small, this gives a good indication of how the remaining parameters appear in the decay-rate equations. Without CP
violation, the parameters $\Cs$ and $\Ss$ are equal to zero and $\Ds$ is equal to minus one. With Equation~\ref{eq:CDSPolarToCommon}, this
gives
\begin{subequations}
  \label{eq:CDSPolarToNoCPV}
  \begin{alignat}{5}
    \label{eq:CDSPolarToNoCPVEqual}
    \Cf[ij]^+ &\to  1      \quad\ \ &
    \Cf[ij]^- &\to  0      \quad\ \ &
    \Df[ij]   &\to -\eta_i \quad\ \ &
    \Sf[ij]   &\to  0      \quad&
    (\eta_i&=+\eta_j)      \\
    \label{eq:CDSPolarToNoCPVNotEqual}
    \Cf[ij]^+ &\to 0         \quad\ \ &
    \Cf[ij]^- &\to 1         \quad\ \ &
    \Df[ij]   &\to 0         \quad\ \ &
    \Sf[ij]   &\to i\,\eta_i \quad&
    (\eta_i&=-\eta_j)        \ .
  \end{alignat}
\end{subequations}

\begin{table}[htb]
  \centering
  \caption{Functions of decay time without CP violation.}
  \label{tab:timeFunctionsNoCPV}
  \renewcommand{\arraystretch}{1.3}
  \begin{tabular}{cl}
    \hline
    $ij$  &  $f(t) \times e^{+\Gs\,t}$  \\
    \hline
    $ii$
      &  $|\AAv|^2 \left[ \cDGs - \eta_i\,\sDGs \right]$  \\
    0$\parallel$
      &  $\magzeroAv\magparAv\cos(\delparzero) \left[ \cDGs - \sDGs \right]$  \\
    0$\perp$
      &  $\magzeroAv\magperpAv \left[ \sin(\delperpzero)\,\cDms - \cos(\delperpzero)\,\sDms \right]$  \\
    $\parallel\perp$
      &  $\magparAv\magperpAv \left[ \sin(\delperppar)\,\cDms - \cos(\delperppar)\,\sDms \right]$  \\
    0S
      &  $\CSP\magzeroAv\magSAv[i] \left[ \cos(\delSzero)\,\cDms + \sin(\delSzero)\,\sDms \right]$  \\
    $\parallel$S
      &  $\CSP\magparAv\magSAv \left[ \cos(\delSpar)\,\cDms + \sin(\delSpar)\,\sDms \right]$  \\
    $\perp$S
      &  $\CSP\magperpAv\magSAv\sin(\delSperp) \left[ \cDGs + \sDGs \right]$  \\
    \hline
  \end{tabular}
\end{table}

Equation~\ref{eq:CDSPolarToNoCPV} and Table~\ref{tab:timeFunctionsNoCPV} show that for terms with $\eta_i$\texteq$\eta_j$ only the $\cDGs$
and $\sDGs$ functions remain and for $\eta_i$\texteq\tm$\eta_j$ terms only the $\cDms$ and $\sDms$ functions. In the experiment,
flavour tagging is required to determine the coefficients of the latter, since these terms have opposite sign for $\Bs$ and $\Bsbar$ decays
(see Equation~\ref{eq:timeqfDep}) and cancel in the sum of the two differential decay rates. Because it is not possible to tag all decay
candidates (correctly), the statistical uncertainties of the associated parameters will generally be larger than the uncertainties of
parameters in the $\cDGs$ and $\sDGs$ terms.

Without CP violation, light and heavy eigenstates of the \BsBsbar{} system coincide with the CP-even and CP-odd states, respectively.
The time dependence of the diagonal terms, which form the decay-time distribution integrated over the decay angles, can consequently be
expressed as
\begin{equation}
  \begin{aligned}
    \label{eq:timeCPEvenOdd}
      |\AAv|^2\; \eGst&\left[ \cDGs - \eta_i\,\sDGs \right] \\
      &= \left\{\begin{array}{ll}
                  |\AAv|^2\; e^{-\GL\,t}  &  \text{if } \eta_i = +1 \quad\text{(CP even)}  \\
                  |\AAv|^2\; e^{-\GH\,t}  &  \text{if } \eta_i = -1 \quad\text{(CP odd)}
         \end{array}\right.
      \ .
  \end{aligned}
\end{equation}

The values of the parameters $\Gs$, $\DGs$, and $|\AAv|$ can be estimated relatively precisely from untagged decay candidates, but these
estimates are correlated. The mean of the decay-time distribution is controlled by $\Gs$, but also by the relative contributions of CP-even
and CP-odd states ($|\AAv|$), which have different lifetimes. The impact of changing the even and odd contributions depends on the value of
$\DGs$, which controls the difference in lifetime between the two states.

The phases of the transversity amplitudes can only be determined from interference terms. The cosine of $\delparzero$ and the sine of
$\delSperp$ appear in the untagged distribution, but tagged decay candidates are required to measure the sine of $\delparzero$ and the
cosine of $\delSperp$ from combinations of the remaining interference terms. As a result, the measurements of $\sin(\delparzero)$ and
$\cos(\delSperp)$ are less precise than the measurements of $\cos(\delparzero)$ and $\sin(\delSperp)$ and an approximate symmetry in the
estimates of these parameters arises for $\delparzero$\textto2$\pi$\textminus$(\delparzero)$ and
$\delSperp$\textto$\pi$\textminus$(\delSperp)$, for which the values of $\cos(\delparzero)$ and $\sin(\delSperp)$ do not change. The
parameter $\delperpzero$ is only determined with $\cDms$ and $\sDms$ terms in the approximation of Table~\ref{tab:timeFunctionsNoCPV} and
does not show a similar symmetry.

The interference term of the parallel and perpendicular states depends on both the $\delparzero$ and the $\delperpzero$ parameters:
\begin{subequations}
  \label{eq:parperpPhaseExp}
  \begin{align}
    \sin(\delperppar) &= \cos(\delparzero)\, \sin(\delperpzero) - \sin(\delparzero)\, \cos(\delperpzero) \nonumber\\
                      &\approx \sin(\delparzero) - \sin(\delperpzero)  \\
    \cos(\delperppar) &= \cos(\delparzero)\, \cos(\delperpzero) + \sin(\delparzero)\, \sin(\delperpzero) \nonumber\\
                      &\approx 1 \ .
  \end{align}
\end{subequations}
In these equations an approximation was used where the values of $\delparzero$ and $\delperpzero$ are approximately equal to $\pi$, which
is justified given previous measurements of these phase differences (see for example \cite{LHCb-PAPER-2013-002,*LHCb-ANA-2012-067}). The
combination of $\sin(\delparzero)$ and $\sin(\delperpzero)$ in the $\sDms$ coefficient of this term introduces a correlation between the
parameters $\delparzero$ and $\delperpzero$.

Since the fraction of S-wave is small, the sensitivity for the parameter $\Dms$ is expected to mainly come from the ``0$\perp$'' and
``$\parallel\perp$'' interference terms. The $\cDms$ and $\sDms$ functions in these terms can be combined into a single sine function with
a phase:
\begin{equation}
  \sin\delta\, \cDms - \cos\delta\, \sDms = -\sin\!\left(\Dms\,t - \delta \right) \ .
\end{equation}
Because a change in the frequency of a sine function in a limited range can be partially compensated by a phase shift, these terms
introduce correlations between $\Dms$ and the phases of the transversity amplitudes. Since practically all information on $\delperpzero$
comes from the ``0$\perp$'' and ``$\parallel\perp$'' terms, this parameter is affected most.

Table~\ref{tab:timeFunctionsApprox} shows the time dependence of the differential decay rate including CP violation, but with several
approximations for small parameters. Given previous measurements, it is known that the CP violation in \BstoJpsiKK{} is small, which leads
to $\lamsiAbs$\textapprox1 and $\phisi$\textapprox0. To parameterize also CP violation in decay and CP violation in mixing with a small
parameter, $\lamsiAbs$ is replaced with
\begin{equation}
  \label{eq:CsDef}
  \Csi \equiv \Cf[ii]^- = \frac{1-\lamsiSq}{1+\lamsiSq} \ .
\end{equation}

An expansion in $\Csi$ and $\phisi$ is used for the approximation in Table~\ref{tab:timeFunctionsApprox}. Expanding the coefficients of the
decay-rate equations (Equation~\ref{eq:CDSPolarDep}) in terms of these parameters at first order yields
\begin{subequations}
\begin{equation}
  \begin{aligned}
    \Cf[ij]^+ &\approx 1 + i \cdot \tfrac{1}{2} (\phisi - \phisi[j]) \\
    \Cf[ij]^- &\approx \tfrac{1}{2} (\Csi + \Csi[j]) - i \cdot \tfrac{1}{2} (\phisi - \phisi[j]) \\
    \Df[ij]   &\approx -\eta_i \left[ 1 + i \cdot \tfrac{1}{2} (\phisi - \phisi[j]) \right] \\
    \Sf[ij]   &\approx -\eta_i \left[ i \cdot \tfrac{1}{2} (\Csi - \Csi[j]) + \tfrac{1}{2} (\phisi + \phisi[j]) \right]
  \end{aligned}
\end{equation}
for $\eta_i$\texteq$\eta_j$ and
\begin{equation}
  \begin{aligned}
    \Cf[ij]^+ &\approx \tfrac{1}{2} (\Csi + \Csi[j]) - i \cdot \tfrac{1}{2} (\phisi - \phisi[j]) \\
    \Cf[ij]^- &\approx 1 + i \cdot \tfrac{1}{2} (\phisi - \phisi[j]) \\
    \Df[ij]   &\approx +\eta_i \left[ \tfrac{1}{2} (\Csi - \Csi[j]) - i \cdot \tfrac{1}{2} (\phisi + \phisi[j]) \right] \\
    \Sf[ij]   &\approx +\eta_i \left[ i - \tfrac{1}{2} (\phisi - \phisi[j]) \right]
  \end{aligned}
\end{equation}
for $\eta_i$\texteq\tm$\eta_j$.
\end{subequations}

Also the value of $\DGs$ is small enough to allow an approximation of the $\cDGs$ and $\sDGs$ functions. A first-order expansion in $\DGs$
gives
\begin{equation}
  \cDGs \approx 1 \qquad \sDGs \approx \tfrac{1}{2}\DGs\,t \ .
\end{equation}

The sines and cosines of the phase differences $\delparzero$ and $\delperpzero$ are expanded around $\pi$, as in
Equation~\ref{eq:parperpPhaseExp}:
\begin{equation}
  \begin{alignedat}{4}
    &\sin(\delparzero)  &&\approx \pi - (\delparzero) \qquad\quad &&\cos(\delparzero)  &&\approx -1  \\
    &\sin(\delperpzero) &&\approx \pi - (\delperpzero)            &&\cos(\delperpzero) &&\approx -1  \\
    &\sin(\delperppar)  &&\approx \delperppar                     &&\cos(\delperppar)  &&\approx +1  \ .
  \end{alignedat}
\end{equation}
In the table, the remaining non-trivial sine and cosine functions are denoted by
\begin{equation}
  \sij{i}{j} \equiv \sin(\delta_j - \delta_i) \qquad\quad
  \cij{i}{j} \equiv \cos(\delta_j - \delta_i) \ .
\end{equation}

\begin{table}[p]
  \centering
  \caption{Functions of decay time with approximations for small CP violation, small $\DGs$, and phase differences close to 0 or $\pi$.}
  \label{tab:timeFunctionsApprox}
  \renewcommand{\arraystretch}{1.7}
  \begin{tabular}{cl}
    \hline
    $ij$  &  $f(t) \times e^{+\Gs\,t}$  \\
    \hline
    $ii$
      &  $|\AAv|^2 \Big[
                     1 - \tfrac{1}{2}\,\eta_i\,\DGs\,t
                     + \Csi\,\cDms + \eta_i\,\phisi\,\sDms
                   \Big]$  \\
    0$\parallel$
      &  $-\magzeroAv\magparAv \Big[
                             1 - \tfrac{1}{2}\,\DGs\,t$  \\
      &  $\qquad\qquad\quad  + \tfrac{1}{2}(\Cszero + \Cspar) \cDms + \tfrac{1}{2}(\phiszero + \phispar) \sDms
                           \Big]$  \\
    0$\perp$
      &  $-\magzeroAv\magperpAv \Big[
                              \tfrac{1}{2}(\phisperp-\phiszero) - \tfrac{1}{4}(\phiszero+\phisperp)\,\DGs\,t$  \\
      &  $\qquad\qquad\quad   - (\szeroperp + \tfrac{1}{2}\phisperp - \tfrac{1}{2}\phiszero) \cDms - \sDms
                            \Big]$  \\
    $\parallel\perp$
      &  $+\magparAv\magperpAv \Big[
                             \tfrac{1}{2}(\phisperp-\phispar) - \tfrac{1}{4}(\phispar+\phisperp)\,\DGs\,t$  \\
      &  $\qquad\qquad\quad  + (\sparperp - \tfrac{1}{2}\phisperp + \tfrac{1}{2}\phispar) \cDms - \sDms
                           \Big]$  \\
    0S
      &  $-\CSP\magzeroAv\magSAv \Big[
                           \tfrac{1}{2}(\Cszero+\CsS)\cperpS - \tfrac{1}{2}(\phisS-\phiszero)\sperpS$  \\
      &  $\qquad\qquad\quad    - \big[ \tfrac{1}{2}(\CsS-\Cszero)\cperpS - \tfrac{1}{2}(\phiszero+\phisS)\sperpS \big]
                                 \cdot\tfrac{1}{2}\DGs\,t$ \\
      &  $\qquad\qquad\quad\   + \big[ \cperpS + (\szeroperp + \tfrac{1}{2}\phisS - \tfrac{1}{2}\phiszero)\sperpS \big]\cDms$  \\
      &  $\qquad\qquad\quad\ \ - \big[ (\szeroperp + \tfrac{1}{2}\phisS - \tfrac{1}{2}\phiszero)\cperpS - \sperpS \big]\sDms
                         \Big]$  \\
    $\parallel$S
      &  $+\CSP\magparAv\magSAv \Big[
                           \tfrac{1}{2}(\Cspar+\CsS)\cperpS - \tfrac{1}{2}(\phisS-\phispar)\sperpS$  \\
      &  $\qquad\qquad\quad    - \big[ \tfrac{1}{2}(\CsS-\Cspar)\cperpS - \tfrac{1}{2}(\phispar+\phisS)\sperpS \big]
                                 \cdot\tfrac{1}{2}\DGs\,t$ \\
      &  $\qquad\qquad\quad\   + \big[ \cperpS - (\sparperp - \tfrac{1}{2}\phisS + \tfrac{1}{2}\phispar)\sperpS \big]\cDms$  \\
      &  $\qquad\qquad\quad\ \ + \big[ (\sparperp - \tfrac{1}{2}\phisS + \tfrac{1}{2}\phispar)\cperpS + \sperpS \big]\sDms
                         \Big]$  \\
    $\perp$S
      &  $-\CSP\magperpAv\magSAv \Big[
                           \big[ \tfrac{1}{2}(\phisS-\phisperp)\cperpS - \sperpS \big] (1 + \tfrac{1}{2}\DGs\,t)$ \\
      &  $\qquad\qquad\quad\   - \big[ \tfrac{1}{2}(\phisS-\phisperp)\cperpS + \tfrac{1}{2}(\Csperp + \CsS)\sperpS \big]\cDms$  \\
      &  $\qquad\qquad\quad\ \ - \big[ \tfrac{1}{2}(\CsS-\Csperp)\cperpS - \tfrac{1}{2}(\phisperp + \phisS)\sperpS \big]\sDms
                         \Big]$  \\
    \hline
  \end{tabular}
\end{table}

\begin{table}[tbp]
  \centering
  \caption{Functions of decay time with approximations for small CP violation, $\phisi$\textequiv$\phis$, $\Csi$\textequiv$\Cs$,
           small $\DGs$, and phase differences close to 0 or $\pi$.}
  \label{tab:timeFunctionsApproxEqual}
  \renewcommand{\arraystretch}{1.7}
  \begin{tabular}{cl}
    \hline
    $ij$  &  $f(t) \times e^{+\Gs\,t}$  \\
    \hline
    $ii$
      &  $|\AAv|^2 \Big[
                     1 - \tfrac{1}{2}\,\eta_i\,\DGs\,t
                     + \Cs\,\cDms + \eta_i\,\phis\,\sDms
                   \Big]$  \\
    0$\parallel$
      &  $-\magzeroAv\magparAv \Big[
                                 1 - \tfrac{1}{2}\,\DGs\,t
                                 + \Cs\, \cDms + \phis\, \sDms
                               \Big]$  \\
    0$\perp$
      &  $+\magzeroAv\magperpAv \Big[
                                  \tfrac{1}{2}\,\phis\,\DGs\,t
                                  + \szeroperp\, \cDms + \sDms
                                \Big]$  \\
    $\parallel\perp$
      &  $-\magparAv\magperpAv \Big[
                                 \tfrac{1}{2}\,\phis\,\DGs\,t
                                 - \sparperp\, \cDms + \sDms
                               \Big]$  \\
    0S
      &  $-\CSP\magzeroAv\magSAv \Big[
                                   \Cs\,\cperpS
                                   + \tfrac{1}{2}\,\phis\,\sperpS\,\DGs\,t$ \\
      &  $\qquad                   + (\cperpS + \szeroperp\,\sperpS) \cDms
                                   - (\szeroperp\,\cperpS - \sperpS)\sDms
                                 \Big]$  \\
    $\parallel$S
      &  $+\CSP\magparAv\magSAv \Big[
                                  \Cs\,\cperpS
                                  + \tfrac{1}{2}\,\phis\,\sperpS\,\DGs\,t$ \\
      &  $\qquad                  + (\cperpS - \sparperp\,\sperpS)\cDms
                                  + (\sparperp\,\cperpS + \sperpS)\sDms
                                \Big]$  \\
    $\perp$S
      &  $+\CSP\magperpAv\magSAv \Big[
                                   \sperpS (1 + \tfrac{1}{2}\DGs\,t)$ \\
      &  $\qquad                   + \Cs\,\sperpS\,\cDms
                                   - \phis\,\sperpS\,\sDms
                                 \Big]$  \\
    \hline
  \end{tabular}
\end{table}

Integrating over the decay angles, the shape of the distribution of decay time is given by the sum of the diagonal terms in the
differential rate. As can be seen in Table~\ref{tab:timeFunctionsApprox}, CP violation only affects the oscillatory terms in this
distribution in the first-order approximation. These terms have opposite sign for $\Bs$ and $\Bsbar$ and can be isolated by taking the
difference between the two differential rates with $\qf$\texteq\tp1 and $\qf$\texteq\tm1 (see also Equation~\ref{eq:timeqfDep}). For a
single term, this difference is approximately given by
\begin{equation}
  \label{eq:timeOscill}
  \timeAmp[i,+1]^*\timeAmp[i,+1]^{\phantom{*}} - \timeAmp[i,-1]^*\timeAmp[i,-1]^{\phantom{*}}
    \propto \eGst\left[ \Csi\,\cDms + \eta_i\,\phisi\,\sDms \right] \ .
\end{equation}

The CP violation parameters also appear in the interference terms, which provides additional information on their values. This is
particularly noticeable in cases where the parameters appear in the $\cDGs$ coefficient, which is approximately equal to one and not
suppressed by the small value of $\DGs$ or by the effects of imperfect flavour tagging. In most cases this contributes to the estimates of
the differences between the $\phisi$ parameters, but in the ``0S'' and ``$\parallel$S'' terms also the values of $\Cszero$\textplus$\CsS$
and $\Cspar$\textplus$\CsS$.

Table~\ref{tab:timeFunctionsApproxEqual} shows the time dependence of the differential decay rate in the same approximation, but with the
additional requirement that CP violation is identical for all intermediate states. In this case the $\phisi$ differences vanish, but
$\Cszero$\textplus$\CsS$ and $\Cspar$\textplus$\CsS$ reduce to 2$\Cs$. As a result, a significant part of the sensitivity for $\Cs$ (or
$\lamsAbs$) comes from the ``0S'' and ``$\parallel$S'' terms, despite the small value of the S-wave amplitude.


%%%%%%%%%%%%%%%%%%%%%%%%%%%%%%%%%%%%%%%
\subsection{A Symmetry in the Equations}
\label{subsec:pheno_equations_symmetry}
%%%%%%%%%%%%%%%%%%%%%%%%%%%%%%%%%%%%%%%

In the limit of equal CP violation for all intermediate states (or small differences), Equation~\ref{eq:angCoefsReIm} reveals an
(approximate) symmetry in the decay-rate equations. As explained in Section~\ref{subsec:pheno_time_commonCPV}, simultaneously applying the
operations $\phis$\textto$\pi$\textminus$\phis$ and $\DGs$\textto\tm$\DGs$ is equivalent to taking the complex conjugates of the
coefficients $\Cf[ij]^\pm$, $\Df[ij]$, and $\Sf[ij]$ in this limit. The corresponding sign flip in the imaginary parts of the coefficients
can be cancelled by flipping the sign of either $\sin(\delta_j$\textminus$\delta_i)$ for Equation~\ref{eq:angCoefsRe} or
$\cos(\delta_j$\textminus$\delta_i)$ for Equation~\ref{eq:angCoefsIm}.

The specific structure of the appearance of real and imaginary parts of the products $c_i^*c_j$ in the angular coefficients in
Table~\ref{tab:angDist} enables the required sign flips in $\sin(\delta_j$\textminus$\delta_i)$ and $\cos(\delta_j$\textminus$\delta_i)$.
Simultaneously applying the operations $\phis$\textto$\pi$\textminus$\phis$, $\DGs$\textto\tm$\DGs$,
$\delparzero$\textto\tm$(\delparzero)$, $\delperpzero$\textto$\pi$\textminus$(\delperpzero)$, and
$\delSperp$\textto$\pi$\textminus$(\delSperp)$ does not change the value of the \BstoJpsiKK{} differential decay rate in the limit of equal
CP violation for all intermediate states. As a result, in this limit there is an alternative set of parameter values for each given set,
for which the decay model fits the experimental data equally well.

To resolve this discrete ambiguity, the measurement is performed in multiple intervals of $\KK$ mass. Integrating over $\mKK$ within each
interval gives different values for the phase $\thetaSP$, which affects the value $\delSperp$ that is measured for each interval. Comparing the behaviour
of $\delSperp$ across the intervals to what is expected from the $\KK$-mass models determines which of the two ambiguous sets of parameter
values corresponds to the physical situation~\cite{Xie:2009fs}.

Moving through the $\phimes$ resonance from low $\mKK$ to high $\mKK$, the phase of the $\phimes\to\KK$ contribution roughly increases by
$\pi$, while the phase of the $\fzero\to\KK$ contribution remains approximately constant. As a result, the phase $\thetaSP$ is expected to
increase from intervals at low $\mKK$ to intervals at high $\mKK$ and the phase difference $\delSperp$ is expected to decrease. This
behaviour has been shown to correspond to a value of $\phis$ close to zero and a positive value of $\DGs$, as opposed to $\phis$ close to
$\pi$ and negative $\DGs$~\cite{LHCb-PAPER-2011-028,LHCb-PAPER-2013-002,*LHCb-ANA-2012-067}.


%%%%%%%%%%%%%%%%%%%%%%%%%%%%%%%%%%%%
\subsection{Parameterization}
\label{subsec:pheno_equations_param}
%%%%%%%%%%%%%%%%%%%%%%%%%%%%%%%%%%%%

As mentioned in Section~\ref{subsec:pheno_equations_approx}, the small parameter $\Csi$ (Equation~\ref{eq:CsDef}) is used instead of
$\lamsAbs$, in addition to $\phisi$. As can be seen from Table~\ref{tab:timeFunctionsApprox}, the actual CP-violation observables in the
(angles-integrated) decay-time distribution are the sums of $\Csi$ and $\phisi$, weighted by the squared magnitudes of the corresponding
transversity amplitudes. These observables form the coefficients of the oscillatory functions in the diagonal terms of the differential
decay rate. Ignoring contributions from interference terms, these are also the observables that are measured with the assumption that CP
violation is identical for all intermediate states.

To parameterize in terms of these CP-violation observables as much as possible, the following linear combinations are defined:
\begin{equation}
  \label{eq:CPVParamDef}
  \begin{alignedat}{2}
    \Csav      &\equiv \tfrac{1}{2}\Cszero + \tfrac{1}{4}\Cspar + \tfrac{1}{4}\Csperp \qquad\quad&
      \phisav       &\equiv \phiszero + \tfrac{1}{2}\phispar - \tfrac{1}{2}\phisperp \\
    \DelCspara &\equiv \Cspar  - \Cszero &
      \Delphispara  &\equiv \phispar - \phiszero \\
    \DelCsperp &\equiv \Csperp - \Cszero &
      \Delphisperpp &\equiv \phisperp - \tfrac{1}{2}\phiszero - \tfrac{1}{2}\phispar \\
    \CsavS     &\equiv \tfrac{1}{2}\Cszero + \tfrac{1}{2}\CsS &
      \DelphisS  &\equiv \phisS - \phiszero \\
  \end{alignedat}
\end{equation}
The parameters $\Csav$ and $\phisav$ are the observables measured in the angles-integrated time distribution, ignoring interference terms,
also ignoring the S-wave, and approximating the ratio $\magzeroAvSq$\,:\,$\magparAvSq$\,:\,$\magperpAvSq$ by 2\,:\,1\,:\,1.  This last
approximation is justified by the values reported in reference~\cite{LHCb-PAPER-2013-002,*LHCb-ANA-2012-067}, which are given by
$\magzeroAvSq$\textapprox0.52, $\magparAvSq$\textapprox0.23, and $\magperpAvSq$\textapprox0.25. The normalization factors are chosen such
that $\Csav$\textto$\Cs$ and $\phisav$\textto$\phis$ in the case of equal CP violation for all states.

The remaining parameters are the differences between the parameters of the different states and the parameters of the longitudinal state,
with the exceptions of $\Delphisperpp$ and $\CsavS$. The parameter $\Delphisperpp$ is a combination of $\phisperp$\textminus$\phiszero$ and
$\phisperp$\textminus$\phispar$, which both appear as coefficients of $\cDGs$ in the interference terms. $\CsavS$ appears in the $\cDGs$
coefficient of the ``0S'' term. Using these parameters instead of $\phisperp$\textminus$\phiszero$ and $\CsS$\textminus$\Cszero$ reduces
correlations between the $\phisi$ parameters and between the $\Csi$ parameters, respectively.

In the case of equal CP violation for all intermediate states, the coefficients of the decay-time functions are given by (see also
Equation~\ref{eq:CDSCommon})
\begin{equation}
  \begin{aligned}
    \Cs &\equiv \frac{1-\lamsiSq[]}{1+\lamsiSq[]} \\
    \Ds &\equiv -\frac{2\,\Re(\lamsi[])}{1+\lamsiSq[]} = -\sqrt{1-\Cs^2}\, \cos\phis \approx -1 \\
    \Ss &\equiv +\frac{2\,\Im(\lamsi[])}{1+\lamsiSq[]} = -\sqrt{1-\Cs^2}\, \sin\phis \approx -\phis \ ,
  \end{aligned}
\end{equation}
where a first-order expansion in the parameters $\Cs$ and $\phis$ was used for the approximation. For historical reasons, the parameter
$\lamsAbs$ is used in this case instead of $\Cs$.

The magnitudes of the transversity amplitudes are parameterized by their squares, $|\AAv|^2$. Since only the shape of the time and angular
distributions are measured and not the absolute scale of the differential decay rate, the overall scale of the transversity amplitudes is
arbitrary. To fix the scale, the magnitudes of the amplitudes are multiplied by a common factor, such that the sum of the squares for the
$\Jpsiphi$ polarization states is equal to one:
\begin{equation}
  \label{eq:ampNorm}
  \magzeroAvSq + \magparAvSq + \magperpAvSq \equiv 1 \ .
\end{equation}
The parameters $\magzeroAvSq$ and $\magperpAvSq$ are used in the decay model and the value of $\magparAvSq$ follows from
Equation~\ref{eq:ampNorm}. This procedure makes the squared magnitudes of the $\Jpsiphi$ amplitudes essentially polarization fractions,
although one should keep in mind that these amplitudes are combinations of the $\Bs$ and $\Bsbar$ values, which also contain mixing
parameters (see Equation~\ref{eq:CPAvAmp}).

The magnitude of the S-wave amplitude is parameterized by a fraction, given by
\begin{equation}
  \FSAv \equiv \frac{\magSAvSq}{\magzeroAvSq + \magparAvSq + \magperpAvSq + \magSAvSq} = \frac{\magSAvSq}{1 + \magSAvSq} \ .
\end{equation}
As discussed in Section~\ref{sec:pheno_KKMass}, the differences between the phases of the transversity amplitudes are parameterized by
$\delparzero$, $\delperpzero$, and $\delSperp$. Both the S-wave fraction and the phase difference between the S-wave and the $\Jpsiphi$
polarizations are measured in intervals of $\KK$ mass to resolve the discrete ambiguity discussed in
Section~\ref{subsec:pheno_equations_symmetry}: $\FSAv[,b]$ and $\delSperp[,b]$.

Values of the parameters in the model are estimated by fitting the shape of the differential decay-rate equation to the \BstoJpsiKK{}
data. To be able to describe the experimental data several experimental effects have to be added to the model that was discussed in this
chapter. The final experimental model will be discussed in Chapter~\ref{chap:ana}.

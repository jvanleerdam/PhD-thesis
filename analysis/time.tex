\section{Decay Time}
\label{sec:ana_time}

The theoretical model of the decay time of \BstoJpsiKK{} decays is distorted by two experimental effects; the uncertainty in the time
measurement (\emph{resolution}) and the efficiency of the measurement (\emph{acceptance}) as a function of time. As discussed in
Section~\ref{subsec:intro_LHCb_Jpsiphi}, the resolution is roughly 0.05\unitsp{}ps. The resolution model is presented in
Section~\ref{subsec:ana_time_res}. The decay-time measurement is affected by non-trivial acceptance effects from both the trigger and
reconstruction processes, which will be discussed in Section~\ref{subsec:ana_time_acc}.


%%%%%%%%%%%%%%%%%%%%%%%%%%%
\subsection{Resolution}
\label{subsec:ana_time_res}
%%%%%%%%%%%%%%%%%%%%%%%%%%%

The uncertainty in the decay-time measurement causes a difference between the measured decay time and the true decay time. This difference
is a random variable and the resulting measured time distribution is a smeared version of the underlying true distribution. For the
oscillatory $\cDms$ and $\sDms$ functions in the differential decay rate this smearing causes a decrease, or \emph{dilution}, of the
oscillation amplitude. As described in Section~\ref{subsec:pheno_equations_approx}, the main sensitivity for the CP-violation parameters
originates from the oscillation terms and hence a diluted oscillation amplitude reduces the statistical precision of the estimates for
these parameters.

For each decay the decay-time uncertainty is estimated by propagating the uncertainties in the positions of the primary and secondary
vertices and particle momenta, as discussed in Section~\ref{subsec:intro_LHCb_Jpsiphi}. The distribution of this estimate ($\sigmat$) is
shown in Figure~\ref{fig:sigmat}. For each decay, the probability for the measured decay time to deviate by a given amount from the true
decay time is approximately described by a Gaussian distribution with width $\sigmat$.
\begin{figure}[htbp]
  \centering
  \includegraphics[width=0.5\textwidth]{graphics/analysis/sigmat}
  \caption{Distribution of \BstoJpsiKK{} signal decays in the estimated decay-time uncertainty.}
  \label{fig:sigmat}
\end{figure}

Decay-time resolution is included in the model for the signal decay by convolving the theoretical model with the model for the difference
between the measured time and the true time. This convolution is the integral of the two-dimensional PDF of the true and measured decay
times over all possible values of the former:
\begin{equation}
  P_\text{meas}(t_\text{meas}, \Omega)
    \equiv \int_0^\infty \ud t_\text{true}\, R(t_\text{meas} - t_\text{true}|\sigmat)\, P_\text{true}(t_\text{true}, \Omega)\ ,
\end{equation}
where $R$ is the PDF for the difference between the measured and true times, or the \emph{resolution model}. The resolution model is
conditional on $\sigmat$., i.e. normalized with respect to decay time for each individual value of $\sigmat$.

Although the resolution model is approximated by a Gaussian PDF with width $\sigmat$, a more sophisticated model is required to describe
the resolution in the PDF with sufficient precision. The model is a sum of two Gaussian PDFs with a common, non-zero mean and a width that
depends quadratically on $\sigmat$. The width of the first Gaussian PDF is approximately equal to $\sigmat$, while the width of the second
Gaussian PDF is roughly a factor two larger.

The double-Gaussian model is based on and validated with both simulated \BstoJpsiKK{} decays and prompt background decay candidates. Since
all tracks of prompt background candidates originate from the primary vertex, their ``true'' decay time is equal to zero and the
distribution of measured decay times is essentially the resolution model. Studies of the model and its parameters are described in detail
in references~\cite{Aaij:2015} and \cite{LHCb-ANA-2014-039}. The values of the resolution parameters are fixed in the fit of decay time and
angles. Uncertainties in the parameter values are propagated after the fit and accounted for as systematic uncertainties (see
Section~\ref{sec:result_syst}).


%%%%%%%%%%%%%%%%%%%%%%%%%%%
\subsection{Acceptance}
\label{subsec:ana_time_acc}
%%%%%%%%%%%%%%%%%%%%%%%%%%%

To account for the efficiencies of the decay-candidate reconstruction and selection processes, the PDF is multiplied by an acceptance
function and re-normalized to create a new PDF. The acceptance is is modelled as a product of two functions of decay time and one function
of decay angles. The angular part will be discussed in Section~\ref{sec:ana_angles}. This section describes the two decay-time functions.

\subsubsection{Track-Reconstruction Acceptance}
The first part of the non-trivial acceptance in decay time originates from an inefficiency in the reconstruction of particle tracks. This
efficiency decreases for increasing distance between the track and the proton beams. Since the distance between the primary and secondary
vertex and, therefore, the decay time are correlated with the distance between the four \BstoJpsiKK{} tracks and the beams, the efficiency
also decreases with increasing decay time.

In the measurement presented here the track-reconstruction acceptance is modelled by an exponential function in \emph{true} decay time. The
advantage of this model is that its implementation in the model of the decay-time distribution is straightforward. An exponential function
can be absorbed in the $\eGst$ factor of the differential decay rate (Equation~\ref{eq:angCoefs}):
\begin{equation}
  \eGst \longrightarrow e^{\beta\,t}\,\eGst = e^{-(\Gs-\beta)\,t} \equiv e^{-\Gs^\text{eff}\,t}\ ,
\end{equation}
where $\beta$ is the parameter that quantifies the rate at which the efficiency changes as a function of decay time. The parameter
$\Gs^\text{eff}$\textequiv$\Gs$\textminus$\beta$ can now be included in the model in the place of the parameter $\Gs$.

The parameter $\beta$ has been determined by a combination of studies with real and simulated data~\cite{LHCb-ANA-2014-039}. Because of
changes in the online reconstruction algorithms between the 2011 and 2012 runs and the different proton-collision energies in these
periods, the corresponding values of $\beta$ are evaluated separately: $\beta_\text{2011}$\texteq\mbox{\tm0.0090\textpm0.0022\unitsp\invps}
and $\beta_\text{2012}$\texteq\mbox{\tm0.0124\textpm0.0019\unitsp\invps}. These values have to be compared with
$\Gs$\textapprox0.66\unitsp\invps.

Although there is no sensitivity in the \BstoJpsiKK{} data to the parameters $\Gs$, $\beta_\text{2011}$, and $\beta_\text{2012}$
separately, the value of $\Gs^\text{eff}$ can be determined in the fit of decay time separately for the 2011 and 2012 periods.
Reparameterizing, there is sensitivity for the combinations $\Gs$\textminus$\tfrac{1}{2}(\beta_\text{2011}$\textplus$\beta_\text{2012})$
and $\beta_\text{2012}$\textminus$\beta_\text{2011}$.

To combine the information on the difference between the two $\beta$ values with the externally determined values, the $\beta$ parameters
are varied with constraints in the time and angular fit. These constraints are implemented by adding a parabolic term to the NLL of the
form $\frac{1}{2\hat{\sigma}^2}(\beta-\hat{\beta})^2$ for each parameter, where the external value is denoted by $\hat{\beta}$ and the
external uncertainty by $\hat{\sigma}$. Because these external constraints are much tighter than the constraints from the \BstoJpsiKK{}
data, the values that are estimated in the fit are comparable to the external values:
$\beta_\text{2011}$\texteq\mbox{\tm0.0086\textpm0.0021\unitsp\invps} and
$\beta_\text{2012}$\texteq\mbox{\tm0.0127\textpm0.0018\unitsp\invps}.

There are some limitations to this model of the track-reconstruction acceptance. The model is implemented for true decay time, whereas the
external values of the $\beta$ parameters were evaluated with the measured decay time. In this measurement, this is assumed to be a good
approximation, because the time scale of the variations in the model is much larger than the resolution. That is,
$\beta^{-1}$\textapprox\mbox{\tenpow{2}\unitsp{}ps}\textgg\mbox{0.05\unitsp{}ps}.

Also the model itself is an approximation. As was shown in reference~\cite{LHCb-ANA-2014-039}, the shape of the acceptance is better
described by the function 1\textplus$\beta\,t$\textplus$\beta'\,t^2$ than by $e^{\beta\,t}$\textapprox1\textplus$\beta\,t$. A systematic
uncertainty in the parameter estimates corresponding to the assumption of the shape $e^{\beta\,t}$ is estimated in
Section~\ref{sec:result_syst}.

\subsubsection{Trigger Acceptance}
As described in Section~\ref{subsec:ana_bkgSub_sel}, there are also trigger requirements that introduce non-trivial acceptance effects in
decay time. The shapes of the acceptance functions corresponding to the decay-time biasing trigger categories are determined relative to
the shapes of the unbiased categories, which have a uniform acceptance function.

The shape of the trigger acceptance is described in bins of decay time. For each bin, numbers of decays are counted in the different
trigger categories to determine the relative efficiency in the bin. To also include information on the exponential shape of the decay in
the resulting binned acceptance function, the decay counts are varied in the fit of decay time and angles.

Since originally only the shape of the differential decay rate in time and angles is determined in the fit, additional terms need to be
included in the NLL to count decays in the different trigger categories. For unweighted decays the PDF for the number of decays in a
category would be a Poisson distribution, which is proportional to $\nu^n\,e^{-\nu}$, where $n$ is the observed number of decays and $\nu$
is the parameter for the expected number of decays. This PDF would give an additional NLL term of $\nu$\textminus$n\,\ln\nu$. However,
because each decay candidate is counted with its signal weight, this term is modified to obtain the correct uncertainty on the parameter
$\nu$ from a maximum-likelihood fit.

Instead of only replacing the variable $n$ in the Poissonian NLL term by the sum of the decay-candidate weights, the term is also
multiplied by weight factor:
\begin{equation}
  \frac{\sum w}{\sum w^2}\left( \nu - \ln\nu\,\sum w \right)\ ,
\end{equation}
where $\sum w$ is the sum of the decay weights and $\sum w^2$ the sum of the squared decay weights. This function reaches its minimum at
$\nu$\texteq$\sum w$, so the estimated value for $\nu$ in a maximum-likelihood fit with only this function would be the sum of the decay
weights. The inverse of the second derivative of the function, from which the uncertainty in $\nu$ is estimated, is given by $\sum w^2$
instead of $\sum w$ for an unmodified Poisson term. The former number is equal to the variance that is expected when counting weighted
decay candidates.

\begin{itemize}
  \item describe trigger categories and efficiency parameters
  \item using all categories (6) vs merging categories (4 or 5)
  \item short description of resulting acceptance functions (plots): $\longrightarrow$ Roel
\end{itemize}

%Since only decay candidates that are selected by the HLT2-biased trigger line are used for the fit of decay time and angles, only the
%trigger-acceptance functions for the HLT1-unbiased/HLT2-biased and exclusively-HLT1-biased/HLT2-biased combinations are required.

\begin{figure}[htbp]
  \centering
  \begin{subfigure}{0.49\textwidth}
    \includegraphics[width=\textwidth]{graphics/analysis/trigTimeAcc_2011_UB}
    \caption{}
    \label{fig:trigAcc_2011_UB}
  \end{subfigure}%
  \hfill%
  \begin{subfigure}{0.49\textwidth}
    \includegraphics[width=\textwidth]{graphics/analysis/trigTimeAcc_2011_exclB}
    \caption{}
    \label{fig:trigAcc_2011_exclB}
  \end{subfigure}

  \vspace*{0.02\textwidth}
  \begin{subfigure}{0.49\textwidth}
    \includegraphics[width=\textwidth]{graphics/analysis/trigTimeAcc_2012_UB}
    \caption{}
    \label{fig:trigAcc_2012_UB}
  \end{subfigure}%
  \hfill%
  \begin{subfigure}{0.49\textwidth}
    \includegraphics[width=\textwidth]{graphics/analysis/trigTimeAcc_2012_exclB}
    \caption{}
    \label{fig:trigAcc_2012_exclB}
  \end{subfigure}
  \caption{Trigger acceptance in bins of decay time for (a, b) the 2011 run and (c, d) the 2012 run
           and for (a, c) the HLT1-unbiased/HLT2-biased selection and (b, d) the exclusively-HLT1-biased/HLT2-biased selection.
           Notice that the efficiency range of the HLT1-unbiased/HLT2-biased graphs is 90--100\%.
           Because the absolute efficiency of the HLT1 selections is unknown, the scales of the efficiencies
           in the graphs are given with respect to the efficiency of the HLT1-unbiased selection.}
  \label{fig:trigAcc}
\end{figure}


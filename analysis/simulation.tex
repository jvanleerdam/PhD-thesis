\section{Simulation}
\label{sec:ana_sim}
\begin{itemize}
  \item candidate weights and conditional observables in toys from fit dataset
\end{itemize}

Datasets of simulated decays are used to study the statistical precision of parameter estimates and the effect of the detector on
distributions of relevant variables. These data are produced using Monte Carlo methods, where random values are generated for the variables
that describe a decay. This can be achieved either by directly generating values for the decay time and decay angles or by generating the
particles in the decay and their interactions with the detector.

The latter possibility is a full simulation of the proton--proton collisions of the LHC, the decays of particles that are produced, and the
response of the LHCb detector to the resulting particles. These stages of the simulation are performed separately for each simulated LHC
event, generating values for variables that describe particles and detector components. The resulting data are stored in the format of real
detector data and are processed in the same way by applying trigger and selection requirements. This type of simulation is applied to
determine the contribution of resonant backgrounds (see Section~\ref{subsec:ana_bkgSub_bkgSub}), the detector resolution (see
Section~\ref{subsec:ana_bkgSub_bkgSub} for $\JpsiKK$ mass and Section~\ref{subsec:ana_time_res} for decay time), and the detector
acceptance (see Section~\ref{subsec:ana_time_acc} for decay time and Section~\ref{sec:ana_angles} for decay angles).

In the direct type of simulation the values of decay time and decay angles in \BstoJpsiKK{} decays are drawn from the PDF that is
constructed with the model discussed in Chapter~\ref{chap:pheno} and this chapter. All effects of $\Bs$ production, \BstoJpsiKK{} decay,
detection, reconstruction, and selection are then assumed to be described by this model, as in the fit of time and angles. This type of
simulation is applied to generate so called \emph{pseudo experiments}, each of which consists of a dataset that is equivalent to the
dataset obtained from the real experiment. Distributions of the parameter estimates in the final fit are obtained by performing the time
and angular analysis on each of the pseudo experiments. These distributions are used to determine the statistical uncertainty in the
parameter estimates (see Section~\ref{sec:result_paramEst}) and the effects of variations in the assumed model.

To get a realistic estimate of the parameter distributions from the pseudo experiments not only signal decays, but also background
decay candidates should be generated. Because no PDF for the time and angular distributions of these candidates is constructed for this
measurement, these distributions are taken from real background data.

Background decay-candidates are selected from the real data by taking data from the $\JpsiKK$-mass side bands. As in the
background-subtraction procedure, it is assumed that the signal contribution in the side bands is small (and in this case negligible) and
the time and angular distributions of background candidates in the side bands are representative for the full mass range (see
Section~\ref{sec:ana_bkgSub}). The distribution of time and angles of the real candidates are used to generate the distribution of
background in the pseudo experiments.

Values of the $\JpsiKK$ mass for signal and background candidates are generated by the respective mass PDFs that are also used in the
background-subtraction procedure. Both these mass PDFs and PDFs for decay time and decay angles are conditional on a number of other
variables in the decay, such as the run-period category (2011 or 2012), the $\KK$-mass category, the trigger category, the estimated
decay-time uncertainty, and the tagging variables. As for the time and angular distributions of background candidates, no PDFs are
available for these conditional variables and their distributions are taken from the real data.

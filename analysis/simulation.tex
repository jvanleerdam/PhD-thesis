\section{Simulation}
\label{sec:ana_sim}
\begin{itemize}
  \item \underline{briefly} discuss detector simulation
  \item toys: generate signal with model and background with real side-band data
  \item candidate weights and conditional observables in toys from fit dataset
\end{itemize}

Datasets of simulated decays are used to study the statistical precision of parameter estimates and the effect of the detector on
distributions of relevant variables. These data are generated using Monte Carlo methods, where the values of the variables that describe a
decay are drawn from the input distributions of those variables.

Two types of simulation are applied for this measurement. The first is a full simulation of the proton--proton collisions of the LHC, the
decays of particles that are produced, and the response of the LHCb detector to the resulting particles. These stages of the simulation are
performed separately for each simulated LHC event, generating values for variables that describe particles and detector components. The
resulting data are stored in the format of real detector data and are processed in the same way by applying trigger and selection
requirements. This type of simulation is applied to determine the contribution of resonant backgrounds (see
Section~\ref{subsec:ana_bkgSub_bkgSub}), the detector resolution (see Section~\ref{subsec:ana_bkgSub_bkgSub} for $\JpsiKK$ mass and
Section~\ref{subsec:ana_time_res} for decay time), and the detector acceptance (see Section~\ref{subsec:ana_time_acc} for decay time and
Section~\ref{sec:ana_angles} for decay angles).

In the second type of simulation the values of decay time and decay angles in \BstoJpsiKK{} decays are drawn from the PDF that is
constructed with the model discussed in Chapter~\ref{chap:pheno} and this chapter. All effects of $\Bs$ production, \BstoJpsiKK{} decay,
detection, reconstruction, and selection are then assumed to be described by this model, as in the fit of time and angles. This type of
simulation is applied to generate so called \emph{pseudo experiments}, each of which consists of a dataset that is equivalent to the
dataset obtained from the real experiment. Distributions of the parameter estimates in the final fit are obtained by performing the time
and angular analysis on each of the pseudo experiments. These distributions are used to determine the statistical uncertainty in the
parameter estimates (see Section~\ref{sec:result_paramEst}) and the effects of variations in the assumed model.

\section{Simulation}
\label{sec:ana_sim}

Datasets of simulated decays are used to study the statistical precision of parameter estimates and the effect of the detector on
distributions of relevant variables. These data are produced using Monte Carlo methods, where random values are generated for the variables
that describe a decay. This can be achieved either by directly generating values for the decay time and decay angles or by generating the
particles in the decay and their interactions with the detector.

The latter option involves a full simulation of the proton--proton collisions of the LHC, the decays of particles that are produced, and
the response of the LHCb detector to the resulting particles. These stages of the simulation are performed separately for each simulated
LHC event, generating values for variables that describe particles and detector components. The resulting data are stored in the format of
real detector data and are processed in the same way by applying trigger and selection requirements. This type of simulation is applied to
determine the contribution of resonant backgrounds (see Section~\ref{subsec:ana_bkgSub_bkgSub}), the detector resolution (see
Section~\ref{subsec:ana_bkgSub_bkgSub} for $\JpsiKK$ mass and Section~\ref{subsec:ana_time_res} for decay time), and the detector
acceptance (see Section~\ref{subsec:ana_time_acc} for decay time and Section~\ref{sec:ana_angles} for decay angles).

In the direct type of simulation the values of decay time and decay angles in \BstoJpsiKK{} decays are drawn from the PDF that is
constructed with the model discussed in Chapter~\ref{chap:pheno} and this chapter. All effects of $\Bs$ production, \BstoJpsiKK{} decay,
detection, reconstruction, and selection are then assumed to be described by this model, as in the fit of time and angles. This type of
simulation is applied to generate so-called \emph{pseudo experiments}, each of which consists of a dataset that is equivalent to the
dataset obtained from the real experiment. Distributions of the parameter estimates in the final fit are obtained by performing the time
and angular analysis on each of the pseudo experiments. These distributions are used to determine the statistical uncertainty in the
parameter estimates (see Section~\ref{sec:result_paramEst}) and the effects of variations in the assumed model.

To get a realistic estimate of the parameter distributions from the pseudo experiments not only signal decays, but also background
decay candidates should be generated. Because no PDF for the time and angular distributions of these candidates is constructed for this
measurement, these distributions are taken from real background data.

Background decay-candidates are selected from the real data by taking data from the $\JpsiKK$-mass side bands. As in the
background-subtraction procedure, it is assumed that the signal contribution in the side bands is small (and in this case negligible) and
the time and angular distributions of background candidates in the side bands are representative for the full mass range (see
Section~\ref{sec:ana_bkgSub}). The distribution of time and angles of the real candidates are used to generate the distribution of
background in the pseudo experiments.

Values of the $\JpsiKK$ mass are generated with the mass PDF that is also used in the background-subtraction procedure. Since the mass
distribution is not modelled in the final fit of decay time and angles, statistical fluctuations in this distribution are not propagated to
the distributions of the fit parameters. A single set of $\JpsiKK$-mass values and corresponding signal weights is generated, which is used
for all pseudo experiments. The number of decay candidates in this set is equal to the number of candidates in the real experiment.

Both the $\JpsiKK$-mass PDFs and PDFs for decay time and decay angles are conditional on a number of other variables in the decay, such as
the run-period category (2011 or 2012), the $\KK$-mass category, the trigger category, the estimated decay-time uncertainty, and the
tagging variables. As for the time and angular distributions of background candidates, no PDFs are available for these conditional
variables and their distributions are taken from the real data.

The distributions of conditional variables are not modelled in the time and angular fit, so a single set of values is used for all pseudo
experiments, as for the $\JpsiKK$ mass. This set is ``generated'' by taking the values of the conditional variables from the decay
candidates in the real data. Differences between the signal and the background distributions are taken into account by generating the
$\JpsiKK$-mass values in intervals of the conditional variables, using the fractions of signal and background candidates that are found for
each individual interval.

Pseudo experiments are produced by generating values for the decay time and the decay angles for each candidate in the set of mass and
conditional-variable values. For each candidate it is first decided whether to draw values from the signal or from the background
distribution for time and angles. The probabilities to get either one of the distributions are given by the signal and background
fractions, which are determined by the values of the mass PDFs and the fractions in the conditional-variable interval of the decay
candidate.  As described above, the values of time and angles are generated with the signal PDF for ``signal candidates'' and with the
side-band distributions for ``background candidates''.

The application of pseudo experiments is discussed in the next chapter, where the results of the \BstoJpsiKK{} measurement are presented.
Parameter uncertainties as estimated from the shape of the NLL are verified with the distributions of parameter estimates in pseudo
experiments. In addition to these statistical uncertainties, systematic uncertainties in the parameter estimates are evaluated.

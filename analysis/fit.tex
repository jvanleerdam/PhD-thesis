\section{Maximum-Likelihood Fit}
\label{sec:ana_fit}

The fit of the decay model to the data is an \emph{unbinned maximum-likelihood fit}. A probability density function (PDF) is constructed by
normalizing the expression for the differential decay rate, including experimental effects, by dividing by its integral over decay time and
decay angles. A likelihood function of the PDF parameters for one $\Bs$ decay is given by the PDF at the values of time and angles for
that decay. The likelihood function for the full sample of decays is given by the product of all individual likelihoods.

Parameter values are estimated by maximizing the combined likelihood function for the sample. In practice, the negative logarithm of the
likelihood function (NLL) is minimized to find the maximum likelihood. Instead of a product, the NLL is a sum of the contributions from
individual decays.

The shape of the NLL around the minimum can be approximated by a second order Taylor series, i.e. a parabola. Since the first derivative of
this function vanishes at the point where the NLL reaches its minimum, the approximation for a given parameter $\mu$ can be written in the
form
\begin{equation}
  \label{eq:NLLPara}
  \text{NLL}(\mu) \approx \frac{1}{2\,\sigma_\mu^2}\, (\mu-\hat{\mu})^2 + C \ ,
\end{equation}
where $\hat{\mu}$ is the value of parameter $\mu$ in the minimum, $\sigma_\mu$ determines the width of the NLL shape around the minimum,
and $C$ is the NLL value in the minimum. The width, $\sigma_\mu$, is (related to) the statistical uncertainty of the estimate of parameter
$\mu$. The wider the NLL shape around the minimum is, the larger the value of $\sigma_\mu$ and the uncertainty are.

It depends on the actual shape of the NLL how well it is approximated by a parabola away from the minimum. Important factors are
correlations between parameters and the amount of data that is used to build the NLL. With more data the statistical uncertainties of
the parameter estimates become smaller and, in general, the NLL becomes more parabolic within an interval of a few times the value of
$\sigma_\mu$ around the minimum.

In case the distribution of parameter estimates $\hat{\mu}$ from different measurements is described by a Gaussian shape, the shape of the
NLL will be truly parabolic. In this case the parameter $\sigma_\mu$ is an estimate of the standard deviation of the $\hat{\mu}$
distribution, which can be used as a measure of the statistical uncertainty of a parameter estimate. The value of $\sigma_\mu$ is
determined from the second derivative of the NLL, which is given by $\frac{1}{\sigma_\mu^2}$.

If the shape of the NLL around its minimum is not sufficiently parabolic, a different measure of the statistical uncertainty may be used.
Commonly the (absolute) difference between the value of $\mu$ at the point where the NLL reaches a value of $\tfrac{1}{2}+C$ and
$\hat{\mu}$ is taken as the uncertainty. In the parabolic case the NLL reaches this point at $\hat{\mu}+\sigma_\mu$ and
$\hat{\mu}-\sigma_\mu$, which results in an uncertainty of $\sigma_\mu$, as expected. In a more general case this NLL value may be reached
on the left and the right of $\hat{\mu}$ at different distances, which results in an asymmetric uncertainty.

In some cases the shape of the NLL differs from a parabola too much to give a single estimate of the parameter value and a corresponding
uncertainty. The simplest way to provide an estimate of the value in these cases is to specify the interval bounded by the values of $\mu$
at a given NLL value. Common NLL values are the values that a parabola would reach at $n\cdot\sigma_\mu$ from the minimum, which are given
by $\tfrac{1}{2}\,n^2+C$. Hence the intervals are referred to as ``$n$ sigma'' intervals.

In general the NLL is a function of more than one parameter. The distribution of parameter estimates in the parabolic approximation is
given by a multivariate Gaussian shape, which includes also correlations between the estimates. Both the uncertainties and the correlations
of the parameter estimates are estimated from the shape of the NLL in the minimum, by determining second derivatives for all combinations
of parameters.

The parabolic shape of Equation~\ref{eq:NLLPara} for each one of the parameters is given by the minimum value of the NLL at each value of
the parameter $\mu$. In the parabolic approximation the minimum is reached at
\begin{equation}
  \frac{1}{\sigma_\nu}\, (\nu-\hat{\nu})  = \rho_{\mu\nu}\, \frac{1}{\sigma_\mu}\, (\mu-\hat{\mu})
\end{equation}
for any other parameter $\nu$, where $\rho_{\mu\nu}$ is the correlation coefficient between the parameters $\mu$ and $\nu$. The resulting
function of parameter $\mu$ is termed \emph{profiled NLL}, the second derivative of which is again equal to $\frac{1}{\sigma_\mu^2}$.

\section{Maximum-Likelihood Fit}
\label{sec:ana_fit}

The fit of the decay model to the data is an \emph{unbinned maximum-likelihood fit}. A probability density function (PDF) is constructed by
normalizing the expression for the differential decay rate, including experimental effects, by dividing by its integral over decay time and
decay angles. A likelihood function of the PDF parameters for one $\Bs$ decay is given by the PDF at the values of time and angles for
that decay. The likelihood function for the full sample of decays is given by the product of all individual likelihoods.

Parameter values are estimated by maximizing the likelihood function for the sample. In practice, the negative logarithm of the likelihood
function (NLL) is minimized to find the maximum likelihood. Instead of a product, the NLL is a sum of the contributions from individual
decays.

The shape of the NLL around the minimum can be approximated by a second order Taylor series, i.e. a parabola. Since the first derivative of
this function vanishes at the point where the NLL reaches its minimum, the approximation for a given parameter $\mu$ can be written in the
form
\begin{equation}
  \label{eq:NLLPara}
  \text{NLL}(\mu) \approx \frac{1}{2\,\sigma_\mu^2}\, (\mu-\hat{\mu})^2 + C \ ,
\end{equation}
where $\hat{\mu}$ is the value of parameter $\mu$ in the minimum, $\sigma_\mu$ determines the width of the NLL shape around the minimum,
and $C$ is the NLL value in the minimum. The width, $\sigma_\mu$, is (related to) the statistical uncertainty of the estimate of parameter
$\mu$. The wider the NLL shape around the minimum, the larger the value of $\sigma_\mu$ and thus the uncertainty.

It depends on the actual shape of the NLL how well it is approximated by a parabola away from the minimum. An important factor is the
number of decays that is used to build the NLL. With more data the statistical uncertainties of the parameter estimates become smaller and,
in general, the NLL becomes more parabolic within an interval of a few times the value of $\sigma_\mu$ around the minimum.

In case the distribution of parameter estimates $\hat{\mu}$ from different measurements is described by a Gaussian shape, the shape of the
NLL will be truly parabolic. In this case the parameter $\sigma_\mu$ is an estimate of the standard deviation of the $\hat{\mu}$
distribution, which can be used as a measure of the statistical uncertainty of a parameter estimate. The parameter interval between
$\hat{\mu}$\textminus$\sigma_\mu$ and $\hat{\mu}$\textplus$\sigma_\mu$ is called a \emph{confidence interval}, which contains the true
value of $\mu$ in 68\% of the measurements in the Gaussian case (a \emph{coverage} of 68\%). The value of $\sigma_\mu$ is determined from
the second derivative of the NLL, which is given by $\frac{1}{\sigma_\mu^2}$.

If the shape of the NLL around its minimum is not sufficiently parabolic, the value of $\sigma_\mu$ and the interval defined by the values
$\hat{\mu}$\textpm$\sigma_\mu$ are still well defined, but may not provide the desired measure of the uncertainty in the parameter
estimate.  In general, the value of $\sigma_\mu$ is not an estimate of the $\hat{\mu}$ standard deviation and the confidence interval does
not have a coverage of 68\%.

A more general confidence interval with a coverage of approximately 68\% can be constructed with the values of $\mu$ at the points where
the NLL reaches a value of $\tfrac{1}{2}+C$. In the parabolic case the NLL reaches this point at $\hat{\mu}$\textpm$\sigma_\mu$, which
results in the expected Gaussian confidence interval with corresponding uncertainty $\sigma_\mu$. In a more general case this NLL value
may be reached at different distances on the left and the right of $\hat{\mu}$, which results in an asymmetric uncertainty.

If the NLL has multiple minima or a very broad, shallow minimum, a point estimate of the parameter value is not representative for the
$\hat{\mu}$ distribution and the estimate is usually represented by confidence intervals only. A straightforward way of constructing these
intervals is to determine the points between which the NLL is smaller than a certain value, as was done above with $\tfrac{1}{2}+C$. Common
NLL values to use are the values that a parabola would reach at $n\cdot\sigma_\mu$ from the minimum, which are given by
$\tfrac{1}{2}\,n^2+C$. Hence the intervals are referred to as ``$n$ sigma'' intervals.

In general the NLL is a function of multiple parameters. The NLL as a function of one parameter, $\mu$, is obtained by minimizing the
function with respect to all other parameters for each value of $\mu$. The estimate of $\mu$ and its uncertainty are determined from this
minimized NLL. The likelihood function corresponding to this NLL for $\mu$ is called a \emph{profile likelihood}.

In the (approximate) Gaussian case, the distribution of parameter estimates is given by a multivariate Gaussian shape, which includes also
correlations between the estimates. A covariance matrix takes the place of $\sigma_\mu^2$ in Equation~\ref{eq:NLLPara},
which contains the standard deviations and correlation coefficients of the parameter estimates.
The  are estimated from the shape of
the NLL in the minimum, by determining second derivatives for all pairs of parameters.

The parabolic shape of Equation~\ref{eq:NLLPara} for a parameter $\mu$ is given by the minimum of the NLL at each $\mu$ value. In the
parabolic approximation the minimum for a given value of $\mu$ is reached at
\begin{equation}
  \frac{1}{\sigma_{\nu_i}}\, (\nu_i-\hat{\nu}_i)  = \rho_{\mu\nu_i}\, \frac{1}{\sigma_\mu}\, (\mu-\hat{\mu})
\end{equation}
for all the other parameters $\nu_i$, where $\rho_{\mu\nu_i}$ is the correlation coefficient between the parameters $\mu$ and $\nu_i$., the second derivative of which is again equal to
$\frac{1}{\sigma_\mu^2}$.


%%%%%%%%%%%%%%%%%%%%%%%%%%%%%%%%%%%%%%%%%%%%%%%
\subsection{Fit with Weighted Decay Candidates}
\label{subsec:ana_fit_weights}
%%%%%%%%%%%%%%%%%%%%%%%%%%%%%%%%%%%%%%%%%%%%%%%

As will be described in Section~\ref{sec:ana_bkgSub}, the time and angular distribution for \BstoJpsiKK{} signal decays is obtained by
subtracting the background distribution from the distribution that is observed in the data. This is accomplished by adding background
decay-candidates to the sample with negative weights. In the NLL, the contribution of each decay candidate is then multiplied with the
value of its weight.

Although the position of the minimum of a weighted NLL still gives a good estimate of the values of the NLL parameters, the parameter
uncertainties cannot be estimated directly from the shape of the NLL at its minimum any more. This can be seen by considering a fit in
intervals of a given variable, where the observed number of decays in each interval is compared to the prediction of this number by a
model. The uncertainties in the estimates of the model parameters are now related to the uncertainties in the observed number of decays in
each interval.

For unweighted decays the distribution of the observed number of decays ($N$) is a Poisson distribution, for which both the mean and the
variance are given by the expected number in the interval ($\nu$). An estimate of the expected number of decays, $\hat{\nu}$, is given by
$N$ and an estimate of the corresponding uncertainty by the square root of the estimated variance, $\sqrt{N}$.

In a fit with weighted decay candidates, where each candidate counts with a weight $w_c$, the observed number of decays is replaced by the
sum of weights, $W$\textequiv$\sum_c w_c$. The estimate of $\nu$ is now also given by $W$, as expected, but the uncertainty is estimated by
$\sqrt{W}$, which cannot be correct. If all weights are multiplied by a constant number, $n$, the relative uncertainty in the estimate of
$\nu$ should not change, since no information was added to the data sample. This means that the absolute uncertainty should increase by a
factor $n$, as $\hat{\nu}$\texteq$W$ does. If the uncertainty is estimated by $\sqrt{W}$, it only increases by a factor $\sqrt{n}$.

The correct uncertainty estimate for the expected number of decays is given by the square root of $W'$\textequiv$\sum_c w_c^2$. Unlike
$\sqrt{W}$, $\sqrt{W'}$ increases by a factor $n$ if all weights are multiplied by this common factor. To obtain this estimate of the
uncertainty, the original estimate from the Poisson distribution should be divided by a factor $\sqrt{\alpha}$, where $\alpha$ is given by
\begin{equation}
  \alpha \equiv \frac{W}{W'} = \frac{\sum_c w_c}{\sum_c w_c^2} \ .
\end{equation}

In an unbinned maximum likelihood fit the correction factor $\alpha$ can be used to modify the shape of the NLL. Since the uncertainties
are estimated from the second derivatives of the NLL, which are given by $\frac{1}{\sigma_\mu^2}$ in the parabolic case, the NLL is
multiplied by a factor $\alpha$:
\begin{equation}
  \label{eq:NLLPara_alpha}
  \text{NLL}'(\mu) \approx \frac{\alpha}{2\,\sigma_\mu^2}\, (\mu-\hat{\mu})^2 + C' \ .
\end{equation}
This will make the shape of the parabola around the NLL minimum wider (if $\alpha$\textlt1) or narrower (if $\alpha$\textgt1). Notice that
this does not affect the position of the minimum, from which the parameter values are estimated.

Note that whereas the factor $\alpha$ is an exact correction for the uncertainty estimate of the expected number of decays in the above
example, it is only an approximate correction for the parameters in the NLL. The resulting uncertainties have to be checked by evaluating
the shape of the distribution of parameter estimates in simulated experiments, as will be discussed in Sections~\ref{sec:ana_sim} and
\ref{sec:result_paramEst}.
